\documentclass[10pt]{beamer}

\usetheme[progressbar=foot]{metropolis}
\usepackage{appendixnumberbeamer}

\usepackage{booktabs}
\usepackage[scale=2]{ccicons}

\usepackage{pgfplots}
\usepgfplotslibrary{dateplot}
\pgfplotsset{compat=1.18} 

\usepackage{xspace}
\usepackage{xcolor}

\DeclareMathOperator{\stdev}{stdev}
\DeclareMathOperator{\var}{var}
\DeclareMathOperator{\cov}{cov}
\DeclareMathOperator{\corr}{corr}
\DeclareMathOperator{\prob}{prob}
\DeclareMathOperator{\n}{n}
\DeclareMathOperator{\N}{N}
\DeclareMathOperator{\Cov}{Cov}

\newcommand{\hlf}{\frac{1}{2}}
\newcommand{\bi}{\begin{itemize}}
\newcommand{\ei}{\end{itemize}}
\newcommand{\im}{\item}
\newcommand{\D}{\mathrm{d}}
\newcommand{\E}{\mathrm{e}}
\newcommand{\mye}{\ensuremath{\mathsf{E}}}
\newcommand{\myreal}{\ensuremath{\mathbb{R}}}
\newcommand{\bq}{\begin{equation}}
\newcommand{\eq}{\end{equation}}
\newcommand{\eqdef}{\;\buildrel \text{d{}ef}\over = \;}
\newcommand{\xstar}{\buildrel *\over X}
\newcommand{\pmax}{p^{\text{max}}}
\newcommand{\qmax}{q^{\text{max}}}
\newcommand{\bfr}{\begin{frame}}
\newcommand{\bfrp}{\begin{frame}[plain]}
\newcommand{\efr}{\end{frame}}
\newcommand{\F}{\mathcal{F}}
\newcommand{\FF}{\mathbb{F}}
\newcommand{\ve}{\varepsilon}
\newcommand{\lh}{\hat{\lambda}}
\definecolor{mycolor}{gray}{0.8}
\definecolor{mymaincolor}{rgb}{0.6862745098039216,0.9333333333333333,0.9333333333333333}
\newcommand{\alr}[1]{\textcolor{blue}{#1}}
\definecolor{LightCyan}{rgb}{0.88,1,1}
\newcommand{\yel}{\cellcolor{yellow}}
\newcommand{\blue}{\cellcolor{SkyBlue}}
\newcommand{\gr}{\cellcolor{SpringGreen}}
\newcommand{\pink}{\cellcolor{pink}}
\newcommand{\apr}{\cellcolor{Apricot}}
\newcommand{\tve}{\tilde{\varepsilon}}
\newcommand{\tw}{\tilde{w}}
\newcommand{\ttth}{\tilde{\theta}}
\newcommand{\te}{\tilde{e}}
\newcommand{\ts}{\tilde{s}}
\newcommand{\tx}{\tilde{x}}
\newcommand{\ty}{\tilde{y}}
\newcommand{\tv}{\tilde{v}}
\newcommand{\tp}{\tilde{p}}
\newcommand{\tF}{\tilde{F}}
\newcommand{\tf}{\tilde{f}}
\newcommand{\tZ}{\tilde{Z}}
\newcommand{\ow}{\overline{w}}
\newcommand{\tm}{\tilde{m}}
\newcommand{\tc}{\tilde{c}}
\newcommand{\tz}{\tilde{z}}
\newcommand{\tr}{\widetilde{R}}
\newcommand{\tR}{\widetilde{\mathbf{R}}}
\newcommand{\bms}{\begin{multline*}}
\newcommand{\ems}{\end{multline*}}
\newcommand{\bas}{\begin{align*}}
\newcommand{\eas}{\end{align*}}
\newcommand{\qr}{\mathbb{Q}}
\newcommand{\tX}{\tilde{X}}
\newcommand{\tY}{\tilde{Y}}

\setbeamertemplate{frame footer}{BUSI 521/ECON 505, Spring 2024}

\title{Chapter 7: Representative Investors}

\date{}
\author{Kerry Back\\ 
BUSI 521/ECON 505\\
Rice University}


\begin{document}

\maketitle


\begin{frame}{Equilibrium with Date--0 Consumption}
Assume there is no labor income.
Investors $h=1,\ldots,H$ have endowments of date--0 consumption $\bar{c}_{h0}$ and asset shares $\bar{\theta}_h$.  Assets $i=1,\ldots,n$ have payoffs $\tx_i$.

Take date--0 consumption to be the numeraire (price=1).  An equilibrium is a price vector $p\in \myreal^n$ for assets, a date--0 consumption allocation $(c_{10},\ldots,c_{H0})$ and asset allocations $(\theta_1,\ldots,\theta_H)$ such that
\begin{itemize}
    \item date--0 consumption $c_{h0}$ and portfolio $\theta_h$ are optimal for investor $h$, for all $h$
    \item the date--0 consumption market: $\sum_h c_{h0} = \sum_h \bar{c}_{h0}$
    \item the asset markets clear: $\sum_h \theta_h = \sum_h\bar{\theta}_h$
\end{itemize}
\end{frame}

\begin{frame}{Representative Investor}
    \bi 
    \im There is a representative investor 
    if each asset price vector $p$ that is part of a securities market equilibrium is also part of a securities market equilibrium in the economy in which there is only the representative investor, and the representative investor's endowments are $\bar{c}_0 := \sum_h \bar{c}_{h0}$ and $\bar{\theta} := \sum_h \bar{\theta}_{h0}$.  
    
    \im By the FOC in the representative investor economy, the representative investor's MRS is an SDF.
    \ei
 \end{frame}
 
 \begin{frame}{Plan for Today}
     Assume there is a representative investor with CRRA utility.  Derive
     \begin{itemize}
         \item formula for market return 
         \item formula for risk-free rate
         \item formula for log equity premium (assuming also lognormal consumption growth)
         \item variation of the CAPM
     \end{itemize}
     
Then discuss:
\begin{itemize}
    \item There is a representative investor if the first welfare theorem holds (complete markets or LRT utility with same cautiousness parameter)
    \item With LRT utility for all investors and same cautiousness parameter, the representative investor has the same utility function.
\end{itemize}
 \end{frame}

\section{Equity Premium}\subsection{}
\begin{frame}{Representative Investor with CRRA Utility}
    \bi 
    \im 
    Assume there is a representative investor with utility function
      $$(c_0,c_1) \mapsto u(c_0) + \delta u(c_1)$$
      where
      $$u(c) = \frac{1}{1-\rho}c^{1-\rho}$$
      
      \im Let $c_0$ denote aggregate consumption at date 0, and let $\tilde{c}_1$ denote aggregate consumption at date 1.
      
      \im Then
      $$\delta \left(\frac{\tc_1}{c_0}\right)^{-\rho}$$
      is an SDF.
      \ei 
     \end{frame}

\bfr\frametitle{Market Return}
 \bi 
 \im Assume $\tc_1$ is spanned by the assets.  Its cost is 
$$\mye[\tm\tc_1]= \mye\left[\frac{\delta \tc_1^{-\rho}}{c_0^{-\rho}}\tc_1\right] = c_0\mye\left[\frac{\delta \tc_1^{-\rho}}{c_0^{-\rho}}\cdot\frac{\tc_1}{c_0}\right] = \delta c_0\mye\left[\left(\frac{\tc_1}{c_0}\right)^{1-\rho}\right]$$
 \im The market return is 
 $$\tr_m := \frac{\tc_1 }{ \mye[\tm\tc_1]} = \frac{1}{\delta\mye\left[\left(\frac{\tc_1}{c_0}\right)^{1-\rho}\right]}\cdot \frac{\tc_1}{c_0} := \frac{1}{\nu_1} \cdot \frac{\tc_1}{c_0}$$
 \ei 
 \end{frame}
 
\bfr\frametitle{Risk-Free Return}
 The risk-free return is
$$R_f = \frac{1}{\mye[\tm]} = \frac{1}{\delta\mye[(\tc_1/c_0)^{-\rho}]} := \frac{1}{\nu_0}$$

\end{frame}

\begin{frame}{Log Equity Premium}
$$\frac{\tr_m}{R_f} = \frac{\nu_0}{\nu_1}\cdot \frac{\tc_1}{c_0}$$
So,
$$
\frac{\mye[\tr_m]}{R_f} = \frac{\nu_0\mye[\tc_1/c_0]}{\nu_1} = \  \frac{\mye[(\tc_1/c_0)^{-\rho}]\mye[\tc_1/c_0]}{\mye[(\tc_1/c_0)^{1-\rho}]}
= \frac{\mye[\tc_1]\mye[\tc_1^{-\rho}]}{ \mye[\tc_1^{1-\rho}]}
$$
\end{frame}


\bfr\frametitle{Lognormal Consumption}
\bi 
\im 
 Assume $\log \tc_1 - \log c_0 = \mu + \sigma \tve$ for constants $\mu$ and $\sigma$ and a standard normal $\tve$.
\begin{align*}\tc_1 = c_0\E^{\mu+\sigma\tve} \;&\Rightarrow\; \mye[\tc_1] = c_0\E^{\mu+\sigma^2/2}\\
\tc_1^{-\rho} = c_0^{-\rho}\E^{-\rho\mu-\rho\sigma\tve}\;&\Rightarrow\;\mye[\tc_1^{-\rho}] = c_0^{-\rho}\E^{-\rho\mu+\rho^2\sigma^2/2}\\
\tc_1^{1-\rho} = c_0^{1-\rho}\E^{(1-\rho)\mu + (1-\rho)\sigma\tve}\;&\Rightarrow\; \mye[\tc_1^{1-\rho}] = c_0^{1-\rho}\E^{(1-\rho)\mu+(1-\rho)^2\sigma^2/2}
\end{align*}

 \im This implies
$$\frac{\mye[\tr_m]}{R_f} = \E^{\rho\sigma^2}$$
\im So,
$$\log \mye[\tr_m] - \log R_f = \rho \sigma^2$$
\ei
\end{frame}

\bfr\frametitle{Equity Premium and Risk-Free Rate Puzzles}
\bi
\im To match this model to the historical equity premium, a risk aversion around 50 is required.  Much too high.
\im Using $\rho=10$ and $\delta=0.99$, the model implies a high risk-free rate (12.7\%) and low equity premium ($\mye[\tr_m]-R_f = 1.4\text{\%}$).
\im The historical (U.S.) numbers are around 1\% for the real risk-free rate and 6\% for the equity premium.
\ei
\end{frame}

\section{CAPM Alternative}

\begin{frame}{SDF and Market Return}
    \bi 
\im The market return is 
$$\tr_m = \frac{1}{\nu_1}\cdot \frac{\tc_1}{c_0}$$
\im The SDF is
$$\tm = \delta \left(\frac{\tc_1}{c_0}\right)^{-\rho}$$
\im So, the SDF is
$$\tm = \delta \nu^{-\rho} \tr_m^{-\rho}$$
\ei 
\end{frame}

\begin{frame}{CAPM Alternative}
    \bi 
    \im Risk premia of all assets are
    $$\mye[\tr] - R_f = - R_f \cov(\tr,\tm) = - \delta\nu^{-\rho} R_f \cov(\tr,\tr_m^{-\rho})$$
    
    \im This implies
    $$\mye[\tr] - R_f = \lambda \frac{ \cov(\tr,\tr_m^{-\rho})}{\var(\tr_m^{-\rho})}$$
    for a $\lambda$ that is the same for all assets.  \im So, risk premia depend on betas with respect to $\tr_m^{-\rho}$.
\ei 
    
\end{frame}

\section{When is There a Representative Investor?}\subsection{}

\bfr\frametitle{Social Planner's Problem}\
\bi 
\im 
For each value $w$ of market wealth,  the social planner solves
$$\max \quad \sum_{h=1}^H \lambda_hu_h(w_h) \quad \text{subject to} \quad \sum_{h=1}^H w_h = w$$
\im Let $U(w)$ denote the maximum value.  This is the social planner's utility function.
\im Let $\eta$ denote the Lagrange multiplier (which depends on market wealth $w$).  Then, for all $h$,
$$\lambda_h u_h'(w_h) = \eta$$
\im Also, the social planner's marginal utility (the marginal value of market wealth) is equal to $\eta$.  \im So, for all $h$, we have the envelope result:
$$U'(w) = \lambda_h u_h'(w_h)$$
\im Hence, the social planner's marginal utility is proportional to an SDF.
\ei 
\end{frame}

\bfr\frametitle{Social Planner's Problem with Date--0 Consumption}
\bi 
\im Suppose investor $h$ has utility $u_h(c_{h0}) + \delta_h u_h(c_{h1})$.  
\im The social planner's problem is now separable in dates and in states. 
\ei 
\end{frame}

\begin{frame}[plain] 
    \bi 
\im Given aggregate date--0 consumption $c_{m0}$ and aggregate date--1 consumption $c_{m1}$, the social planner solves 
$$U_0(c_{m0}) := \max \quad \sum_{h=1}^H \lambda_hu_h(c_{h0}) \quad \text{subject to} \quad \sum_{h=1}^H c_{h0} = c_{m0}$$
and
$$U_1(c_{m1}) := \max \quad \sum_{h=1}^H \lambda_h\delta_h u_h(c_{h1}) \quad \text{subject to} \quad \sum_{h=1}^H c_{h1} = c_{m1}$$

\im The envelope theorem tells us that, for all $h$,
$$U_0'(c_{m0}) = \lambda_h u_h'(c_{h0}) \quad \text{and} \quad U_1'(c_{m1}) = \lambda_h \delta_h u_h'(c_{h1})$$
\im So,
$$\frac{U_1'(c_{m1})}{U_0'(c_{m0})} = \frac{\delta_h u_h'(c_{h1})}{u_h'(c_{h0})} = \text{SDF}$$
\ei
\end{frame}

\bfr\frametitle{Common Discount Factors}
\bi 
\im 
If all investors have the same discount factor $\delta$, then we can pull $\delta$ outside the sum in the definition of $U_1$ and see that, as functions, $U_1 = \delta U_0$.  

\im Writing $U=U_0$, an SDF is
$$\frac{\delta U'(\tilde{c}_{m1})}{U'(c_{m0})}$$
\ei
\end{frame}


\bfr\frametitle{Linear Risk Tolerance}
\bi 
\im Suppose all investors have linear risk tolerance $\tau_h(c) = A_h + B c$ with same cautiousness parameter $B\ge 0$.  
\im Then, the social planner's utility functions $U_0$ and $U_1$ have linear risk tolerance with the same cautiousness parameter.
\im 
Example: all investors have CRRA utility with risk aversion $\rho$ and the same discount factor $\delta$.  
\im Then, an SDF is
$$\frac{\delta U'(\tilde{c}_{m1})}{U'(c_{m0})}$$
where
$$U(c) = \frac{1}{1-\rho}c^{1-\rho}$$
\im So, the SDF is
$$\delta \left(\frac{\tilde{c}_{m1}}{c_{m0}}\right)^{-\rho}$$
\ei
\end{frame}

\begin{frame}{Proof of LRT Social Planner in CARA Case}
    \bi 
    \im 
We solved the social planner's problem before and found
$$ w_h =\frac{\tau _h}{\tau }w -\frac{\tau _h}{\tau }\sum_{\ell =1}^H \tau _\ell \log (\lambda_\ell \alpha_\ell ) + \tau _h\log(\lambda_h\alpha_h)$$
which we wrote as $w_h = a_h + b_h w$ with $b_h = \tau_h/\tau$
\im So,
$$U(w) = -\sum_{h=1}^H \lambda_h \E^{-\alpha_h (a_h + b_h w)} = -\sum_{h=1}^H \lambda_h \E^{-\alpha_ha_h}\E^{-\alpha_hb_hw} $$
 \im Moreover,
$$\alpha_h b_hw = \frac{\alpha_h \tau_hw}{\tau} = \frac{w}{\tau} = \alpha w$$
\im So
$$U(w) = -\E^{-\alpha w}\sum_{h=1}^H \lambda_h \E^{-\alpha_ha_h}$$
which is a monotone affine transform of CARA utility.
\ei 
\end{frame}

\end{document}
