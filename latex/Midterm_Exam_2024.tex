\documentclass[letterpaper,english]{article}
\RequirePackage{amsthm,amsmath,amsfonts,amssymb,color,graphicx,babel,natbib,booktabs,geometry,fancyvrb}
%\usepackage{times,amsfonts}%\usepackage[T1]{fontenc}

\usepackage{fancyhdr}
\pagestyle{fancy}
\fancyhead{}
\fancyfoot{}
%\fancyfoot[L]{\copyright Kerry Back, 2012}
\fancyfoot[C]{--\;\;\thepage\;\;-- }
\renewcommand{\footrulewidth}{0.4pt}
\renewcommand{\headrulewidth}{0pt}
\setlength{\headsep}{0in}
\setlength{\footskip}{0.25in}
\setlength{\headheight}{0in}
\setlength{\topmargin}{0in}
\setlength{\textheight}{8.5in}

\providecommand{\e}[1]{\ensuremath{\times 10^{#1}}}
%\renewcommand{\baselinestretch}{1.5}
\newcommand{\inclgr}[2]{\begin{center}
    \includegraphics[scale=#2]{#1}
    \end{center}}
%\newcommand{\ask}[2]{\vskip #1\baselineskip \textsc{#2}}
\newcommand{\ask}[2]{\vskip #1\baselineskip \begin{itemize}\renewcommand{\labelitemi}{\LARGE $\star$} \item  #2\end{itemize}}
\newcommand{\note}[2]{\vskip #1\baselineskip \underline{Note:} #2}
\newcommand{\next}{\vskip\baselineskip}
\setlength{\parindent}{0cm}
\newcommand{\mysection}[1]{\newpage\centerline{\Large \textbf{#1}}\vskip\baselineskip}
\newcommand{\newsection}[1]{\centerline{\Large \textbf{#1}}\vskip\baselineskip}
\newcommand{\mybox}[1]{ \framebox[\textwidth][c]{ \begin{minipage}[c][1.1\height]{0.95\textwidth} #1 \end{minipage} } }

\newcommand{\bi}{\begin{itemize} }
\newcommand{\im}{\item}
\newcommand{\ei}{\end{itemize} }

\newcommand{\D}{\mathrm{d}}
\newcommand{\E}{\mathrm{e}}
\newcommand{\mye}{\ensuremath{\mathsf{E}}}
\newcommand{\myreal}{\ensuremath{\mathbb{R}}}
\newcommand{\mytitle}[1]{\begin{center}
\fbox{\rule[-\baselineskip]{0cm}{2.5\baselineskip}\hspace{5ex} \textbf{#1}\hspace{5ex} }
\end{center}}

\DeclareMathOperator{\var}{var}
\DeclareMathOperator{\stdev}{stdev}
\DeclareMathOperator{\cov}{cov}
\DeclareMathOperator{\corr}{corr}
\DeclareMathOperator{\prob}{prob}
\DeclareMathOperator{\N}{N}
\DeclareMathOperator{\n}{n}

\setlength{\baselineskip}{1.5\baselineskip}

\DeclareMathOperator{\Cov}{Cov}
\newcommand{\pts}[1]{(#1 points)\hspace{2ex}}
\newcommand{\tm}{\widetilde{m}}
\newcommand{\tw}{\widetilde{w}}
\newcommand{\tx}{\widetilde{x}}
\newcommand{\tX}{\widetilde{X}}
\newcommand{\tr}{\widetilde{R}}
\newcommand{\br}{\widetilde{\mathbf{R}}}
\newcommand{\np}{\newpage\noindent This page is intentionally blank.\newpage}
\newcommand{\halfskip}{\vskip 14\baselineskip}

\begin{document}\thispagestyle{empty}

\fontfamily{cmss}\fontseries{sbx}\fontshape{n}\fontsize{12}{18}
\selectfont
\thispagestyle{empty}

\noindent \parbox{3in}{\setlength{\baselineskip}{0.67\baselineskip}
\noindent BUSI 521 /ECON 505\\
\noindent Asset Pricing Theory / Financial Economics\\
\noindent Prof. Kerry Back\\
\noindent Spring 2024}
\hfill
\parbox{1.5in}{\includegraphics[scale=0.9]{Logo.jpg}}
\centerline{MIDTERM EXAM}
\vskip 0.5\baselineskip
\hrule
\vskip 0.5in
This exam is closed book and closed notes.  There are three questions, and they will be equally weighted.  
\vskip\baselineskip
\begin{enumerate}
\item Assume there are three states of the world that are equally likely.  There are two assets with prices $p_1=p_2=1$.  The payoffs of the first asset across the three states of the world are $(1,2,1)$.  The payoffs of the second asset across the three states of the world are $(0,1,3)$.  
\begin{enumerate}
\item Describe the one-dimensional family of state price vectors.
\item Find the SDF that is spanned by the assets.
\end{enumerate}

\item Assume there is a risk-free asset and multiple risky assets with joint normal returns. 
\begin{enumerate}
    \item Derive the optimal portfolio for an investor with CARA utility.
    \item Show that the return of the investor's optimal portfolio is a pricing factor.
\end{enumerate}

\item Use the Bellman equation to derive the optimal portfolio for a log utility investor with an infinite horizon.  You can assume that returns are iid.

\end{enumerate}
\end{document}