\documentclass[xcolor=dvipsnames,10pt]{beamer}

\mode<presentation> {\usetheme{Singapore}}
\usepackage{pgfpages}

%\setbeamercovered{transparent} 
\usepackage[english]{babel}
\usepackage[latin1]{inputenc}
\usepackage{times,amsfonts}
\usepackage[T1]{fontenc}
\setlength{\parskip}{\baselineskip}
% Or whatever. Note that the encoding and the font should match. If T1
% does not look nice, try deleting the line with the fontenc.

%\usecolortheme{sidebartab}
\setbeamertemplate{itemize item}[triangle]



\usepackage{calc}
\usepackage{environ}
\newcommand{\halfmargin}{0.0001\paperwidth}


\RequirePackage{booktabs,colortbl,ulem}

\usepackage{animate}
\RequirePackage{booktabs,colortbl,gensymb}
\setlength{\parskip}{\baselineskip}

\usepackage{calc}
\usepackage{environ}

% \newcommand{\halfmargin}{0.0001\paperwidth}


\NewEnviron{wideframe}[1][]{%
\begin{frame}{#1}
\makebox[\textwidth][c]{
\begin{minipage}{\dimexpr\paperwidth-\halfmargin-\halfmargin\relax}
\BODY
\end{minipage}}
\end{frame}
}


\DeclareMathOperator{\stdev}{stdev}
\DeclareMathOperator{\var}{var}
\DeclareMathOperator{\cov}{cov}
\DeclareMathOperator{\corr}{corr}
\DeclareMathOperator{\prob}{prob}
\DeclareMathOperator{\n}{n}
\DeclareMathOperator{\N}{N}
\DeclareMathOperator{\Cov}{Cov}

\newcommand{\hlf}{\frac{1}{2}}
\newcommand{\bi}{\begin{itemize}}
\newcommand{\ei}{\end{itemize}}
\newcommand{\im}{\item}
\newcommand{\D}{\mathrm{d}}
\newcommand{\E}{\mathrm{e}}
\newcommand{\mye}{\ensuremath{\mathsf{E}}}
\newcommand{\myreal}{\ensuremath{\mathbb{R}}}
\newcommand{\bq}{\begin{equation}}
\newcommand{\eq}{\end{equation}}
\newcommand{\eqdef}{\;\buildrel \text{d{}ef}\over = \;}
\newcommand{\xstar}{\buildrel *\over X}
\newcommand{\pmax}{p^{\text{max}}}
\newcommand{\qmax}{q^{\text{max}}}
\newcommand{\bfr}{\begin{frame}}
\newcommand{\bfrp}{\begin{frame}[plain]}
\newcommand{\efr}{\end{frame}}
\newcommand{\F}{\mathcal{F}}
\newcommand{\FF}{\mathbb{F}}
\newcommand{\ve}{\varepsilon}
\newcommand{\lh}{\hat{\lambda}}
\definecolor{mycolor}{gray}{0.8}
\definecolor{mymaincolor}{rgb}{0.6862745098039216,0.9333333333333333,0.9333333333333333}
\newcommand{\alr}[1]{\textcolor{blue}{#1}}
\definecolor{LightCyan}{rgb}{0.88,1,1}
\newcommand{\yel}{\cellcolor{yellow}}
\newcommand{\blue}{\cellcolor{SkyBlue}}
\newcommand{\gr}{\cellcolor{SpringGreen}}
\newcommand{\pink}{\cellcolor{pink}}
\newcommand{\apr}{\cellcolor{Apricot}}
\newcommand{\tve}{\tilde{\varepsilon}}
\newcommand{\tw}{\tilde{w}}
\newcommand{\ttth}{\tilde{\theta}}
\newcommand{\te}{\tilde{e}}
\newcommand{\ts}{\tilde{s}}
\newcommand{\tx}{\tilde{x}}
\newcommand{\ty}{\tilde{y}}
\newcommand{\tv}{\tilde{v}}
\newcommand{\tp}{\tilde{p}}
\newcommand{\tF}{\tilde{F}}
\newcommand{\tf}{\tilde{f}}
\newcommand{\tZ}{\tilde{Z}}
\newcommand{\ow}{\overline{w}}
\newcommand{\lb}{\left[}
\newcommand{\rb}{\right]}
\newcommand{\lp}{\left(}
\newcommand{\rp}{\right)}
\newcommand{\tm}{\tilde{m}}
\newcommand{\tc}{\tilde{c}}
\newcommand{\tz}{\tilde{z}}
\newcommand{\str}[1]{\textcolor{blue}{\sout{#1}}}
\newcommand{\tr}{\widetilde{R}}
\newcommand{\tR}{\widetilde{\mathbf{R}}}
\newcommand{\bms}{\begin{multline*}}
\newcommand{\ems}{\end{multline*}}
\newcommand{\bas}{\begin{align*}}
\newcommand{\eas}{\end{align*}}
\newcommand{\qr}{\mathbb{Q}}
\newcommand{\IMAGES}{/home/kerry/Dropbox/Images}
\newcommand{\tX}{\tilde{X}}
\newcommand{\tY}{\tilde{Y}}

\author{\vskip 0.5in \small Kerry Back \\BUSI 521--ECON 505\\ Rice University \\Spring 2022}
%\institute{Rice University\\ Spring 2019}
\date[]





\newcommand{\tu}{\tilde{u}}
\begin{document}
\title{\vskip 0.5in Day 21}
\subtitle{Dynamic Programming}

\begin{frame}
  \titlepage
\end{frame}

\section{Capital Investment Problem}\subsection{}

\bfr\frametitle{Optimal Capital Investment}
A firm combines capital $k$ and labor $\ell$ to produce output $q=k^\alpha \ell^\psi$ with $\alpha+\psi \le 1$.  The wage rate $w$ is constant.

The firm is a monopolist facing a constant elasticity demand curve:
$$\frac{\D \log q}{\D \log p} = \gamma \quad \Rightarrow \quad p = aq^{-1/\gamma}$$
for a constant (soon to be stochastic process) $a$.
We call $pq-w\ell$ the firm's operating cash flow.

Capital depreciates over time.  Also, adjusting the capital stock is costly.  To adjust by $\Delta k$ costs $\Delta k$ + (1/2)(\Delta k)^2b$.  We call $-\Delta k - (1/2)(\Delta k)^2 b$ the firm's investment cash flow.  $\Delta k$ can be negative.

Total cash flow is operating cash flow plus financing cash flow.
\end{frame}

\begin{frame}
We formulate the problem in continuous time.  The firm chooses an investment rate $I_t$ and capital $K_t$ evolves as
$$\D K_t = -\delta K_t\,\D t + I_t\,\D t$$

Given the capital stock $K$ at time $t$, the firm chooses the labor input $L_t$ to maximize the operating cash flow $pq-w\ell$, i.e., 
$$L_t = \text{argmax}_\ell \; a\left(K_t^\alpha \ell^\psi\right)^{1-1/\gamma} - w\ell$$

There is random time-varying demand: the constant $a$ is a stochastic process $A$.  The firm chooses an investment policy to maximize
$$\mye\int_0^\infty M_t \left[\text{Operating Cash Flow} - I_t - (1/2)bI_t^2\right]\,\D t$$
where $M$ is the SDF process.
    
\end{frame}

\begin{frame}
    Assume the risk-free rate is constant.  By switching to the risk-neutral probablility, we can formulate the firm's objective as
    $$\mye^*\int_0^\infty \E^{-rt} \left[\text{Operating Cash Flow} - I_t - (1/2)bI_t^2\right]\,\D t$$
    
    Assume $A$ is a geometric Brownian motion (hence Markov) under the risk-neutral probability.  The state variables are $A_t$ and $K_t$.  
    
    The problem is time-stationary, so the firm's value function is of the form $J(A_t,K_t)$.
    
    \end{frame}
    
    \section{Solution}\subsection{}
    \begin{frame}
        What sort of solution can we find?
        
        We want to find the optimal investment rate $I_t$ as a function of $A_t$ and $K_t$.
        
        We can show that the optimum satisfies
        $$I^*_t = \frac{1}{b}\left[\frac{\partial J(A_t,K_t)}{\partial K_t} - 1\right]$$
        
        Can we compute $\partial J/\partial K$?  
        
        Not analytically.  But, we can show that $J$ satisfies a partial differential equation (PDE), which can be solved numerically.
        
    \end{frame}


\begin{frame}
The PDE is the Bellman equation, called Hamilton-Jacobi-Bellman (HJB) equation in continuous time.

The HJB equation is
$$\max_{\text{controls}} \bigg\{\text{current reward} - rJ + \text{drift of J}\bigg\} = 0$$
The optimal controls must achieve the max here.

Let $\pi(A_t,K_t)$ denote the operating cash flow maximized over $L_t$.  Then the current reward is
$$\pi(A_t,K_t) - I_t - \frac{1}{2}bI^2_t$$
The control is I_t$.

The drift of $J$ is
$$\frac{\partial J}{\partial A}\text{drift of A} + \frac{\partial J}{\partial k}(-\delta K + I)+ \frac{1}{2}\frac{\partial^2 J}{\partial A^2}(\D A)^2$$
dropping the $\D t$ from $(\D A)^2$.
\end{frame}

\begin{frame}
Substituting into the HJB equation and dropping terms that do not depend on $I$, the maximization problem is
$$\max_I \bigg\{-I - \frac{1}{2}bI^2 + \frac{\partial J}{\partial K}I\bigg\}$$

The solution is
$$I = \frac{\partial J/\partial K - 1}{b}$$

Write $J_k$ for the partial derivative.  So, the solution is $I=(J_k-1)/b$.

At the max $I=(J_k-1)/b$, the HJB equation tells us
$$\text{current reward} - rJ + \text{drift of J} = 0$$
This is a PDE in $J$. 
\end{frame}

\begin{frame}
 

Write $J_a$ for $\partial J/\partial A$ and $J_{aa}$ for the second partial.  Then, suppressing the arguments $(A,K)$, we have
\begin{multline*}
    \pi - \frac{J_k-1}{b} - \frac{1}{2}b\left(\frac{J_k-1}{b}\right)^2 - rJ \\+J_a \text{drift of A} + \left(\frac{J_k-1}{b} - \delta K\right)J_k + \frac{1}{2}J_{aa}(\D A)^2 = 0
\end{multline*}
This is an equation that must hold for $A>0$ and $K >0$, with boundary conditions at 0 and $\infty$.
\end{frame}

\section{HJB Equation}
\begin{frame}{HJB Equation}
Discrete-time, no uncertainty, control $c$, state variable $x$, and state evolution equation $x' = f(x,c)$:
$$J(x) = \max_c \; u(c) + \E^{-\delta} J(x')$$
With uncertainty and expected utility
$$J(x) = \max_c \; u(c) + \E^{-\delta} \mye[J(x') \mid x,c]$$
Equivalently,
$$0 = \max_c \; u(c) + \mye[J(x')-J(x) \mid x,c] - (1-\E^{-\delta})\mye[J(x') \mid x,c]$$
In continuous time,
$$0 = \max_c \; u(c) + \text{drift of J} - \delta J$$
\end{frame}

\begin{frame}{Another Derivation}
    Let $C\,$ control process, $X=\,$ state process, $u=\,$ utility function, $\delta=\,$ discount rate ($r$ in capital investment problem), $\mye=\,$ appropriate expectation.  Problem is
    $$\max_C\; \mye \int_0^\infty \E^{-\delta t}u(C_t,X_t)\,\D t$$
    where evolution of $X$ depends on current $X$ and $C$.
    
    Let $J$ denote value function.
    Suppose there is an optimal policy $C^*$, which induces the state process $X^*$, so
      $$J(X_0) = \mye \int_0^\infty \E^{-\delta t}u(C^*_t,X^*_t)\,\D t$$
      \end{frame}
      
      \begin{frame}
    Bellman's principle of optimality states that, for any $s$,
    $$J(X_0) = \mye \left[\int_0^s \E^{-\delta t}u(C^*_t,X^*_t)\,\D t + \max_C\; \mye_s\int_s^\infty \E^{-\delta t}u(C_t,X_t)\,\D t\right]$$
    In other words, if a plan is optimal, then if we re-optimize at some later date, we cannot improve upon the original plan.
    
    By changing from $t$ to $a = t-s$, we obtain 
    \begin{align*}
        \mye_s\int_s^\infty \E^{-\delta t}u(C_t,X_t)\,\D t &= \mye_s\int_0^\infty \E^{-\delta (s+a)}u(C_{s+a},X_{s+a})\,\D a\\
        &= \E^{-\delta s} \mye_s\int_0^\infty \E^{-\delta a}u(C_{s+a},X_{s+a})\,\D a
    \end{align*}
    and by the Markov property of $X$, the distribution of $X_{s+a}$ given $X_s=x$ is the same as the distribution of $X_a$ given $X_0=x$.  Therefore,
     
    $$J(X_0) = \mye \left[\int_0^s \E^{-\delta t}u(C^*_t,X^*_t)\,\D t + \E^{-\delta s}J(X_s)\right]$$
\end{frame}

\begin{frame}
The equation
$$J(X_0) = \mye \left[\int_0^s \E^{-\delta t}u(C^*_t,X^*_t)\,\D t + \E^{-\delta s}J(X_s)\right]$$
holds for every $s$.

So, the expectation is constant with respect to $s$.  It is not very much extra work to show that, in fact, 
        $$\int_0^s \E^{-\delta t}u(C^*_t,X^*_t)\,\D t + \E^{-\delta s}J(X_s)$$
    is a martingale.  
 
 A martingale has no drift.  Calculate the drift and set to zero:
$$\E^{-\delta s}u(C^*_s,X^*_s) + \E^{-\delta s}\text{drift of J} - \delta \E^{-\delta s}J = 0$$
Multiply by $\E^{\delta s}$ to cancel the $\E^{-\delta s}$ factor:
$$u(C^*_s,X^*_s) + \text{drift of J} - \delta J = 0$$   
\end{frame}

\end{document}


\im 
 $X_{t+1}$ denotes the random state (node) at date $t\!+\!1$, the distribution of which may depend on the decision~$\pi$ and the state~$x$ at date~$t$.
 \im $X$ can have two components, one being exogenous (a Markov process) and one being endogenous (for example, wealth).
 \im $V_t$ is the maximum expected sum of utilities at dates $t, t+1, \ldots$ measured in terms of date--0 utility, i.e., discounted to date~0.
 \im We don't always assume an infinite-horizon stationary model, because we might, for example, want to study maximizing the expected utility of terminal wealth.  
 \ei  
\end{frame}

\begin{frame}{Deriving the Bellman Equation}

\begin{align*}
V_t \ = \ \max_{\text{plan}} \mye_t \sum_{s=t}^\infty u_s \ & = \ \max_{\text{plan}} \left\{ u_t + \mye_t \sum_{s=t+1}^\infty u_s \right\}\\
& = \ \max_{\text{plan at $t$}} \left\{ u_t + \max_{\text{plan at $t+1$\,\ldots }}\mye_t \sum_{s=t+1}^\infty u_s \right\}\\
& = \ \max_{\text{plan at $t$}} \left\{ u_t + V_{t+1} \right\}
\end{align*}
\end{frame}

\bfr\frametitle{Discounting to Date $t$ Instead of Date 0}
\bi
\im Suppose $u_t(x,\pi) = \E^{-\delta t} u(x,\pi)$ 
\bi
\im $\E^{-\delta}$ is a number less than 1, and we raise it to the $t$--th power to discount.
\ei
\im As stated before, $V_t$ is the maximum expected sum of utilities at dates $t, t+1, \ldots$ measured in terms of date--0 utility, i.e., discounted to date~0.
\im We can measure in terms of date--$t$ utility by reversing the discounting from $t$ to 0 as
$$J_t(x) \ = \ \E^{\delta t}V_t(x)$$
\im Multiplying by $\E^{\delta t}$ in the Bellman equation gives: 
$$
J_{t}( x) \ = \ \max_{\pi} \bigg\{u(x,\pi) + \E^{-\delta} \mye[J_{t+1}(X_{t+1})\mid X_t=x]\bigg\}
$$
\ei
\end{frame}

\bfr\frametitle{Infinite Horizon}
\bi
\im If there is an infinite horizon, and everything is stationary, then $J$ does not depend on $t$.  
\im So the Bellman equation is: 
$$
J( x) \ = \ \max_{\pi} \bigg\{u(x,\pi) + \E^{-\delta} \mye[J(X_{1})\mid X_0=x]\bigg\}
$$

\ei
\end{frame}

\section{Continuous Time}\subsection{}

\bfr\frametitle{Continuous Time}
\bi
\im Let's go back to the `discounted-to-time-zero' value function.  Substituting $x=X_t$, the Bellman equation is
$$V_t(X_t) \ = \ \max_{\pi} \bigg\{u_{t}(X_t,\pi) + \mye[V_{t+1}(X_{t+1})\mid X_t]\bigg\}$$
\im Define 
$$\Delta V_{t+1}(X_t,X_{t+1}) \ = \ V_{t+1}(X_{t+1}) - V_t(X_t)$$

\im Then, the Bellman equation is
$$0 \ = \ \max_{\pi} \bigg\{u_{t}(X_t,\pi) + \mye[\Delta V_{t+1}(X_t,X_{t+1})\mid X_t]\bigg\}$$
\im The continuous-time Bellman equation is the same, replacing $u$ by $u\,\D t$ and $\Delta V$ with $\D V$:
$$0 \ = \ \max_{\pi} \bigg\{u_{t}(X_t,\pi)\,\D t + \mye[d V]\bigg\}$$
\ei
\end{frame}


\bfr\frametitle{Discounting to Date $t$ Instead of Date 0}
\bi
\im Suppose again that $u_t = \E^{-\delta t} u$. As before, we can measure value in terms of date--$t$ utility by reversing the discounting from $t$ to 0 as $J_t  = \E^{\delta t}V_t$.
\im Multiplying by $\E^{\delta t}$ in the Bellman equation gives: 
$$0 \ = \ \max_{\pi} \bigg\{u\,\D t + \mye[\E^{\delta t}\D V]\bigg\}$$
\im From It\^o's formula (product rule), we have
$$\D J = \delta \E^{\delta t}V\,\D t + \E^{\delta t}\, \D V$$
So,
$$\E^{\delta t} \D V  \ = \ \D J - \delta \E^{\delta t}V\,\D t \ = \ \D J - \delta J\,\D t$$
\im So, the continuous-time Bellman equation is
$$ 0 \ = \ \max_{\pi} \bigg\{u \,\D t - \delta J\,\D t + \mye[\D J]\bigg\} $$
\ei
\end{frame}

\end{document}

%%%%%%%%%%%%%%%%%%%%%%%%%%%%%%%%%%%%%

