\documentclass[10pt]{beamer}

\usetheme[progressbar=foot]{metropolis}
\usepackage{appendixnumberbeamer}

\usepackage{booktabs}
\usepackage[scale=2]{ccicons}

\usepackage{pgfplots}
\usepgfplotslibrary{dateplot}
\pgfplotsset{compat=1.18} 

\usepackage{xspace}
\usepackage{xcolor}

\DeclareMathOperator{\stdev}{stdev}
\DeclareMathOperator{\var}{var}
\DeclareMathOperator{\cov}{cov}
\DeclareMathOperator{\corr}{corr}
\DeclareMathOperator{\prob}{prob}
\DeclareMathOperator{\n}{n}
\DeclareMathOperator{\N}{N}
\DeclareMathOperator{\Cov}{Cov}

\newcommand{\hlf}{\frac{1}{2}}
\newcommand{\bi}{\begin{itemize}}
\newcommand{\ei}{\end{itemize}}
\newcommand{\im}{\item}
\newcommand{\D}{\mathrm{d}}
\newcommand{\E}{\mathrm{e}}
\newcommand{\mye}{\ensuremath{\mathsf{E}}}
\newcommand{\myreal}{\ensuremath{\mathbb{R}}}
\newcommand{\bq}{\begin{equation}}
\newcommand{\eq}{\end{equation}}
\newcommand{\eqdef}{\;\buildrel \text{d{}ef}\over = \;}
\newcommand{\xstar}{\buildrel *\over X}
\newcommand{\pmax}{p^{\text{max}}}
\newcommand{\qmax}{q^{\text{max}}}
\newcommand{\bfr}{\begin{frame}}
\newcommand{\bfrp}{\begin{frame}[plain]}
\newcommand{\efr}{\end{frame}}
\newcommand{\F}{\mathcal{F}}
\newcommand{\FF}{\mathbb{F}}
\newcommand{\ve}{\varepsilon}
\newcommand{\lh}{\hat{\lambda}}
\definecolor{mycolor}{gray}{0.8}
\definecolor{mymaincolor}{rgb}{0.6862745098039216,0.9333333333333333,0.9333333333333333}
\newcommand{\alr}[1]{\textcolor{blue}{#1}}
\definecolor{LightCyan}{rgb}{0.88,1,1}
\newcommand{\yel}{\cellcolor{yellow}}
\newcommand{\blue}{\cellcolor{SkyBlue}}
\newcommand{\gr}{\cellcolor{SpringGreen}}
\newcommand{\pink}{\cellcolor{pink}}
\newcommand{\apr}{\cellcolor{Apricot}}
\newcommand{\tve}{\tilde{\varepsilon}}
\newcommand{\tw}{\tilde{w}}
\newcommand{\ttth}{\tilde{\theta}}
\newcommand{\te}{\tilde{e}}
\newcommand{\ts}{\tilde{s}}
\newcommand{\tx}{\tilde{x}}
\newcommand{\ty}{\tilde{y}}
\newcommand{\tv}{\tilde{v}}
\newcommand{\tp}{\tilde{p}}
\newcommand{\tF}{\tilde{F}}
\newcommand{\tf}{\tilde{f}}
\newcommand{\tZ}{\tilde{Z}}
\newcommand{\ow}{\overline{w}}
\newcommand{\tm}{\tilde{m}}
\newcommand{\tc}{\tilde{c}}
\newcommand{\tz}{\tilde{z}}
\newcommand{\tr}{\widetilde{R}}
\newcommand{\tR}{\widetilde{\mathbf{R}}}
\newcommand{\bms}{\begin{multline*}}
\newcommand{\ems}{\end{multline*}}
\newcommand{\bas}{\begin{align*}}
\newcommand{\eas}{\end{align*}}
\newcommand{\qr}{\mathbb{Q}}
\newcommand{\tX}{\tilde{X}}
\newcommand{\tY}{\tilde{Y}}
\newcommand{\lb}{\left[}
\newcommand{\lp}{\left(}
\newcommand{\rb}{\right]} 
\newcommand{\rp}{\right)} 

\setbeamertemplate{frame footer}{BUSI 521/ECON 505/ECON 505}

\title{Chapter 2: Portfolio Choice}

\date{}
\author{Kerry Back\\ 
BUSI 521/ECON 505\\
Rice University}


\begin{document}

\maketitle

\begin{frame}{Simplest Problem}
    \bi
    \im Single risky asset, price $p$ per share at date 0, price $\tx$ per share at date 1.
    \im Risk-free asset with interest rate $r_f$.
    \im Investor has $w_0$ to invest.  Allocates between risk-free and risky assets.  \im Let $\theta=$ number of shares of risky asset.  Then $w_0-p\theta$ is invested risk-free (this could be negative, meaning borrowing).
    \im $\theta$ is chosen to maximize
    $$\mye\big[u\big(\theta \tx + (w_0-p\theta)(1+r_f)\big)\big]$$
    \im FOC is
    $$\mye\big[u'\big(\theta^*\tx + (w_0-p\theta^*)(1+r_f)\big)\big\{\tx - p(1+r_f)\big\}\big] = 0$$
    \ei
\end{frame}

\begin{frame}{More on the FOC}
    \bi 
    \im Date-1 wealth is
    $$ \tw^* := \theta^*\tx + (w_0-p\theta^*)(1+r_f)$$
    \im Divide the FOC by p.  Set $\tR = \tx / p$.  This is the (gross) return on the risky asset, meaning 1 + rate of return.
    \im Set $R_f = 1 + r_f$.  This is the (gross) risk-free return.
    \im The FOC is
    $$ \mye[u'(\tw^*)\{\tR - R_f\}] = 0$$
    \pause 
    \im In words, marginal utility at the optimum is orthogonal to the excess return.
    \ei
\end{frame}

\begin{frame}{Preview of Chapter 3}
    \bi 
    \im Set $\tm = u'(\tw^*)$.  The FOC is $\mye[\tm(\tR-R_f)] = 0$.
    \im By the definition of covariance, 
    $$\mye[\tm(\tR-R_f)] = \mye[\tm]\mye[\tR-R_f] + \cov(\tm, \tR-R_f)$$
    \im So, the risk premium is
    $$\mye[\tR - R_f] = - \frac{1}{\mye[\tm]} \cov(\tm, \tR)$$
    \im What sign should the covariance with marginal utility have?
    \ei
\end{frame}

\bfr\frametitle{Notation}
\bi
\im Single consumption good at each of two dates 0 and 1
\im Date--0 wealth $w_0$ (in units of consumption good)
\im Assets
\bi
\im Assets $i=1,\ldots, n$
\im Date--0 prices $p_i$ (in units of consumption good)
\im Date--1 payoffs $\tx_i$ (in units of consumption good)
\ei
\im Returns
\bi
\im Returns $\tr_i = \tx_i/p_i$ (assuming $p_i>0$)
\im Rates of return $(\tx - p_i)/p_i = \tr_i - 1$
\im If there is a risk-free asset ($\tx$ constant) then return is $R_f$
\ei
\im Portfolios
\bi
\im $\theta_i=\,$ number of shares held in portfolio
\im $\phi_i = \theta_i p_i = \,$ units of consumption good invested 
\im $\pi_i = \theta_ip_i/w_0=\,$ fraction of wealth invested 
\ei
\ei
\end{frame}

\bfr\frametitle{Portfolio Choice Problem}
\bi
\im Choose $\theta_1, \ldots, \theta_n$ to 
$$\max \; \mye\lb u\lp \sum_{i=1}^n \theta_i\tx_i \rp \rb \quad \text{subject to} \quad \sum_{i=1}^n p_i \theta_i = w_0\,.$$
\im Choose $\phi_1, \ldots, \phi_n$ to 
$$\max \;\mye\lb u\lp \sum_{i=1}^n \phi_i\tr_i \rp \rb \quad \text{subject to} \quad \sum_{i=1}^n \phi_i = w_0\,.$$
\im Choose $\pi_1, \ldots, \pi_n$ to 
$$\max \;\mye\lb u\lp w_0\sum_{i=1}^n \pi_i\tr_i \rp \rb \quad \text{subject to} \quad \sum_{i=1}^n \pi_i = 1\,.$$
\ei
\end{frame}

\bfr\frametitle{Comments}
\bi
\im Short sales are allowed ($\theta_i <0$)
\im There are no margin requirements.  
\bi
\im In the U.S. stock market, an investor with \$100 cash can only buy \$200 of stock (borrowing \$100).  
\im In our formulation, there are no limits on borrowing, except that $\sum \theta_i \tx_i$ must be in the domain of $u(\cdot)$---for example, positive if $u= \log$.
\im In real markets, collateral (margin) also has to be posted against short sales, but we do not require that in our formulation.
\ei
\im Here, we take date--0 consumption and investment as given and optimize over the portfolio.  We can also optimize over date--0 consumption and investment.
\im We can sometimes allow for other non-portfolio income $\ty$ at date--1 (for example, labor income).
\ei
\end{frame}

\bfr\frametitle{First-Order Condition}
\bi
\im Lagrangean:
$$\mye\lb u\lp \sum_{i=1}^n \theta_i\tx_i \rp \rb - \lambda \lp \sum_{i=1}^n p_i \theta_i - w_0 \rp$$
\im Assume interior optimum and assume can interchange differentiation and expectation to obtain
$$(\forall \, i) \quad \mye\lb u'\lp \sum_{i=1}^n \theta_i\tx_i \rp \tx_i \rb = \lambda p_i$$
\im If $p_i > 0$,
$$\mye\lb u'\lp \sum_{i=1}^n \theta_i\tx_i \rp \tr_i \rb = \lambda $$
\ei
\end{frame}

\bfr\frametitle{First-Order Condition cont.}
\bi
\im If $p_i>0$ and $p_j>0$,
$$\mye\lb u'\lp \sum_{i=1}^n \theta_i\tx_i \rp (\tr_i -\tr_j)\rb = 0 $$
\im In words: marginal utility at the optimal wealth is orthogonal to excess returns.
\bi
\im A return is the payoff of a unit-cost portfolio.  
\im An excess return is the payoff of a zero-cost portfolio (for example, a difference of returns).
\ei 
\im Why?  Investing a little less in asset $j$ and a little more in asset~$i$ (or the reverse) cannot increase expected utility at the optimum.
\ei
\end{frame}

\section{Some Results for One Risky Asset}
\subsection{}

\bfr\frametitle{Go Long if Risk Premium is Positive}
\bi
\im Let $\phi=\,$ amount invested in risky asset, so $w_0-\phi$ is invested in risk-free asset.  Let $\mu=\mye[\tr]$ and $\sigma^2 = \var(\tr)$.
\im Date--1 wealth is 
$$\tw = (w_0-\phi)R_f + \phi \tr = w_0 R_f + \phi (\tr - R_f)$$
\im We will show: $\mu > R_f \Rightarrow\phi^*>0$ (by symmetry, $\mu<R_f \Rightarrow \phi^*<0$).
\ei
\end{frame}

\bfr\frametitle{Proof}
We want to compare $\mye[u(w_0R_f + \phi(\tr-R_f)]$ to $u(w_0R_f)$.  

Define
$\ow = w_0R_f + \phi(\mu-R_f)$ and $\tve = \phi(\tr-\mu)$, so
$$w_0R_f + \phi(\tr-R_f) = \ow + \tve\,.$$
Define $\pi$ by
$$u(\ow - \pi) = \mye[u(\ow + \tve)]\,.$$
The variance of $\tve$ is $\phi^2\sigma^2$, so by second-order risk aversion,
$$\pi \approx \frac{1}{2}\alpha(\ow)\phi^2\sigma^2 < (\mu-R_f)\phi$$
when $\phi>0$ and small, so
$$u(\ow-\pi) > u(\ow - (\mu-R_f)\phi) = u(w_0R_f)$$

\end{frame}

\bfr\frametitle{DARA Implies Risky Asset is a Normal Good}
\bi
\im Normal good: demand rises when income (wealth) rises.  Inferior: demand falls when income (wealth) rises.
\im A single risky asset with $\mu>R_f$ is a normal good if the investor has decreasing absolute risk aversion.
\im Proof:  The FOC is
$$\mye[u'(\tw)(\tr-R_f)] = 0$$ 
Differentiate it:
\begin{align*}
0 &= \frac{\D }{\D w_0} \mye[u'(w_0R_f + \phi(\tr-R_f))(\tr-R_f)]\\
&= \mye[u''(\tw)\{R_f + \phi'(w_0)(\tr-R_f)\}(\tr-R_f)]
\end{align*}
Rearrange as
$$\phi'(w_0) = - \frac{R_f \mye[u''(\tw)(\tr-R_f)]}{\mye[u''(\tw)(\tr-R_f)^2]}\,.$$
Can show: DARA $\,\Rightarrow\,$ $\phi' >0$.
\ei
\end{frame}

\bfr\frametitle{CARA-Normal with Single Risky Asset}
Assume CARA utility $\mye[-\E^{-\alpha \tw}]$.  Assume $\tr\sim\,$ normal $(\mu,\sigma)$.  Then $\tw$ is normally distributed.  

Recall: If $\tx$ is normally distributed with mean $\mu_x$ and std dev $\sigma_x$, then
$$\mye[\E^{\tx}] = \E^{\mu_x + \sigma_x^2/2}$$

Given an investment $\phi$ in the risky asset, $-\alpha\tw$ is normal with mean $-\alpha w_0R_f - \alpha \phi(\mu-R_f)$ and std dev $\alpha\phi\sigma$.  Hence,
$$\mye[-\E^{-\alpha \tw}] = -\E^{-\alpha [w_0R_f + \phi(\mu-R_f) - \alpha\phi^2\sigma^2/2]}$$
Thus,
$$w_0R_f + \phi(\mu-R_f) - \alpha\phi^2\sigma^2/2$$
is the certainty equivalent (mean minus one-half risk aversion times variance).
\end{frame}



\bfr\frametitle{CARA-Normal cont.}
Optimal portfolio maximizes the certainty equivalent.  Therefore, the optimum is
$$\phi^* = \frac{\mu-R_f}{\alpha \sigma^2}$$

The optimal fraction of wealth to invest is
$$\pi^* = \frac{\mu-R_f}{(\alpha w_0)\sigma^2}$$

Usually assume $\alpha w_0$ is between 1 and 10.
\end{frame}


\section{Multiple Risky Assets}
\subsection{}

\bfr{Portfolio Mean and Variance}
\bi
\im $\tR=\,$ $n$--vector of risky asset returns
\im $\mu=\,$ $n$--vector of expected returns
\im $\phi=\,$ $n$--vector of investments in consumption good units
\im $\pi = (1/w_0)\phi$
\im $\iota=\,$ $n$--vector of 1's
\im $\Sigma = \,$ $n \times n$ covariance matrix, $\Sigma_{ij} = \cov(\tr_i,\tr_j)$
$$\Sigma = \mye[(\tr-\mu)(\tr-\mu)']$$
\im date--1 wealth $\tw = w_0R_f + \phi'(\tR - R_f\iota)$
\im expected wealth $\ow = w_0R_f + \phi'(\mu - R_f\iota)$
\im variance of wealth $\, = \phi'\Sigma\phi$.  Proof:
$$\mye[(\tw - \ow)^2] = \mye[\{\phi'(\tR - \mu)\}^2] = \mye[\phi'(\tr-\mu)(\tr-\mu)'\phi] = \phi'\Sigma\phi$$
\ei
\end{frame}

\bfr\frametitle{Diversification}
\bi
\im Portfolio variance is
$$\pi'\Sigma\pi = \sum_{i=1}^n \pi_i^2 \var(\tr_i) + 2 \sum_{i=1}^n \sum_{j=i+1}^n \pi_i\pi_j \cov(\tr_i,\tr_j)$$
\im We can generally make 
$\sum_{i=1}^n \pi_i^2 \var(\tr_i)$
small by diversifying, if there are many assets.
\im Suppose for example that the risky assets are uncorrelated and have the same variance $\sigma^2$ ($\Sigma = \sigma^2I$).  Then
$$\pi'\Sigma\pi = \sigma^2 \sum_{i=1}^n \pi_i^2$$
Among portfolios fully invested in risky assets ($\pi_i$ sum to 1), this variance is minimized at $\pi_i=1/n$ and
$$\pi'\Sigma\pi = \sigma^2 \sum_{i=1}^n \lp \frac{1}{n}\rp ^2 = \frac{\sigma^2}{n} \rightarrow 0 \quad \text{as $n \rightarrow \infty$}$$
\ei
\end{frame}

\bfr\frametitle{CARA-Normal with Multiple Risky Assets}
\bi
\im Certainty equivalent is mean minus one-half risk aversion times variance:
$$w_0R_f + \phi'(\mu-R_f\iota) - \frac{1}{2}\alpha \phi'\Sigma\phi$$
\im FOC is
$$\mu-R_f\iota - \alpha\Sigma\phi = 0\,.$$
\im Optimum is
$$\phi^* = \frac{1}{\alpha}\Sigma^{-1}(\mu-R_f\iota)$$
Note no wealth effects.
\im Similar form to single risky asset case.  Optimum investment in each asset depends on its covariances with other assets unless $\Sigma$ is diagonal.
\ei
\end{frame}

\section{Wealth Expansion Paths}
\subsection{} 

\begin{frame}{Portfolio Return}
   \bi 
   \im With a risk-free asset, date-1 wealth is 
$$\sum_i \phi \tR_i + \left(w_0- \sum_i \phi_i\right)R_f$$
\im Divide by $w_0$ and write $\pi_i = \phi_i/w_0$.
\im Date-1 wealth is
$$w_0 \left[\sum_i \pi \tR_i + \left(1- \sum_i \pi_i\right)R_f\right] := w_0 \tR_p$$
\im In words, initial wealth times the (gross) portfolio return.
    \ei
\end{frame}

\begin{frame}{How does a Log Utility Investor's Portfolio Depend on Wealth?}
    \bi 
    \im Utility is
    $$\log (w_0\tR_p) = \log w_0 + \log \tR_p = \log w_0 + \log\left(\sum_i \pi \tR_i + \left(1- \sum_i \pi_i\right)R_f\right)$$
     \im The optimal portfolio maximizes expected log of portfolio return and is independent of initial wealth.
    \im Optimal dollar investments are proportional to initial wealth: $\phi_i^* = w_0 \pi_i^*$.
    \im Optimal shares are also proportional to initial wealth: $\theta_i^* = \phi_i^*/p_i = w_0 \pi_i^*/p_i$.
    \ei
\end{frame}

\begin{frame}{Power Utility}
    \bi
    \im Utility is
    $$\frac{1}{1-\rho} (w_0 \tR_p)^{1-\rho} = w_0^{1-\rho} \times \frac{1}{1-\rho} \tR_p^{1-\rho}$$
    \im So $w_0^{1-\rho}$ is a positive constant that multiplies the expected utility of the portfolio return.
    \im Optimal portfolio $\pi^*$ maximizes expected utility of portfolio return and is independent of initial wealth.
    \im Optimal dollar investments are proportional to initial wealth: $\phi_i^* = w_0 \pi_i^*$.
    \im Optimal shares are also proportional to initial wealth: $\theta_i^* = \phi_i^*/p_i = w_0 \pi_i^*/p_i$.
\ei \end{frame}


\begin{frame}{CARA Utility} 
    \bi 
    \im Stick with \$ investments.  Date-1 wealth is
    $$\sum_i \phi \tR_i + \left(w_0- \sum_i \phi_i\right)R_f = w_0R_f + \phi'(\tR - R_f\iota)$$
    \im Here, $\phi$ is vector of $\phi_i$, $\tR$ is vector of $\tR_i$ and $\iota$ is vector of 1's.
    \im Utility is
    $$-\E^{-\alpha \tw_1} = \E^{-\alpha w_0R_f} \times \left(- \E^{-\alpha \phi'(\tR - R_f\iota)}\right)$$
    \im Optimal dollar investments are independent of initial wealth (absence of wealth effects).
    \im Optimal shares are also independent of initial wealth.
    \ei 
\end{frame}

\begin{frame}{LRT Utility}
    \bi 
    \im Optimal dollar investments are affine in initial wealth: $\phi_i^* = a_i + b_i w_0$.
    \im Optimal shares are also affine in initial wealth (divide $a_i$ and $b_i$ by $p_i$).
    \im We say ``wealth expansion paths are linear.''
    \im Slope coefficient depends on cautiousness parameter. (Recall LRT means $\tau = A + B w$ and $B$ is called the cautiousness parameter.)
    \im CARA investors have horizontal (zero slope) expansion paths.
    \im CRRA investors with same relative risk aversion have parallel expansion paths (same $b_i$).
    \ei
\end{frame}

\section{Euler Equation}
\subsection{}

\bfr\frametitle{Time-Additive Utility and the Euler Equation}
\bi
\im Date--0 and date--1 consumption.  Utility function $v(c_0,c_1)$.  Assume time-additive utility
$$v(c_0,c_1) = u(c_0) + \delta u(c_1)$$
\im Consumption/investment problem:  choose $c_0, \phi_1, \ldots, \phi_n$ to 
$$\max \;u(c_0) +  \mye\lb \delta u\lp \sum_{i=1}^n \phi_i\tr_i \rp \rb \quad \text{subject to} \quad c_0 + \sum_{i=1}^n \phi_i = w\,.$$
\im FOC: $u'(c_0) = \lambda$ and 
$$(\forall \,i) \quad  \mye\lb \delta u'\lp \sum_{i=1}^n \phi_i\tr_i \rp \tr_i\rb = \lambda$$
\im So
$$(\forall \,i) \quad \mye\lb \frac{\delta u'\lp \sum_{i=1}^n \phi_i\tr_i \rp}{u'(c_0)} \tr_i\rb = 1$$
\ei
\end{frame}



\bfr\frametitle{Exercise 2.6}
\textit{With time-additive CRRA utility, the elasticity of intertemporal substitution is the reciprocal of relative risk aversion.}

Definition of EIS: for a utility function $v(c_0,c_1)$.  
$$\text{MRS} = \frac{\partial v/\partial c_0}{\partial v/\partial c_1}$$
$$\text{EIS} = \frac{\D \log (c_1/c_0)}{\D \log MRS}$$

Set $x=c_1/c_0$.  Assume time-additive CRRA utility:
$$v(c_0,c_1) = \frac{1}{1-\rho}c_0^{1-\rho} + \frac{\delta}{1-\rho}c_1^{1-\rho}$$
Then MRS $\,= -\log \delta + \rho \log x$.  So, $\D \log \text{MRS} /\D \log x = \rho$.
 This implies $$\text{EIS}  = \frac{1}{\rho}$$
\end{frame}

\end{document}


\section{LRT}
\subsection{}

\bfr\frametitle{Linear Risk Tolerance}
\bi
\im Assume no labor income.  Assume LRT $\tau(w) = A + B w$.  Assume a risk-free asset exists.
\im Then optimal investment in each risky asset (in consumption good units) is
$$\phi^*_i = \xi_i [A +  B R_f w_0]$$
where $\xi_i$ does not depend on $A$ or on $w_0$.  
\im Investors with LRT and the same cautiousness parameter have parallel wealth expansion paths.
\im CARA: $\phi^*_i = \xi_i A$ (no wealth effects)
\im CRRA: $\pi^*_i = \xi_i B R_f$ (proportion of wealth invested is independent of wealth)
\ei
\end{frame}

\bfr\frametitle{Example: Log Utility}
\bi
\im $\tw = w_0[R_f + \pi'(\tR - R_f\iota)]$
\im $\log \tw = \log w_0 + \log [R_f + \pi'(\tR-R_f\iota)]$
\im Optimal $\pi$ maximizes expected value of $\log [R_f + \pi'(\tR-R_f\iota)]$ and is independent of $w_0$.
\ei
\end{frame}

\bfr\frametitle{Two-Fund Separation with LRT Utility}
\bi
\im Suppose all investors have LRT utility with the same cautiousness parameter.  
\im Investor $h$ invests
$$\phi^*_{hi} = \xi_i [A_h +  B R_f w_{h0}]$$
in risky asset~$i$.
The total investment of investor $h$ in risky assets is
$$\sum_{i=1}^n \phi^*_{hi} = \lp \sum_{i=1}^n \xi_i \rp [A_h + B R_f w_{h0}]$$
\im Of the total investment in risky assets, the fraction in asset~$i$ is
$$\frac{\xi_i}{\sum_{j=1}^n \xi_j}$$
\im This is the same for all investors.  Hence it is the market portfolio of risky assets.  All investors hold a combination of the risk-free asset and the market portfolio.
\ei
\end{frame}

\begin{frame}[plain]
What is the sign of $\mye[u''(\tw)(\tr-R_f)]$?
\bi
\im By the definition of covariance,
$$\mye[u''(\tw)(\tr-R_f)] = \cov(u''(\tw),\tr-R_f) + \mye[u''(\tw)]\mye[\tr-R_f]$$
and the second term is negative.
\im DARA implies the first term is positive and larger in absolute value than the second term.
\ei
\end{frame}

\bfr\frametitle{Proof}
Define $w_f = w_0R_f$ and substitute
\begin{align*}
u''(\widetilde{w}) &= -\alpha (\widetilde{w})u'(\widetilde{w})\\
&= -\alpha (w_f)u'(\widetilde{w}) + \left[\alpha (w_f)- \alpha (\widetilde{w})\right]u'(\widetilde{w})
\end{align*}
in $\mye[u''(\tw)(\tr-R_f)]$ to obtain
$$-\alpha (w_f)\mye\left[(\widetilde{R}-R_f)u'(\widetilde{w})\right] + \mye\left[\left[\alpha (w_f)- \alpha (\widetilde{w})\right](\widetilde{R}-R_f)u'(\widetilde{w})\right]\,.$$
The first term in this expression is zero, due to the FOC.  The second term is positive because
$$\left[\alpha (w_f)- \alpha (\widetilde{w})\right](\widetilde{R}-R_f)$$
is everywhere positive, due to~$\widetilde{w}$ being greater than $w_f$ whenever $\widetilde{R}>R_f$ and the assumption that absolute risk aversion is a decreasing function of wealth.  
\end{frame}


\bfr\frametitle{Exercises}

\end{frame}

\end{document}


\begin{frame}\frametitle{}
\bi
\im
\ei
\end{frame}

\begin{frame}\frametitle{}
\bi
\im
\ei
\end{frame}

\begin{frame}\frametitle{}
\bi
\im
\ei
\end{frame}

\begin{frame}\frametitle{}
\bi
\im
\ei
\end{frame}

\end{document}


\bfr\frametitle{Linear Risk Tolerance}
\bi
\im Linear risk tolerance (LRT): $\tau(w) = A + Bw$.  $B$ is cautiousness parameter.
\im Also called HARA (hyperbolic absolute risk aversion) because absolute risk aversion is $\alpha(w) = 1/(A+Bw)$.
\ei
Examples:
\bi
\im $B=0 \Rightarrow\,$ absolute risk aversion $\,=1/A$.  Called CARA (constant absolute risk aversion).  
\bi
\im Up to a monotone affine transform, all CARA utility functions are $u(w) = - \E^{-\alpha w}$.
\im Here, $\alpha=\,$ absolute risk averison
\ei
\im $A=0
\im CARA $\,\Rightarrow \tau(w) = A = 1/\alpha$.
\im CRRA (constant relative risk aversion) $\,\Rightarrow \tau(w) = Bw = w/\rho$.
\ei
\im Aggregate risk tolerance for individuals $h=1,\ldots,H$: $\sum_{h=1}^H \tau_h(w_h)$
\im Aggregate absolute risk aversion: $1/\sum_{h=1}^H \tau_h(w_h) = 1/ \sum_{h=1}^h (1/\alpha_h(w_h))$
\im $\alpha$ constant nonzero $\,\Rightarrow$ negative exponential utility $u(w) = -\E^{-\alpha w}$.
\im CARA utility $\,\Rightarrow\,$ no wealth effects
\ei
\im CARA
\im CRRA
\bi
\im $u(w) = \log w \Rightarrow \,$ relative risk aversion $\,=1$.
\im $u(w) = \frac{1}{1-\rho}w^{1-\rho} \Rightarrow \,$ relative risk aversion $\,=\rho \neq 1$.
\ei
\im Shifted CRRA
\bi
\im For a constant $\zeta$ and $w>\zeta$, $u(w) = \log (w-\zeta)$.
\im For a constant $\zeta $ and a constant $\rho $ with $\rho  \neq 0$ and $\rho  \neq 1$ and for~$w$ such that $(w-\zeta)/\rho >0$, and including $w=\zeta$ if $\rho<1$,
 $$u(w) = \frac{\rho }{1-\rho }\left(\frac{w-\zeta}{\rho}\right)^{1-\rho }\,.$$
\im When $\rho>0$ (usual case), we can write shifted power as
$$u(w) = \frac{1}{1-\rho }\left(w-\zeta\right)^{1-\rho }\,.$$
\ei 

\ei

\end{frame}

\bfr\frametitle{Quadratic Utility}
Shifted power with $\rho=-1$:
$$u(w) = -\frac{1 }{2}\left(\zeta-w\right)^2$$ 
It is monotone increasing for $w<\zeta$ ($\zeta$ is bliss point).

Mean-Variance Preferences:
$$\mye[u(\tilde{w})] \sim  \zeta \overline{w} - \frac{1}{2}\overline{w}^2 - \frac{1}{2}\var(\widetilde{w})$$

Risk tolerance is $\zeta-w$ which decreases as $w$ increases up to $\zeta$.
\end{frame}

\bfr\frametitle{Higher Moments}
\bi
\im DARA (decreasing absolute risk aversion) $\,\Rightarrow u'''(w)>0$
\im DARA and decreasing absolute prudence $\,\Rightarrow u''''(w)<0$
\im Taylor series expansion:
\begin{multline*}
\mye[u(\tilde{w})] \approx u(\overline{w}) + \overbrace{u'(\ow)\mye[\tw - \ow]}^{\textcolor{red}{0}} + \frac{1}{2}u''(\ow)\mye\lb (\tw-\ow)^2\rb \\+ \frac{1}{6}u'''(\ow)\mye\lb(\tw-\ow)^3\rb + \frac{1}{24} u''''(\ow)\mye\lb(\tw-\ow)^4\rb\,.
\end{multline*}
\im Approximate mean-variance-skewness-kurtosis preferences, liking mean, disliking variance, preferring positive skewness, and disliking kurtosis.
\ei
\end{frame}


\end{document}
\end{document}