%%%%%%%%%%%%%%%%%%%%%%%%%%%%%%%%%%%%%%%%%%%%%%%%%%%%%%%%%%%%%%%%%%%%%%%%%%%%
\documentclass[xcolor=dvipsnames,10pt]{beamer}

\mode<presentation> {\usetheme{Singapore}}
\usepackage{pgfpages}

%\setbeamercovered{transparent} 
\usepackage[english]{babel}
\usepackage[latin1]{inputenc}
\usepackage{times,amsfonts}
\usepackage[T1]{fontenc}
\setlength{\parskip}{\baselineskip}
% Or whatever. Note that the encoding and the font should match. If T1
% does not look nice, try deleting the line with the fontenc.

%\usecolortheme{sidebartab}
\setbeamertemplate{itemize item}[triangle]



\usepackage{calc}
\usepackage{environ}
\newcommand{\halfmargin}{0.0001\paperwidth}


\RequirePackage{booktabs,colortbl,ulem}

\usepackage{animate}
\RequirePackage{booktabs,colortbl,gensymb}
\setlength{\parskip}{\baselineskip}

\usepackage{calc}
\usepackage{environ}

% \newcommand{\halfmargin}{0.0001\paperwidth}


\NewEnviron{wideframe}[1][]{%
\begin{frame}{#1}
\makebox[\textwidth][c]{
\begin{minipage}{\dimexpr\paperwidth-\halfmargin-\halfmargin\relax}
\BODY
\end{minipage}}
\end{frame}
}


\DeclareMathOperator{\stdev}{stdev}
\DeclareMathOperator{\var}{var}
\DeclareMathOperator{\cov}{cov}
\DeclareMathOperator{\corr}{corr}
\DeclareMathOperator{\prob}{prob}
\DeclareMathOperator{\n}{n}
\DeclareMathOperator{\N}{N}
\DeclareMathOperator{\Cov}{Cov}

\newcommand{\hlf}{\frac{1}{2}}
\newcommand{\bi}{\begin{itemize}}
\newcommand{\ei}{\end{itemize}}
\newcommand{\im}{\item}
\newcommand{\D}{\mathrm{d}}
\newcommand{\E}{\mathrm{e}}
\newcommand{\mye}{\ensuremath{\mathsf{E}}}
\newcommand{\myreal}{\ensuremath{\mathbb{R}}}
\newcommand{\bq}{\begin{equation}}
\newcommand{\eq}{\end{equation}}
\newcommand{\eqdef}{\;\buildrel \text{d{}ef}\over = \;}
\newcommand{\xstar}{\buildrel *\over X}
\newcommand{\pmax}{p^{\text{max}}}
\newcommand{\qmax}{q^{\text{max}}}
\newcommand{\bfr}{\begin{frame}}
\newcommand{\bfrp}{\begin{frame}[plain]}
\newcommand{\efr}{\end{frame}}
\newcommand{\F}{\mathcal{F}}
\newcommand{\FF}{\mathbb{F}}
\newcommand{\ve}{\varepsilon}
\newcommand{\lh}{\hat{\lambda}}
\definecolor{mycolor}{gray}{0.8}
\definecolor{mymaincolor}{rgb}{0.6862745098039216,0.9333333333333333,0.9333333333333333}
\newcommand{\alr}[1]{\textcolor{blue}{#1}}
\definecolor{LightCyan}{rgb}{0.88,1,1}
\newcommand{\yel}{\cellcolor{yellow}}
\newcommand{\blue}{\cellcolor{SkyBlue}}
\newcommand{\gr}{\cellcolor{SpringGreen}}
\newcommand{\pink}{\cellcolor{pink}}
\newcommand{\apr}{\cellcolor{Apricot}}
\newcommand{\tve}{\tilde{\varepsilon}}
\newcommand{\tw}{\tilde{w}}
\newcommand{\ttth}{\tilde{\theta}}
\newcommand{\te}{\tilde{e}}
\newcommand{\ts}{\tilde{s}}
\newcommand{\tx}{\tilde{x}}
\newcommand{\ty}{\tilde{y}}
\newcommand{\tv}{\tilde{v}}
\newcommand{\tp}{\tilde{p}}
\newcommand{\tF}{\tilde{F}}
\newcommand{\tf}{\tilde{f}}
\newcommand{\tZ}{\tilde{Z}}
\newcommand{\ow}{\overline{w}}
\newcommand{\lb}{\left[}
\newcommand{\rb}{\right]}
\newcommand{\lp}{\left(}
\newcommand{\rp}{\right)}
\newcommand{\tm}{\tilde{m}}
\newcommand{\tc}{\tilde{c}}
\newcommand{\tz}{\tilde{z}}
\newcommand{\str}[1]{\textcolor{blue}{\sout{#1}}}
\newcommand{\tr}{\widetilde{R}}
\newcommand{\tR}{\widetilde{\mathbf{R}}}
\newcommand{\bms}{\begin{multline*}}
\newcommand{\ems}{\end{multline*}}
\newcommand{\bas}{\begin{align*}}
\newcommand{\eas}{\end{align*}}
\newcommand{\qr}{\mathbb{Q}}
\newcommand{\IMAGES}{/home/kerry/Dropbox/Images}
\newcommand{\tX}{\tilde{X}}
\newcommand{\tY}{\tilde{Y}}

\author{\vskip 0.5in \small Kerry Back \\BUSI 521--ECON 505\\ Rice University \\Spring 2022}
%\institute{Rice University\\ Spring 2019}
\date[]






\begin{document}
\title{\vskip 0.5in Day 14}
\subtitle{Rational Expectations Equilibria}

\begin{frame}
  \titlepage
\end{frame}

%%%%%%%%%%%%%%%%%%%%%%%%%%%%%%%%%%%%%%%%%%%


\section{Equilibrium}\subsection{}

\begin{frame}{Informational Efficiency}

An important issue, dating back to Hayek, is the extent to which market prices reflect and convey the information of market participants.

In finance, following Fama, prices are said to be 
\vspace*{-\baselineskip}
\bi
\im weak-form efficient if prices contain all of the information that is in past prices,
\im semi-strong-form efficient if prices contain all public information, and
\im strong-form efficient if prices contain all public and private information
\ei
We will look at price formation under asymmetric information to study this.
Two types of models: competitive (price-taking) and strategic (game-theoretic).

\end{frame}

\bfr\frametitle{CARA-Normal}

 Suppose there is a risk-free asset and a single risky asset.  
Suppose the aggregate supply of the risky asset is 1 share.

 Suppose the end-of-period value $\tx$ of the risky asset is normally distributed with mean $\mu$ and variance $\sigma^2$. 

 Suppose there are $H$ investors and each investor $h$ has CARA utility with absolute risk aversion $\alpha_h$ and risk tolerance $\tau_h=1/\alpha_h$.
 
\end{frame}

\begin{frame}{Review of Equilibrium}
Let $p$ denote the price of the risky asset, and set $\tr = \tx/p$.  

 The demand for the risky asset by investor $h$ is (from Chapter 2)
 $$\phi_h \ = \ \frac{\mye[\tr] - R_f}{\alpha_h \var(\tr)}$$
  The mean of $\tr$ is $\mu/p$, and the variance of $\tr$ is $\sigma^2/p^2$, and $\phi = \theta p$ (dollars = shares times price), so
 $$\theta_h p  \ = \ \frac{\mu/p - R_f}{\alpha_h \sigma^2/p^2}\quad \Rightarrow \quad \theta_h \ = \  \tau_h\frac{\mu - R_fp}{\sigma^2}$$
 Equating aggregate demand to supply yields
$$ \tau\frac{\mu - R_fp}{\sigma^2} \ = \ 1 \quad \Rightarrow \quad p \ = \ \frac{\mu - \alpha \sigma^2}{R_f}$$
where $\tau = \sum_h \tau_h$ and $\alpha = 1/\tau$.

\end{frame}



\bfr\frametitle{Asymmetric Information}

Assume some investors observe a signal $\ts = \tx + \tve$ before trade.  Let $I \subset \{1,\ldots, H\}$ denote the informed investors.  Let $U$ denote the other investors, who do not observe the signal.

Let $\mu_I$ denote the expectation of $\tx$ for the informed traders (which depends on $\ts$) and let $\mu_U$ denote the expectation for the uninformed traders (which maybe depends on $\ts$).

Write $\nu = 1/\sigma^2$ (called a precision).  Let $\nu_{I}$ denote the precision (reciprocal of variance) for the informed traders and let $\nu_{U}$ denote the precision for the uninformed traders.

Assume both types of investors still regard $\tx$ as normally distributed. We can calculate $\mu_I$ and $\nu_{I}$ using standard statistical arguments.  But $\mu_U$ and $\nu_{U}$ depend on how much uninformed investors learn from prices.

Assume $R_f$ is given exogenously, so information comes only from $p$.
\end{frame}

\begin{frame}{Equilibrium}
    Informed investors' demands are
    $\theta_h \ = \  \tau_h\nu_{I}(\mu_I - R_fp)$.
    
    Uninformed investors' demands are $\theta_h \ = \  \tau_h\nu_{U}(\mu_U - R_fp)$.
    
    Market clearing is
    $$\sum_{h \in I} \tau_h\nu_{I}(\mu_I - R_fp) + \sum_{h \in U} \tau_h\nu_{U}(\mu_U - R_fp) = 1$$
    Set $\tau_I = \sum_{h \in I} \tau_h$ and $\tau_U = \sum_{h \in U} \tau_h$.  Market clearing is
    $$\tau_I\nu_{I}(\mu_I - R_fp) + \tau_U\nu_{U}(\mu_U - R_fp) = 1$$
    Equilibrium is
    $$p = \frac{\tau_I\nu_I\mu_I + \tau_U\nu_U\mu_U - 1}{(\tau_I\nu_I+\tau_U\nu_U)R_f} = \frac{1}{R_f}\left(\frac{\tau_I\nu_I\mu_I + \tau_U\nu_U\mu_U}{\tau_I\nu_I+\tau_U\nu_U} - \frac{1}{\tau_I\nu_I+\tau_U\nu_U}\right)$$
\end{frame}

\section{Informed Investors}\subsection{}
\begin{frame}{Posterior Mean}

Suppose the signal is truth plus noise: $\ts = \tx + \te$, where $\te$ is normally distributed, zero mean, and uncorrelated with $\tx$.

To calculate $\mu_I$ and $\nu_I$, start with a projection:
$$\tx -\mu\ = \beta (\ts -\mu_s)+ \tilde \varepsilon$$
where $\tilde \varepsilon$ is uncorrelated with $\ts$ and mean zero.

We have $\mu_s = \mu$.  This implies
$$ \mu_I := \mye[\tx \mid \ts] \ = \ (1-\beta)\mu + \beta \ts\,.$$
In words: \alert{the posterior mean is a weighted average of the prior mean and the signal.}

Also, $\beta = \cov(\tx,\ts)/\sigma_s^2 = \sigma^2/\sigma_s^2 = \sigma^2/(\sigma^2+\sigma_e^2$).
\end{frame}

\begin{frame}{Posterior Variance}

From the projection equation,
$$\sigma^2 = \beta^2 \sigma_s^2 + \sigma_\varepsilon^2$$
The conditional variance of $\tx$ for the informed investors is the variance of $\tilde\varepsilon$, which is
$$\sigma_\varepsilon^2 = \sigma^2 - \beta^2\sigma_s^2 = \sigma^2(1-\beta)$$

Notice that the conditional mean depends on $\ts$, but the conditional variance does not.
\end{frame}

\begin{frame}{Posterior Precision}
The variance of $\tve$ is
\begin{align*}
    \sigma^2(1-\beta) & = \sigma^2\left(1-\frac{\sigma^2}{\sigma^2+\sigma_e^2}\right)\\
    &= \frac{\sigma^2\sigma_e^2}{\sigma^2+\sigma_e^2}
\end{align*}
 So, the informed precision is
$$
\nu_I = \frac{1}{\sigma^2} + \frac{1}{\sigma_e^2}
$$
In words: \alert{the posterior precision is the sum of the prior precision and the precision of the signal noise term.}
\end{frame}

\section{Models}\subsection{}

\begin{frame}{Model I: Naive Uninformed Investors}
    Suppose uninformed investors learn nothing from the price, so $\mu_U=\mu$ and $\nu_U=\nu$ (prior mean and precision).  Then, the equilibrium is
    
    $$p \ = \ \frac{1}{R_f}\left(\frac{\tau_I\nu_I[(1-\beta)\mu + \beta\ts] + \tau_U\nu\mu}{\tau_I\nu_I+\tau_U\nu} - \frac{1}{\tau_I\nu_I+\tau_U\nu}\right)$$

 The naive investors are irrational, because they should realize that a higher price means a higher $\ts$.  They should extract information from the price.
 \end{frame}
 
 \begin{frame}{Model II: Fully Revealing \\Rational Expectations Equilibrium}
    Suppose uninformed investors are fully rational.  Conjecture that they learn $\ts$ fully from the price, so they have the same posterior mean and precision as do the informed investors.  
    
    Substituting a common mean and precision into the demand functions and solving for equilibrium, we obtain
        $$p \ = \ \frac{1}{R_f}\left((1-\beta)\mu + \beta\ts - \frac{1}{\tau\nu_I}\right)$$
The uninformed can indeed learn $\ts$ from this price, so our conjecture is established.

 Grossman-Stiglitz paradox:  who would pay to collect information if they can get it free from the price?
 \end{frame}
 
 \begin{frame}{Model III: Grossman-Stiglitz}
     Suppose the supply of the asset is a random variable $\tz$.  Randomness could be due to random demand from other traders, sometimes called liquidity traders, because they may be trading due to cash needs or cash surpluses (liquidity shocks).
     
    Suppose $\tz$ is normally distributed and independent of all other random variables in the model.  
    
    We solve by `guess and verify.'
 We conjecture that the equilibrium price is $p = a_0 + a_1 \ts + a_2 \tz$ with $a_1 \neq 0$.  We need to compute equilibrium coefficients $a_0$, $a_1$, and $a_2$.

\end{frame}

\section{Solving the GS Model}\subsection{}

\begin{frame}{Uninformed Investors}
 From the price, uninformed investors can calculate
\begin{align*}
\frac{p  - a_0 - a_2\mu_z}{a_1} \ &= \ \ts + \frac{a_2}{a_1} (\tz - \mu_z) \\
& = \ \ts + b (\tz-\mu_z) \\
&  = \ \tx + \te + b (\tz-\mu_z)\,.
\end{align*}
where we define $b = a_2/a_1$.
 
So, the price provides a truth-plus-noise signal for uninformed investors.  

The noise for uninformed investors is $\te + b(\tz-\mu_z)$, which is `noisier' than the informed investors' signal. 
\end{frame}



\bfr\frametitle{Uninformed Demands}
 For uninformed investors, the conditional (posterior) mean of $\tx$ is
$$(1-\beta_U)\mu + \beta_U [\ts + b (\tz-\mu_z)]$$
where
$$\beta_U \ = \ \frac{\sigma^2}{\sigma^2 + \sigma^2_{e} + b^2 \sigma^2_z}\,.$$

 The posterior precision is
$$\nu_U \ \eqdef \ \nu + \frac{1}{\sigma^2_{e} + b^2 \sigma^2_z}$$

 Aggregate uninformed demands are
$$\tau_U\nu_U\bigg[
(1-\beta_U)\mu + \beta_U [\ts + b (\tz-\mu_z)] - R_f p \bigg]\,.$$
\end{frame}

\begin{frame}{Market Clearing}
 Aggregate demand is
\begin{multline*}
\tau_I\nu_I\bigg[(1-\beta)\mu + \beta \ts - R_fp \bigg] \\
+ \tau_U\nu_U\bigg[
(1-\beta_U)\mu + \beta_U [\ts + b (\tz-\mu_z)] - R_fp \bigg]
\end{multline*}

Substituting $p  = a_0 + a_1 \ts + a_2 \tz$, we see that aggregate 
demand is of the form $A + B \ts + C\tz$, so market-clearing is
$$A + B\ts + C\tz  = \tz$$
where $A$, $B$, and $C$ depend on the unknown coefficients $a_0$, $a_1$, and $a_2$ and on $R_f$.
For this to hold in all states of the world, it must be that $A=0$, $B=0$, and $C=1$.

\end{frame}

\bfr\frametitle{Solution}
 Solve $A=0$, $B=0$ and $C=1$ for $a_0$, $a_1$, and $a_2$.

 In equilibrium,
 $$b = - \frac{1}{\beta\tau_I\nu_I} = - \frac{(1-\beta)\sigma^2}{\beta\tau_I}$$
Recall that uninformed investors can calculate 
$$ \ts + b(\tz - \mu_z)$$
Their information is better (i.e., the variance of $b\tz$ is smaller) when \vspace*{-0.5\baselineskip}
\begin{itemize}
    \item the variance of $\tz$ is smaller
    \item informed investors are more risk tolerant
    \item $\beta$ is larger (the informed investors have a better signal, so they face less residual risk), or
    \item $\sigma^2$ is smaller (again, the informed investors face less residual risk).
  \end{itemize}
\end{frame}

\end{document}

\section{Kyle Model}\subsection{}

\bfr\frametitle{Kyle Model}
\bi
\im Grossman-Stiglitz model assumes informed investors are price takers.  So, they must be small individually, but there must be enough of them for their information/demands to affect prices in the aggregate.
\im What if there is only one informed investor?  If his information gets into prices -- and he can see it in prices -- but he is a price taker (assumes his demands don't affect prices) then he must be schizophrenic (Hellwig).
\im What if he is not a price taker?
\im In reality, all large investors know that their trades move prices.
\im This is called strategic trading.
\im The Kyle model is a model of a single strategic informed trader.
\ei
\end{frame}

\bfr\frametitle{Assumptions and Notation}
\bi
\im Asset value is $\tv$.  Informed trader knows $\tv$.  Risk-free rate normalized to zero.
\im Other investors regard $\tv$ as normally distributed with mean $\mu$ and variance $\sigma^2_v$.
\im Noise traders $\tz$ with mean zero and variance $\sigma^2_z$.
\im Informed trader is risk neutral.  Denote the number of shares he demands by $\tx$.  Set $\ty = \tx + \tz$.
\im Risk-neutral market makers observe $\ty$.  Compete to fill (take the other side of) $\ty$.  
\im Equilibrium price is $p  = \mye[\tv \mid \ty]$.
\im Informed trader knows price depends on $\ty$ and knows that $\tx$ is part of $\ty$.  Chooses his trade strategically.
\ei
\end{frame}

\bfr\frametitle{Linear Equilibrium}
\bi
\im Conjecture $p  = \delta + \lambda \ty$.
\im Informed trader maximizes
\begin{align*}
\mye[\tx(\tv - p ) \mid \tv] \ &= \ \mye[\tx(\tv - \delta - \lambda \ty) \mid \tv]\\
& \ = (\tv - \delta) \mye[\tx\mid \tv] - \lambda \mye[\tx^2\mid \tv] - \lambda \mye[\tx\tz\mid \tv]\\
& \ = (\tv - \delta) \mye[\tx\mid \tv] - \lambda \mye[\tx^2\mid \tv]
\end{align*}
\im \pause FOC is (for each realization of $\tv$)
$$\tv - \delta - 2 \lambda \tx \ = \ 0$$
so
$$  \tx = \frac{\tv - \delta}{2\lambda}\,.$$
\im \pause Hence, 
$$\ty = \frac{\tv - \delta}{2\lambda} + \tz$$
which implies $\tv + 2\lambda\tz = 2\lambda\ty + \delta$.
\ei
\end{frame}

\bfr\frametitle{}
\bi
\im Market makers observe $\ty$ from which they can compute $2\lambda\ty + \delta = \tv + 2\lambda\tz$.  Thus, they have a truth-plus noise signal of $\tv$.
\im It follows that their conditional expectation of $\tv$ is
$$(1-\beta)\mu + \beta (\tv + 2\lambda\tz) \ = \ (1-\beta)\mu + \beta (2\lambda\ty + \delta)$$
where
$$\beta = \frac{\sigma_v^2}{\sigma_v^2 + 4\lambda^2\sigma_z^2}\,.$$
\im  Equilibrium requires that this conditional expectation equal $p $.
\im \pause To confirm our conjecture that $p  = \delta + \lambda \ty$, we must have
$$\delta + \lambda \ty = (1-\beta)\mu + \beta (2\lambda\ty + \delta)$$
\im \pause This is true for all realizations of $\ty$ if and only if
$$
\delta = (1-\beta)\mu + \beta \delta \quad \Leftrightarrow \quad \delta = 
\mu$$
and 
$$\lambda = 2 \beta \lambda \quad \Leftrightarrow \quad \beta = \frac{1}{2}$$
The last condition is equivalent to $\lambda = \sigma_v / (2 \sigma_z)$.

\ei
\end{frame}

\bfr\frametitle{}
\bi
\im The posterior precision of market makers is
$$\frac{1}{\sigma_v^2} + \frac{1}{4\lambda^2\sigma_z^2} \ = \ \frac{2}{\sigma_v^2}\,.$$
\im So, the market learns half of the informed trader's information from her trades.
\im The $\lambda$ is called Kyle's lambda and measures the liquidity of the market.  If $\lambda$ is small, the market is liquid -- large trades can be made with little effect on the price.
\im The market is liquid if private information $\sigma_v^2$ is small or there is a large amount $\sigma_z^2$ of noise trading.
\ei
\end{frame}


\bfr\frametitle{Exercises}

Graded: 22.1, 24.2, 24.4

Recommended: 24.1, 24.3
\end{frame}
\end{document}
