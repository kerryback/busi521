\documentclass[xcolor=dvipsnames,10pt]{beamer}

\mode<presentation> {\usetheme{Singapore}}
\usepackage{pgfpages}

%\setbeamercovered{transparent} 
\usepackage[english]{babel}
\usepackage[latin1]{inputenc}
\usepackage{times,amsfonts}
\usepackage[T1]{fontenc}
\setlength{\parskip}{\baselineskip}
% Or whatever. Note that the encoding and the font should match. If T1
% does not look nice, try deleting the line with the fontenc.

%\usecolortheme{sidebartab}
\setbeamertemplate{itemize item}[triangle]



\usepackage{calc}
\usepackage{environ}
\newcommand{\halfmargin}{0.0001\paperwidth}


\RequirePackage{booktabs,colortbl,ulem}

\usepackage{animate}
\RequirePackage{booktabs,colortbl,gensymb}
\setlength{\parskip}{\baselineskip}

\usepackage{calc}
\usepackage{environ}

% \newcommand{\halfmargin}{0.0001\paperwidth}


\NewEnviron{wideframe}[1][]{%
\begin{frame}{#1}
\makebox[\textwidth][c]{
\begin{minipage}{\dimexpr\paperwidth-\halfmargin-\halfmargin\relax}
\BODY
\end{minipage}}
\end{frame}
}


\DeclareMathOperator{\stdev}{stdev}
\DeclareMathOperator{\var}{var}
\DeclareMathOperator{\cov}{cov}
\DeclareMathOperator{\corr}{corr}
\DeclareMathOperator{\prob}{prob}
\DeclareMathOperator{\n}{n}
\DeclareMathOperator{\N}{N}
\DeclareMathOperator{\Cov}{Cov}

\newcommand{\hlf}{\frac{1}{2}}
\newcommand{\bi}{\begin{itemize}}
\newcommand{\ei}{\end{itemize}}
\newcommand{\im}{\item}
\newcommand{\D}{\mathrm{d}}
\newcommand{\E}{\mathrm{e}}
\newcommand{\mye}{\ensuremath{\mathsf{E}}}
\newcommand{\myreal}{\ensuremath{\mathbb{R}}}
\newcommand{\bq}{\begin{equation}}
\newcommand{\eq}{\end{equation}}
\newcommand{\eqdef}{\;\buildrel \text{d{}ef}\over = \;}
\newcommand{\xstar}{\buildrel *\over X}
\newcommand{\pmax}{p^{\text{max}}}
\newcommand{\qmax}{q^{\text{max}}}
\newcommand{\bfr}{\begin{frame}}
\newcommand{\bfrp}{\begin{frame}[plain]}
\newcommand{\efr}{\end{frame}}
\newcommand{\F}{\mathcal{F}}
\newcommand{\FF}{\mathbb{F}}
\newcommand{\ve}{\varepsilon}
\newcommand{\lh}{\hat{\lambda}}
\definecolor{mycolor}{gray}{0.8}
\definecolor{mymaincolor}{rgb}{0.6862745098039216,0.9333333333333333,0.9333333333333333}
\newcommand{\alr}[1]{\textcolor{blue}{#1}}
\definecolor{LightCyan}{rgb}{0.88,1,1}
\newcommand{\yel}{\cellcolor{yellow}}
\newcommand{\blue}{\cellcolor{SkyBlue}}
\newcommand{\gr}{\cellcolor{SpringGreen}}
\newcommand{\pink}{\cellcolor{pink}}
\newcommand{\apr}{\cellcolor{Apricot}}
\newcommand{\tve}{\tilde{\varepsilon}}
\newcommand{\tw}{\tilde{w}}
\newcommand{\ttth}{\tilde{\theta}}
\newcommand{\te}{\tilde{e}}
\newcommand{\ts}{\tilde{s}}
\newcommand{\tx}{\tilde{x}}
\newcommand{\ty}{\tilde{y}}
\newcommand{\tv}{\tilde{v}}
\newcommand{\tp}{\tilde{p}}
\newcommand{\tF}{\tilde{F}}
\newcommand{\tf}{\tilde{f}}
\newcommand{\tZ}{\tilde{Z}}
\newcommand{\ow}{\overline{w}}
\newcommand{\lb}{\left[}
\newcommand{\rb}{\right]}
\newcommand{\lp}{\left(}
\newcommand{\rp}{\right)}
\newcommand{\tm}{\tilde{m}}
\newcommand{\tc}{\tilde{c}}
\newcommand{\tz}{\tilde{z}}
\newcommand{\str}[1]{\textcolor{blue}{\sout{#1}}}
\newcommand{\tr}{\widetilde{R}}
\newcommand{\tR}{\widetilde{\mathbf{R}}}
\newcommand{\bms}{\begin{multline*}}
\newcommand{\ems}{\end{multline*}}
\newcommand{\bas}{\begin{align*}}
\newcommand{\eas}{\end{align*}}
\newcommand{\qr}{\mathbb{Q}}
\newcommand{\IMAGES}{/home/kerry/Dropbox/Images}
\newcommand{\tX}{\tilde{X}}
\newcommand{\tY}{\tilde{Y}}

\author{\vskip 0.5in \small Kerry Back \\BUSI 521--ECON 505\\ Rice University \\Spring 2022}
%\institute{Rice University\\ Spring 2019}
\date[]






\title{\vskip 0.5in Day 11}
\subtitle{Representative Investors}

\begin{document}

\begin{frame}[plain]
  \titlepage
\end{frame}

\section{Intro}\subsection{}

\begin{frame}{Equilibrium with Date--0 Consumption}
Assume there is no labor income.
Investors $h=1,\ldots,H$ have endowments of date--0 consumption $\bar{c}_{h0}$ and asset shares $\bar{\theta}_h$.  Assets $i=1,\ldots,n$ have payoffs $\tx_i$.

Take date--0 consumption to be the numeraire (price=1).  An equilibrium is a price vector $p\in \myreal^n$ for assets, a date--0 consumption allocation $(c_{10},\ldots,c_{H0})$ and asset allocations $(\theta_1,\ldots,\theta_H)$ such that
\begin{itemize}
    \item date--0 consumption $c_{h0}$ and portfolio $\theta_h$ are optimal for investor $h$, for all $h$
    \item the date--0 consumption market: $\sum_h c_{h0} = \sum_h \bar{c}_{h0}$
    \item the asset markets clear: $\sum_h \theta_h = \sum_h\bar{\theta}_h$
\end{itemize}
\end{frame}

\begin{frame}{Representative Investor}
    There is a representative investor 
    if each asset price vector $p$ that is part of a securities market equilibrium is also part of a securities market equilibrium in the economy in which there is only the representative investor, and the representative investor's endowments are $\bar{c}_0 := \sum_h \bar{c}_{h0}$ and $\bar{\theta} := \sum_h \bar{\theta}_{h0}$.  
    
    By the FOC in the representative investor economy, the representative investor's MRS is an SDF.
 \end{frame}
 
 \begin{frame}{Plan for Today}
     Assume there is a representative investor with CRRA utility.  Derive
     \begin{itemize}
         \item formula for market return 
         \item formula for risk-free rate
         \item formula for log equity premium (assuming also lognormal consumption growth)
         \item variation of the CAPM
     \end{itemize}
     
Then discuss:
\begin{itemize}
    \item There is a representative investor if the first welfare theorem holds (complete markets or LRT utility with same cautiousness parameter)
    \item With LRT utility for all investors and same cautiousness parameter, the representative investor has the same utility function.
\end{itemize}
 \end{frame}

\section{Equity Premium}\subsection{}
\begin{frame}{Representative Investor with CRRA Utility}
    Assume there is a representative investor with utility function
      $$(c_0,c_1) \mapsto u(c_0) + \delta u(c_1)$$
      where
      $$u(c) = \frac{1}{1-\rho}c^{1-\rho}$$
      
      Let $c_0$ denote aggregate consumption at date 0, and let $\tilde{c}_1$ denote aggregate consumption at date 1.
      
      Then
      $$\delta \left(\frac{\tc_1}{c_0}\right)^{-\rho}$$
      is an SDF.
     \end{frame}

\bfr\frametitle{Market Return}
 Assume $\tc_1$ is spanned by the assets.  Its cost is 
$$\mye[\tm\tc_1]= \mye\left[\frac{\delta \tc_1^{-\rho}}{c_0^{-\rho}}\tc_1\right] = c_0\mye\left[\frac{\delta \tc_1^{-\rho}}{c_0^{-\rho}}\cdot\frac{\tc_1}{c_0}\right] = \delta c_0\mye\left[\left(\frac{\tc_1}{c_0}\right)^{1-\rho}\right]$$
 The market return is 
 $$\tr_m := \frac{\tc_1 }{ \mye[\tm\tc_1]} = \frac{1}{\delta\mye\left[\left(\frac{\tc_1}{c_0}\right)^{1-\rho}\right]}\cdot \frac{\tc_1}{c_0} := \frac{1}{\nu_1} \cdot \frac{\tc_1}{c_0}$$
 
 \end{frame}
 
\bfr\frametitle{Risk-Free Return}
 The risk-free return is
$$R_f = \frac{1}{\mye[\tm]} = \frac{1}{\delta\mye[(\tc_1/c_0)^{-\rho}]} := \frac{1}{\nu_0}$$

\end{frame}

\begin{frame}{Log Equity Premium}
$$\frac{\tr_m}{R_f} = \frac{\nu_0}{\nu_1}\cdot \frac{\tc_1}{c_0}$$
So,
$$
\frac{\mye[\tr_m]}{R_f} = \frac{\nu_0\mye[\tc_1/c_0]}{\nu_1} = \  \frac{\mye[(\tc_1/c_0)^{-\rho}]\mye[\tc_1/c_0]}{\mye[(\tc_1/c_0)^{1-\rho}]}
= \frac{\mye[\tc_1]\mye[\tc_1^{-\rho}]}{ \mye[\tc_1^{1-\rho}]}
$$
\end{frame}


\bfr\frametitle{Lognormal Consumption}
 Assume $\log \tc_1 - \log c_0 = \mu + \sigma \tve$ for constants $\mu$ and $\sigma$ and a standard normal $\tve$.
\begin{align*}\tc_1 = c_0\E^{\mu+\sigma\tve} \;&\Rightarrow\; \mye[\tc_1] = c_0\E^{\mu+\sigma^2/2}\\
\tc_1^{-\rho} = c_0^{-\rho}\E^{-\rho\mu-\rho\sigma\tve}\;&\Rightarrow\;\mye[\tc_1^{-\rho}] = c_0^{-\rho}\E^{-\rho\mu+\rho^2\sigma^2/2}\\
\tc_1^{1-\rho} = c_0^{1-\rho}\E^{(1-\rho)\mu + (1-\rho)\sigma\tve}\;&\Rightarrow\; \mye[\tc_1^{1-\rho}] = c_0^{1-\rho}\E^{(1-\rho)\mu+(1-\rho)^2\sigma^2/2}
\end{align*}

 This implies
$$\frac{\mye[\tr_m]}{R_f} = \E^{\rho\sigma^2}$$
So,
$$\log \mye[\tr_m] - \log R_f = \rho \sigma^2$$

\end{frame}
\bfr\frametitle{Equity Premium and Risk-Free Rate Puzzles}
\bi
\im To match this model to the historical equity premium, a risk aversion around 50 is required.  Much too high.
\im Using $\rho=10$ and $\delta=0.99$, the model implies a high risk-free rate (12.7\%) and low equity premium ($\mye[\tr_m]-R_f = 1.4\text{\%}$).
\im The historical (U.S.) numbers are around 1\% for the real risk-free rate and 6\% for the equity premium.
\ei
\end{frame}

\section{CAPM Alternative}

\begin{frame}{SDF and Market Return}
The market return is 
$$\tr_m = \frac{1}{\nu_1}\cdot \frac{\tc_1}{c_0}$$
and the SDF is
$$\tm = \delta \left(\frac{\tc_1}{c_0}\right)^{-\rho}$$
so the SDF is
$$\tm = \delta \nu^{-\rho} \tr_m^{-\rho}$$
\end{frame}

\begin{frame}{CAPM Alternative}
    Risk premia of all assets are
    $$\mye[\tr] - R_f = - R_f \cov(\tr,\tm) = - \delta\nu^{-\rho} R_f \cov(\tr,\tr_m^{-\rho})$$
    
    This implies
    $$\mye[\tr] - R_f = \lambda \frac{ \cov(\tr,\tr_m^{-\rho})}{\var(\tr_m^{-\rho})}$$
    for a $\lambda$ that is the same for all assets.  So, risk premia depend on betas with respect to $\tr_m^{-\rho}$.
    
\end{frame}

\section{When is There a Representative Investor?}\subsection{}

\section{Social Planner}\subsection{}

\bfr\frametitle{Social Planner's Problem}
For each value $w$ of market wealth,  the social planner solves
$$\max \quad \sum_{h=1}^H \lambda_hu_h(w_h) \quad \text{subject to} \quad \sum_{h=1}^H w_h = w$$
Let $U(w)$ denote the maximum value.  This is the social planner's utility function.

Let $\eta$ denote the Lagrange multiplier (which depends on market wealth $w$).  Then, for all $h$,
$$\lambda_h u_h'(w_h) = \eta$$
Also, the social planner's marginal utility (the marginal value of market wealth) is equal to $\eta$.  So, for all $h$, we have the envelope result:
$$U'(w) = \lambda_h u_h'(w_h)$$
Hence, the social planner's marginal utility is proportional to an SDF.
\end{frame}

\bfr\frametitle{Social Planner's Problem with Date--0 Consumption}
Suppose investor $h$ has utility $u_h(c_{h0}) + \delta_h u_h(c_{h1})$.  The social planner's problem is now separable in dates and in states.  Given aggregate date--0 consumption $c_{m0}$ and aggregate date--1 consumption $c_{m1}$, the social planner solves 
$$U_0(c_{m0}) := \max \quad \sum_{h=1}^H \lambda_hu_h(c_{h0}) \quad \text{subject to} \quad \sum_{h=1}^H c_{h0} = c_{m0}$$
and
$$U_1(c_{m1}) := \max \quad \sum_{h=1}^H \lambda_h\delta_h u_h(c_{h1}) \quad \text{subject to} \quad \sum_{h=1}^H c_{h1} = c_{m1}$$
The envelope theorem tells us that, for all $h$,
$$U_0'(c_{m0}) = \lambda_h u_h'(c_{h0}) \quad \text{and} \quad U_1'(c_{m1}) = \lambda_h \delta_h u_h'(c_{h1})$$
So,
$$\frac{U_1'(c_{m1})}{U_0'(c_{m0})} = \frac{\delta_h u_h'(c_{h1})}{u_h'(c_{h0})} = \text{SDF}$$
\end{frame}

\bfr\frametitle{Common Discount Factors}
If all investors have the same discount factor $\delta$, then we can pull $\delta$ outside the sum in the definition of $U_1$ and see that, as functions, $U_1 = \delta U_0$.  

Writing $U=U_0$, an SDF is
$$\frac{\delta U'(\tilde{c}_{m1})}{U'(c_{m0})}$$
\end{frame}


\bfr\frametitle{Linear Risk Tolerance}
Suppose all investors have linear risk tolerance $\tau_h(c) = A_h + B c$ with same cautiousness parameter $B\ge 0$.  Then, the social planner's utility functions $U_0$ and $U_1$ have linear risk tolerance with the same cautiousness parameter.

Example: all investors have CRRA utility with risk aversion $\rho$ and the same discount factor $\delta$.  Then, an SDF is
$$\frac{\delta U'(\tilde{c}_{m1})}{U'(c_{m0})}$$
where
$$U(c) = \frac{1}{1-\rho}c^{1-\rho}$$
So, the SDF is
$$\delta \left(\frac{\tilde{c}_{m1}}{c_{m0}}\right)^{-\rho}$$

\end{frame}

\begin{frame}{Proof of LRT Social Planner in CARA Case}
We solved the social planner's problem in the last class and found
$$ w_h =\frac{\tau _h}{\tau }w -\frac{\tau _h}{\tau }\sum_{\ell =1}^H \tau _\ell \log (\lambda_\ell \alpha_\ell ) + \tau _h\log(\lambda_h\alpha_h)$$
which we wrote as $w_h = a_h + b_h w$ with $b_h = \tau_h/\tau$
So,
$$U(w) = -\sum_{h=1}^H \lambda_h \E^{-\alpha_h (a_h + b_h w)} = -\sum_{h=1}^H \lambda_h \E^{-\alpha_ha_h}\E^{-\alpha_hb_hw} $$
Moreover,
$$\alpha_h b_hw = \frac{\alpha_h \tau_hw}{\tau} = \frac{w}{\tau} = \alpha w$$
So
$$U(w) = -\E^{-\alpha w}\sum_{h=1}^H \lambda_h \E^{-\alpha_ha_h}$$
which is a monotone affine transform of CARA utility.
\end{frame}

\end{document}

\section{Consumption-Based Pricing}\subsection{}

\bfr\frametitle{Marginal Utility of Consumption is a Factor}

Now, write $u$ for social planner's (representative investor's) utility function.  Write $c_0$ for aggregate date--0 consumption and $\tc_1$ for aggregate date--1 consumption.

 Because $\tm \eqdef \delta u'(\tc_{1})/u'(c_{0})$ is an SDF, for all returns $\tr$,
\begin{align*}
\mye[\tr]- \frac{1}{\mye[\tm]} &= - \frac{1}{\mye[\tm]}\cov(\tr,\tm)\\
&= - \frac{1}{\mye[u'(\tc_{1})]} \cov(\tr, u'(\tc_{1}))
\end{align*}
 Most returns are negatively correlated with $u'(\tc_1)$: aggregate consumption is high when returns are high, so marginal utility is low when returns are high.

 If a return has a small negative or even positive correlation with marginal utility, then it is desirable for hedging and hence will trade at a high price, equivalently at a low risk premium.

\end{frame}

\bfr\frametitle{Market Return and CRRA SDF}
 Assume CRRA utility with risk aversion $\rho$, so $u'(c_1) = c_1^{-\rho}$.
 Assume $\tc_1$ is spanned by the assets.  Its cost is 
$$\mye[\tm\tc_1]= \mye\left[\frac{\delta \tc_1^{-\rho}}{c_0^{-\rho}}\tc_1\right] = c_0\mye\left[\frac{\delta \tc_1^{-\rho}}{c_0^{-\rho}}\cdot\frac{\tc_1}{c_0}\right] = \delta c_0\mye\left[\left(\frac{\tc_1}{c_0}\right)^{1-\rho}\right]$$
 The market return is 
 $$\tr_m := \frac{\tc_1 }{ \mye[\tm\tc_1]} = \frac{1}{\delta\mye\left[\left(\frac{\tc_1}{c_0}\right)^{1-\rho}\right]}\cdot \frac{\tc_1}{c_0} \equiv \frac{1}{\nu} \cdot \frac{\tc_1}{c_0}$$
 so 
 $$\tm = \delta \left(\frac{\tc_1}{c_0}\right)^{-\rho} = \delta \nu^{-\rho} \tr_m^{-\rho}$$
 \end{frame}
 
 \begin{frame}{CRRA Version of CAPM}
 Assume CRRA utility and assume a risk-free asset.  Then, for all returns $\tr$,
  \begin{align*}
\mye[\tr]-R_f = - R_f \cov(\tr,\tm) &= -\delta \nu^{-\rho}R_f \cov(\tr,\tr_m^{-\rho})\\
&= -\delta \nu^{-\rho}R_f \var(\tr_m^{-\rho}) \frac{\cov(\tr,\tr_m^{-\rho})}{\var(\tr_m^{-\rho})}\\ & = \lambda \cdot \text{beta with respect to $R_m^{-\rho}$}
\end{align*}
\end{frame}

\section{Equity Premium}\subsection{}
\bfr\frametitle{Risk-Free Rate and Equity Premium}
 The risk-free return is
$$R_f = \frac{1}{\mye[\tm]} = \frac{1}{\delta\mye[(\tc_1/c_0)^{-\rho}]}$$
and $\tr_m = (1/\nu)(\tc_1/c_0)$, so
\begin{align*}
\frac{\mye[\tr_m]}{R_f} = \frac{\mye[\tc_1/c_0]}{\nu R_f} = \frac{\delta\mye[(\tc_1/c_0)^{-\rho}]\mye[\tc_1/c_0]}{\nu} &=  \frac{\delta\mye[(\tc_1/c_0)^{-\rho}]\mye[\tc_1/c_0]}{\delta \mye[(\widetilde{c}_1/c_0)^{1-\rho}]}\\
&= \frac{\mye[\tc_1]\mye[\tc_1^{-\rho}]}{ \mye[\tc_1^{1-\rho}]}
\end{align*}
 

\end{frame}

\bfr\frametitle{CRRA Utility and Lognormal Consumption}
 Assume $\log \tc_1 - \log c_0 = \mu + \sigma \tve$ for constants $\mu$ and $\sigma$ and a standard normal $\tve$.
\begin{align*}\tc_1 = c_0\E^{\mu+\sigma\tve} \;&\Rightarrow\; \mye[\tc_1] = c_0\E^{\mu+\sigma^2/2}\\
\tc_1^{-\rho} = c_0^{-\rho}\E^{-\rho\mu-\rho\sigma\tve}\;&\Rightarrow\;\mye[\tc_1^{-\rho}] = c_0^{-\rho}\E^{-\rho\mu+\rho^2\sigma^2/2}\\
\tc_1^{1-\rho} = c_0^{1-\rho}\E^{(1-\rho)\mu + (1-\rho)\sigma\tve}\;&\Rightarrow\; \mye[\tc_1^{1-\rho}] = c_0^{1-\rho}\E^{(1-\rho)\mu+(1-\rho)^2\sigma^2/2}
\end{align*}

 This implies
$$\frac{\mye[\tr_m]}{R_f} = \E^{\rho\sigma^2}$$

\end{frame}

\bfr\frametitle{Equity Premium and Risk-Free Rate Puzzles}
\bi
\im To match this model to the historical equity premium, a risk aversion around 50 is required.  Much too high.
\im Using $\rho=10$ and $\delta=0.99$, the model implies a high risk-free rate (12.7\%) and low equity premium ($\mye[\tr_m]-R_f = 1.4\text{\%}$).
\im The historical (U.S.) numbers are around 1\% for the real risk-free rate and 6\% for the equity premium.
\ei
\end{frame}


\section{Coskewness-Cokurtosis}
\subsection{}

\bfr\frametitle{Coskewness-Cokurtosis Model}
Assume a representative investor with utility function $u$.  Do not assume CRRA utility or lognormal consumption.

Take a Taylor series expansion of the representative investor's marginal utility around the mean $\bar{w}$ of aggregate wealth $\tw$.  

Keeping only the first three terms, we have
\begin{multline*}
u'(\tw) \approx u'(\bar{w}) + u''(\bar{w})(\tw-\bar{w}) \\+ \frac{1}{2}u'''(\bar{w})(\tw-\bar{w})^2 + \frac{1}{6}u''''(\bar{w})(\tw-\bar{w})^3\,.
\end{multline*}

We are going to substitute this into the basic risk premium formula, remembering that $\tm \propto u'(\tw)$.
\end{frame}

%%%%%%%%%%%%%%%%%%%%%%%%%%%%%%%%%%%%%%%%%%%%%%%%%%%%

\begin{frame}
Assume that $\tw$ is spanned by the assets, so we can define the market return as $\widetilde{R}_m = \tw/\mye[\widetilde{m}\tw]$.  

From our usual formula, we get 
\begin{multline*}
\mye[\widetilde{R}] - R_f \approx \psi_1\cov\left(\widetilde{R},\widetilde{R}_m\right) \\+ \psi_2 \cov\left(\widetilde{R},(\widetilde{R}_m-\overline{R}_m)^2\right) + \psi_3 \cov\left(\widetilde{R},(\widetilde{R}_m-\overline{R}_m)^3\right)\,,
\end{multline*} 
The second covariance is called coskewness and the third is called cokurtosis.

The signs of the prices of risk $\psi_i$ are the opposite of the signs of $u''$, $u'''$, and $u''''$.  DARA implies $u'''>0$.  Decreasing absolute prudence implies $u''''<0$.  We assume investors dislike variance, dislike negative skewness, and dislike kurtosis.  

Therefore, they dislike positive covariance, dislike negative coskewness, and dislike positive cokurtosis.
\end{frame}

%%%%%%%%%%%%%%%%%%%%%%%%%%%%%%%%%%%%%%%%%%%%%%%%%%%%%%%%%%%

\section{Martin Bound}\subsection{}
\begin{frame}{Martin's 2017 QJE Bound}
Consider a return from date 0 to $T$ and the SDF $m$ for pricing claims at $T$.  The definition of the RNP is $\mye^*[x] = R_f\mye[mx]$.  So,
\begin{align*}
\var^*(R) & \ = \ \mye^*[R^2] - \mye^*[R]^2\\
& = \  R_f\mye[mR^2] - R_f^2\,.
\end{align*}
The definition of covariance and the fact that $\mye[mR]=1$ imply
$$\mye[mR^2] = \cov(mR,R) + \mye[R]\,.$$
We can substitute this into the previous formula and rearrange to obtain
$$\mye[R] - R_{f} = \frac{1}{R_{f}} \var^*(R) - \cov(mR,R)\,.$$
\end{frame}

%%%%%%%%%%%%%%%%%%%%%%%%%%%%%%%%%%%%%%%%%%%%%%%%%%%%%%%%%%%

\begin{frame}{Negative Correlation Condition}
The Negative Correlation Condition (NCC) is that 
$$\cov(mR,R)<0$$
It holds for the market portfolio if, for example, there is a representative investor with risk aversion greater than 1.

Martin's lower bound is
$$\frac{1}{R_{f}} \var^*(R)$$
and it holds if the NCC holds.
\end{frame}

%%%%%%%%%%%%%%%%%%%%%%%%%%%%%%%%%%%%%%%%%%%%%%%%%%%%%%%%%%%

\begin{frame}{Risk Neutral Variance}
The key is to calculate $\mye^*[R^2]$.  Write $R=S_T/S_0$ ($S=\,$ price).  For any real number $a$, we can trivially calculate 
$$a^2 \ = \ a \int_0^a \,\D x \ = \ \int_0^a a\,\D x$$
and
$$\frac{1}{2}a^2 \ = \ \int_0^a x\,\D x$$
so
$$a^2 \ = \ 2\int_0^a (a-x)\,\D x \ = \ 2 \int_0^\infty(a-x)^+\,\D x$$
Therefore
$$\mye^*[S_T^2] \ = \ 2 \int_0^\infty \mye^*[(S_T-K)^+]\,\D K$$
So, $\mye^*[S_T^2]$ is the value of a portfolio of call options. And/or puts using put-call parity.
\end{frame}

\begin{frame}{Ian Martin's Lower Bound}
\begin{center}\vspace*{-2\baselineskip}
	\includegraphics[width=0.8\textwidth]{ts_bounds_12.pdf} 
\end{center}
\vspace{-2\baselineskip}
Lower bound on the 12-month S\&P 500 risk premium, in \%.
\end{frame}
%%%%%%%%%%%%%%%%%%%%%%%%%%%%%%%%%%%%%%%%%%%%%%%%%%%%%%%%%%%

\begin{frame}\frametitle{Martin's Bound: Financial Crisis}
\begin{center}\vspace*{-\baselineskip}
	\includegraphics[width=0.8\textwidth]{fig1a.pdf} 
\end{center}
\vspace{-2\baselineskip}
Lower bound on the S\&P 500 risk premium at different horizons, annualized in \%, derived from S\&P 500 option prices.  
\end{frame}

%%%%%%%%%%%%%%%%%%%%%%%%%%%%%%%%%%%%%%%%%%%%%%%%%%%%%%%%%%%

\begin{frame}\frametitle{Martin's Bound: Corona Crisis}
\begin{center}\vspace*{-\baselineskip}
	\includegraphics[width=0.8\textwidth]{fig1b.pdf} 
\end{center}
\vspace{-2\baselineskip}
Lower bound on the S\&P 500 risk premium at different horizons, annualized in \%, derived from S\&P 500 option prices. From Back, Crotty, Kazempour (2020).
\end{frame}

%%%%%%%%%%%%%%%%%%%%%%%%%%%%%%%%%%%%%%%%%%%%%%%%%%%%%%%%%%%

\begin{frame}{Martin Bound and Subsequent Returns}

Daily data on 12-month bound binned into 10-basis point bins  \\ Subsequent slackness ($r_m-r_f-Bound$) averaged within bins
\vspace*{-\baselineskip}
\begin{center}
	\includegraphics[width=0.7\textwidth]{Seminar0.pdf} 
\end{center}
\vspace{-2\baselineskip}
The bound is above 5.6\% on 17\% of the days.  Mean realized slackness is positive in every one of those bins. From Back, Crotty, Kazempour (2020).

\begin{document}