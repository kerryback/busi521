\documentclass[xcolor=dvipsnames,10pt]{beamer}

\mode<presentation> {\usetheme{Singapore}}
\usepackage{pgfpages}

%\setbeamercovered{transparent} 
\usepackage[english]{babel}
\usepackage[latin1]{inputenc}
\usepackage{times,amsfonts}
\usepackage[T1]{fontenc}
\setlength{\parskip}{\baselineskip}
% Or whatever. Note that the encoding and the font should match. If T1
% does not look nice, try deleting the line with the fontenc.

%\usecolortheme{sidebartab}
\setbeamertemplate{itemize item}[triangle]



\usepackage{calc}
\usepackage{environ}
\newcommand{\halfmargin}{0.0001\paperwidth}


\RequirePackage{booktabs,colortbl,ulem}

\usepackage{animate}
\RequirePackage{booktabs,colortbl,gensymb}
\setlength{\parskip}{\baselineskip}

\usepackage{calc}
\usepackage{environ}

% \newcommand{\halfmargin}{0.0001\paperwidth}


\NewEnviron{wideframe}[1][]{%
\begin{frame}{#1}
\makebox[\textwidth][c]{
\begin{minipage}{\dimexpr\paperwidth-\halfmargin-\halfmargin\relax}
\BODY
\end{minipage}}
\end{frame}
}


\DeclareMathOperator{\stdev}{stdev}
\DeclareMathOperator{\var}{var}
\DeclareMathOperator{\cov}{cov}
\DeclareMathOperator{\corr}{corr}
\DeclareMathOperator{\prob}{prob}
\DeclareMathOperator{\n}{n}
\DeclareMathOperator{\N}{N}
\DeclareMathOperator{\Cov}{Cov}

\newcommand{\hlf}{\frac{1}{2}}
\newcommand{\bi}{\begin{itemize}}
\newcommand{\ei}{\end{itemize}}
\newcommand{\im}{\item}
\newcommand{\D}{\mathrm{d}}
\newcommand{\E}{\mathrm{e}}
\newcommand{\mye}{\ensuremath{\mathsf{E}}}
\newcommand{\myreal}{\ensuremath{\mathbb{R}}}
\newcommand{\bq}{\begin{equation}}
\newcommand{\eq}{\end{equation}}
\newcommand{\eqdef}{\;\buildrel \text{d{}ef}\over = \;}
\newcommand{\xstar}{\buildrel *\over X}
\newcommand{\pmax}{p^{\text{max}}}
\newcommand{\qmax}{q^{\text{max}}}
\newcommand{\bfr}{\begin{frame}}
\newcommand{\bfrp}{\begin{frame}[plain]}
\newcommand{\efr}{\end{frame}}
\newcommand{\F}{\mathcal{F}}
\newcommand{\FF}{\mathbb{F}}
\newcommand{\ve}{\varepsilon}
\newcommand{\lh}{\hat{\lambda}}
\definecolor{mycolor}{gray}{0.8}
\definecolor{mymaincolor}{rgb}{0.6862745098039216,0.9333333333333333,0.9333333333333333}
\newcommand{\alr}[1]{\textcolor{blue}{#1}}
\definecolor{LightCyan}{rgb}{0.88,1,1}
\newcommand{\yel}{\cellcolor{yellow}}
\newcommand{\blue}{\cellcolor{SkyBlue}}
\newcommand{\gr}{\cellcolor{SpringGreen}}
\newcommand{\pink}{\cellcolor{pink}}
\newcommand{\apr}{\cellcolor{Apricot}}
\newcommand{\tve}{\tilde{\varepsilon}}
\newcommand{\tw}{\tilde{w}}
\newcommand{\ttth}{\tilde{\theta}}
\newcommand{\te}{\tilde{e}}
\newcommand{\ts}{\tilde{s}}
\newcommand{\tx}{\tilde{x}}
\newcommand{\ty}{\tilde{y}}
\newcommand{\tv}{\tilde{v}}
\newcommand{\tp}{\tilde{p}}
\newcommand{\tF}{\tilde{F}}
\newcommand{\tf}{\tilde{f}}
\newcommand{\tZ}{\tilde{Z}}
\newcommand{\ow}{\overline{w}}
\newcommand{\lb}{\left[}
\newcommand{\rb}{\right]}
\newcommand{\lp}{\left(}
\newcommand{\rp}{\right)}
\newcommand{\tm}{\tilde{m}}
\newcommand{\tc}{\tilde{c}}
\newcommand{\tz}{\tilde{z}}
\newcommand{\str}[1]{\textcolor{blue}{\sout{#1}}}
\newcommand{\tr}{\widetilde{R}}
\newcommand{\tR}{\widetilde{\mathbf{R}}}
\newcommand{\bms}{\begin{multline*}}
\newcommand{\ems}{\end{multline*}}
\newcommand{\bas}{\begin{align*}}
\newcommand{\eas}{\end{align*}}
\newcommand{\qr}{\mathbb{Q}}
\newcommand{\IMAGES}{/home/kerry/Dropbox/Images}
\newcommand{\tX}{\tilde{X}}
\newcommand{\tY}{\tilde{Y}}

\author{\vskip 0.5in \small Kerry Back \\BUSI 521--ECON 505\\ Rice University \\Spring 2022}
%\institute{Rice University\\ Spring 2019}
\date[]





\newcommand{\tu}{\tilde{u}}
\begin{document}
\title{\vskip 0.5in Day 25}
\subtitle{Perpetual Options}

\begin{frame}
  \titlepage
\end{frame}

\begin{frame}{Set-up}
   Single risky asset with price $S$ and constant volatility $\sigma$, single Brownian motion, constant risk-free rate
   
   Dividend paid by risky asset in time period $\D t$ is $qS_t\,\D t$ for constant $q$ (``dividend yield'')
   
   Total return is
   $$\frac{\D S + qS\,\D t}{S} = \frac{\D S}{S} + q\,\D t$$
   Total expected return under RNP is risk-free rate, so
   $$\frac{\D S}{S} = (r-q)\,\D t + \sigma\,\D B^*$$
   for a risk-neutral Brownian motion $B^*$
\end{frame}

\begin{frame}{Perpetual Call}

Perpetual call option with strike $K$

Why exercise?  To capture the dividend.  But the asset price and dividend must be high enough before it is optimal to do so.

An example of a strategy is to pick a number $x$ and exercise the first time $S_t$ gets up to $x$.  The optimal strategy will be of this type.

The problem of finding the optimal exercise time is in the class of problems often called optimal stopping.
\end{frame}

\begin{frame}{Exercise Boundary}
We will first calculate the value if we exercise the first time $S_t$ gets up to $x$ for an arbitrary $x>S_0$.

Let $\tau = \inf\{t \mid S_t \ge x\}$.  This is called the hitting time of $x$.

By the time-homogeneity of $S$, the value at any $t<\tau$ depends only on $S_t$.  Call it $f(S_t)$.  
    
    More formally,
    $$f(s) = \mye^*[\E^{-r\tau}((x-K) \mid S_0=s] = \mye^*[\E^{-r(\tau-t)} (x-K)\mid S_t=s]$$
\end{frame}
\begin{frame}{Fundamental ODE}
    
    The fundamental ODE is 
    $$\frac{\text{drift$^*$ of $f$}}{f} = r$$
    which is
    $$(r-q)Sf' + \frac{1}{2}\sigma^2S^2f'' = rf$$
    Trying a power solution $f(S) = S^\gamma$, we see that $f$ satisfies the ODE if and only if
    $$(r-q)\gamma + \frac{1}{2}\sigma^2\gamma(\gamma-1) = r$$
    The quadratic formula shows that there are two real roots of this equation.  One is negative and the other is greater than 1.  
    

\end{frame}
    
\begin{frame}{General Solution and Boundary Conditions}
Let $\gamma=\,$ absolute value of negative root, and $\beta=\,$ positive root.  The general solution of the ODE is
$$aS^{-\gamma} + bS^\beta$$
for constants $a$ and $b$ that must be determined by boundary conditions.

The value $f$ of the call exercised at the hitting time of $x$ satisfies $f(0)=0$ and $f(x) = x-K$.  The condition $f(0)=0$ implies $a=0$, and the condition $f(x)=x-K$ implies $b=(x-K)x^{-\beta}$.  

The value of the call is
$$f(S_t) = (x-K)\left(\frac{S_t}{x}\right)^\beta$$
    
\end{frame}

\begin{frame}{Optimal Stopping}

To optimize, maximize $(x-K)\left(\frac{S_t}{x}\right)^\beta$ over $x$. 
The factor $S_t^\beta$ is a positive constant and is irrelevant for determining the optimum, so we can maximize
$$(x-K)x^{-\beta} = x^{1-\beta} - Kx^{-\beta}$$

The FOC is
$$(1-\beta)x^{-\beta} + \beta Kx^{-\beta-1}=0$$
Equivalently,
$$(1-\beta)x + \beta K=0$$
So, 
$$x^* = \frac{\beta}{\beta-1}K$$
    
\end{frame}

\begin{frame}{Perpetual Put}

Recall that the general solution of the ODe is $f(s) = as^{-\gamma} + bs^\beta$.

For a put, we exercise the first time $S_t$ drops to a boundary $x$.  
The boundary conditions for a put are $f(\infty)=0$, and $f(x)=K-x$.  The condition $f(\infty)=0$ implies $b=0$.  The condition $f(x)=K-x$ implies $a=(K-x)x^{\gamma}$.

So, the put value is 
$$f(S_t) = (K-x)\left(\frac{x}{S_t}\right)^\gamma$$
The FOC for maximizing over $x$ is
$$\gamma K x^{\gamma-1} - (1+\gamma)x^\gamma =0$$
Maximizing over $x$ yields $x^*=\gamma K/(1+\gamma)$.

    
\end{frame}
\end{document}