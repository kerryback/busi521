aaaaaaaaaaaaaa\documentclass[xcolor=dvipsnames,10pt]{beamer}

\mode<presentation> {\usetheme{Singapore}}
\usepackage{pgfpages}

%\setbeamercovered{transparent} 
\usepackage[english]{babel}
\usepackage[latin1]{inputenc}
\usepackage{times,amsfonts}
\usepackage[T1]{fontenc}
\setlength{\parskip}{\baselineskip}
% Or whatever. Note that the encoding and the font should match. If T1
% does not look nice, try deleting the line with the fontenc.

%\usecolortheme{sidebartab}
\setbeamertemplate{itemize item}[triangle]



\usepackage{calc}
\usepackage{environ}
\newcommand{\halfmargin}{0.0001\paperwidth}


\RequirePackage{booktabs,colortbl,ulem}

\usepackage{animate}
\RequirePackage{booktabs,colortbl,gensymb}
\setlength{\parskip}{\baselineskip}

\usepackage{calc}
\usepackage{environ}

% \newcommand{\halfmargin}{0.0001\paperwidth}


\NewEnviron{wideframe}[1][]{%
\begin{frame}{#1}
\makebox[\textwidth][c]{
\begin{minipage}{\dimexpr\paperwidth-\halfmargin-\halfmargin\relax}
\BODY
\end{minipage}}
\end{frame}
}


\DeclareMathOperator{\stdev}{stdev}
\DeclareMathOperator{\var}{var}
\DeclareMathOperator{\cov}{cov}
\DeclareMathOperator{\corr}{corr}
\DeclareMathOperator{\prob}{prob}
\DeclareMathOperator{\n}{n}
\DeclareMathOperator{\N}{N}
\DeclareMathOperator{\Cov}{Cov}

\newcommand{\hlf}{\frac{1}{2}}
\newcommand{\bi}{\begin{itemize}}
\newcommand{\ei}{\end{itemize}}
\newcommand{\im}{\item}
\newcommand{\D}{\mathrm{d}}
\newcommand{\E}{\mathrm{e}}
\newcommand{\mye}{\ensuremath{\mathsf{E}}}
\newcommand{\myreal}{\ensuremath{\mathbb{R}}}
\newcommand{\bq}{\begin{equation}}
\newcommand{\eq}{\end{equation}}
\newcommand{\eqdef}{\;\buildrel \text{d{}ef}\over = \;}
\newcommand{\xstar}{\buildrel *\over X}
\newcommand{\pmax}{p^{\text{max}}}
\newcommand{\qmax}{q^{\text{max}}}
\newcommand{\bfr}{\begin{frame}}
\newcommand{\bfrp}{\begin{frame}[plain]}
\newcommand{\efr}{\end{frame}}
\newcommand{\F}{\mathcal{F}}
\newcommand{\FF}{\mathbb{F}}
\newcommand{\ve}{\varepsilon}
\newcommand{\lh}{\hat{\lambda}}
\definecolor{mycolor}{gray}{0.8}
\definecolor{mymaincolor}{rgb}{0.6862745098039216,0.9333333333333333,0.9333333333333333}
\newcommand{\alr}[1]{\textcolor{blue}{#1}}
\definecolor{LightCyan}{rgb}{0.88,1,1}
\newcommand{\yel}{\cellcolor{yellow}}
\newcommand{\blue}{\cellcolor{SkyBlue}}
\newcommand{\gr}{\cellcolor{SpringGreen}}
\newcommand{\pink}{\cellcolor{pink}}
\newcommand{\apr}{\cellcolor{Apricot}}
\newcommand{\tve}{\tilde{\varepsilon}}
\newcommand{\tw}{\tilde{w}}
\newcommand{\ttth}{\tilde{\theta}}
\newcommand{\te}{\tilde{e}}
\newcommand{\ts}{\tilde{s}}
\newcommand{\tx}{\tilde{x}}
\newcommand{\ty}{\tilde{y}}
\newcommand{\tv}{\tilde{v}}
\newcommand{\tp}{\tilde{p}}
\newcommand{\tF}{\tilde{F}}
\newcommand{\tf}{\tilde{f}}
\newcommand{\tZ}{\tilde{Z}}
\newcommand{\ow}{\overline{w}}
\newcommand{\lb}{\left[}
\newcommand{\rb}{\right]}
\newcommand{\lp}{\left(}
\newcommand{\rp}{\right)}
\newcommand{\tm}{\tilde{m}}
\newcommand{\tc}{\tilde{c}}
\newcommand{\tz}{\tilde{z}}
\newcommand{\str}[1]{\textcolor{blue}{\sout{#1}}}
\newcommand{\tr}{\widetilde{R}}
\newcommand{\tR}{\widetilde{\mathbf{R}}}
\newcommand{\bms}{\begin{multline*}}
\newcommand{\ems}{\end{multline*}}
\newcommand{\bas}{\begin{align*}}
\newcommand{\eas}{\end{align*}}
\newcommand{\qr}{\mathbb{Q}}
\newcommand{\IMAGES}{/home/kerry/Dropbox/Images}
\newcommand{\tX}{\tilde{X}}
\newcommand{\tY}{\tilde{Y}}

\author{\vskip 0.5in \small Kerry Back \\BUSI 521--ECON 505\\ Rice University \\Spring 2022}
%\institute{Rice University\\ Spring 2019}
\date[]






\title{\vskip 0.5in Day 3}
\subtitle{Stochastic Discount Factors I}

\begin{document}

%%%%%%%%%%%%%%%%%%%%%%%%%%%%%%%%%%%%%%%%%%%%%%%%%%%%%%%%%%%%%%%%%%%%%%%

\begin{frame}[plain]
  \titlepage
\end{frame}

%%%%%%%%%%%%%%%%%%%%%%%%%%%%%%%%%%%%%%%%%%%%%%%%%%%%%%%%%%%%%%%%%%%%%%%





\begin{frame}{Main Concepts}

\begin{enumerate}
    \item Day I
\begin{itemize}
\im Stochastic discount factors (SDFs)
\im Risk premia and risk adjustments
\im Risk-neutral probabilities
\end{itemize}
\item Day II
\begin{itemize}
\im Complete markets
\im Finite state model
\im Arbitrage pricing
\im Projections and the Hansen-Jagannathan bound
\end{itemize}
\end{enumerate}
\end{frame}

\section{SDFs}
\subsection{}

\bfr\frametitle{Discounting Future Cash Flows}
 The value of an asset depends on future cash flows.  We discount future cash flows because of the time value of money and because of risk.

If investors were risk neutral, an asset price would be $p = \mye[\tx]/R_f$ (discounting at the risk-free rate).
\
If investors were risk neutral, all expected returns would be the same.  Proof:
$$\tr = \tx \frac{1}{p} = \tx \cdot \frac{R_f}{\mye[\tx]}\,,$$
so
$$\mye[\tr] = R_f$$

Why? If an asset had a higher (lower) expected return than the risk-free return then risk-neutral investors would want to go infinitely long (short) the asset, which is inconsistent with equilibrium.
\end{frame}

\bfr\frametitle{Stochastic Discount Factors}

Under weak assumptions that we will discuss, there exists a random variable $\tm$, which we call a stochastic discount factor (SDF) such that all asset prices are 
$$p = \mye[\tm\tx]$$

With risk neutrality, $\tm = 1/R_f$.

\end{frame}

\bfr\frametitle{Alternate Characterizations}
Suppose $\tm$ is an SDF, meaning
$$(\forall \,\text{assets\,}\,i) \quad \mye[\tm \tx_i] = p_i$$

If $p_i \neq 0$, then
$$\mye[\tm \tr_i] = 1$$

If also $p_j \neq 0$, then
$$\mye[\tm(\tr_i-\tr_j)] = 0$$

Notice that $\mye[\tm R_f]=1$ implies $\mye[\tm] = 1/R_f$.

\end{frame}



\bfr\frametitle{Marginal Utility Defines an SDF}
The FOC is
$$(\forall \,i) \quad \mye[u'(\tw) \tx_i] = \lambda p_i$$

So, $u'(\tw)/\lambda$ is an SDF.  

More succinctly, $u'(\tw)$ is proportional to an SDF.

\end{frame}

\bfr\frametitle{Time-Additive Utility and the Euler Equation}
\bi
\im Assume time-additive utility $ u(c_0) + \delta u(c_1)$.
\im Consumption/investment problem:  choose $c_0, \phi_1, \ldots, \phi_n$ to 
$$\max \;u(c_0) +  \delta \mye\lb  u\lp \sum_{i=1}^n \phi_i\tr_i \rp \rb \quad \text{subject to} \quad c_0 + \sum_{i=1}^n \phi_i = w$$
\im FOC: 
$$u'(c_0) = \lambda \quad \text{and} \quad (\forall \,i) \quad  \mye\lb \delta u'\lp \sum_{i=1}^n \phi_i\tr_i \rp \tr_i\rb = \lambda \quad \Rightarrow$$
$$\Rightarrow \quad (\forall \,i) \quad \mye\lb \frac{\delta u'\lp \sum_{i=1}^n \phi_i\tr_i \rp}{u'(c_0)} \tr_i\rb = 1 \quad \Rightarrow \quad
\tm = \frac{\delta u'(\tilde{c}_1)}{u'(c_0)}$$
\ei
\end{frame}

\section{Risk Premia}
\subsection{}

\bfr\frametitle{Risk Premia as Covariances}
Because $\cov(\tx,\ty) = \mye[\tx\ty] - \mye[\tx]\mye[\ty]$,
$$1 = \mye[\tm\tr] = \cov(\tm,\tr) + \mye[\tm]\mye[\tr]$$

Rearrange to obtain
$$\mye[\tr] = \frac{1}{\mye[\tm]} - \frac{1}{\mye[\tm]}\cov(\tm,\tr)$$
Thus, expected asset returns depend on covariances with an SDF.

If there is a risk-free asset, then $R_f = 1/\mye[\tm]$.  So,
$$\mye[\tr]-R_f = - R_f\cov(\tm,\tr)$$
\end{frame}

\bfr\frametitle{Risk Adjustments}
If there is a risk-free asset, then
$$p \ = \ \mye[\tilde{m}\tilde{x}]  \ = \ \mye[\tilde{m}]\mye[\tilde{x}]+ \cov(\tilde{m},\tilde{x})
\ = \ \frac{\mye[\tilde{x}] + R_f\cov(\tilde{m},\tilde{x})}{R_f}$$
So, we can adjust the expected payoff for risk and then discount at the risk-free rate.

Also,
$$\mye[\tx/p]-R_f \ = \ - R_f\cov(\tm,\tr)\quad \Rightarrow \quad p \ = \ \frac{\mye[\tilde{x}]}{R_f - R_f\cov(\tilde{m},\tilde{R})}$$
So we can discount the expected payoff at a risk-adjusted rate. 

\end{frame}

\section{RNPs}
\subsection{}

\bfr\frametitle{Risk Neutral Probabilities}
We can price with an SDF: $p = \mye[\tm\tx]$.

Alternatively, given an SDF, we can ``change probabilities'' to price \alert{as if} investors were risk neutral.  That is, we can price as
$$p \ = \ \frac{\mye^*[\tx]}{R_f}$$
where $\mye^*$ denotes expectation with respect to the changed probabilities.

We call the new probabilities a ``risk-neutral probability'' or ``risk-neutral probability measure.''

\end{frame}

\begin{frame}{Example}
$R_f=1.1$. Single risky asset.  Two states of the world.  $p=100$, $x=120$ or 90 with equal probabilities.

Find the SDF.
\begin{align*}
1  \ & = \ 1.1\prob_1m_1  + 1.1\prob_2m_2  \\
100 \ & = \ 120\prob_1 m_1  + 90\prob_2 m_2
\end{align*}

Find the risk-neutral probability:
\begin{align*}
1  \ & = \ \prob_1^* + \prob_2^*\\
100 \ & = \frac{\ 120 \prob_1^* + 90 \prob_2^*}{1.1} 
\end{align*}

Clearly, $\prob_1^* = 1.1\prob_1 m_1$ and $\prob_2^* = 1.1\prob_2m_2$.
\end{frame}

\begin{frame}{Example cont.}
Let's price another asset in the example.  Consider a security that pays \$1 if $x=120$ and 0 otherwise.  

Price using SDF: ???
\vskip 3\baselineskip
Price using RNP: ???
\end{frame}

\bfr\frametitle{General Definition of RNP}
For an event $A$, $1_A=\,$ indicator function, meaning $1_A(\omega) = 1$ if $\omega \in A$ and $1_A(\omega) = 0$ otherwise.

Given an SDF $\tm$, define 
$$\qr(A) = \frac{ \mye[\tilde{m}1_A]}{\mye[\tm]} = R_f \mye[\tilde{m}1_A]$$

Then, $\qr$ is a probability: $\qr(A) \geq 0$, $\qr(\emptyset)=0$, $\qr(\Omega)=1$, and $\qr$ of the union of a sequence of disjoint events is the sum of $\qr$ of the events.

In the example: \vspace*{-\baselineskip}
\begin{itemize}
    \item For $A=\{\tx=120\}$,
$\qr(A) \eqdef \prob_1^* = 1.1 \prob_1 m_1$.
\item For $A = \{\tx=90\}$,
$\qr(A) \eqdef \prob_2^* = 1.1 \prob_2 m_2$.
\end{itemize}


\end{frame}


\begin{frame}{Risk-Neutral Expectation}
Let $\mye^*$ denote expectation with respect to $\qr$.  Note $\qr(A) = \mye^*[1_A]$ for all events $A$.  

So, the definition of $\qr$ implies
$$(\forall\, A) \quad \mye^*[1_A] = R_f \mye[\tm 1_A]$$

This generalizes as
$$(\forall\, \tx) \quad \mye^*[\tx] = R_f \mye[\tm \tx]$$


It follows that \alert{prices are risk-neutral expected payoffs discounted at the risk-free rate, and risk-neutral expected returns equal the risk-free return}: 
$$p = \mye[\tm\tx] = \frac{\mye^*[\tx]}{R_f} \quad \Rightarrow \quad \mye^*[\tilde{R}] = \frac{\mye^*[\tx]}{p} = R_f$$

\end{frame}

\section{CARA-Normal}
\subsection{}

\begin{frame}{CARA Utility}
Suppose there is a CARA investor with risk aversion $\alpha$ whose optimal portfolio has a value of $\tw$ at date 1. 
Her marginal utility is
$$u'(\tw) \ = \ \alpha \E^{-\alpha \tw}$$
There is an SDF proportional to her marginal utility, \alert{meaning
$$\tm  =k \E^{-\alpha \tw}$$
for some $k>0$.  
Because $\mye[\tm]=1/R_f$, we have
$$k \mye[\E^{-\alpha \tw}] = \frac{1}{R_f} \quad \Rightarrow \quad k = \frac{1}{R_f \mye[\E^{-\alpha \tw}]}$$ 
$$\Rightarrow \quad \tm \ = \ \frac{\E^{-\alpha \tw}}{R_f \mye[\E^{-\alpha \tw}]}$$}
\end{frame}

\begin{frame}{Normal Distributions}
Consider an asset with payoff $\tx$.  Suppose $\tx$ and $\tw$ are joint normal, and the investor has CARA utility.

We want to value $\tx$.  Possible approaches:
\begin{enumerate}
\im $p = \mye[\tm\tx]$
\im $p = (\mye[\tx] + R_f\cov(\tm,\tx))/R_f$
\im $p = \mye^*[\tx] / R_f$
\end{enumerate}
From CARA utility,
$$\tm \ = \ \frac{\E^{-\alpha \tw}}{R_f \mye[\E^{-\alpha \tw}]}$$
\end{frame}

\begin{frame}{Method 1}
    Setting \alert{$b=\cov(\tx,\tw)/\var(\tw)$} and $a=\mye[\tx]-b\mye[\tw]$, we have
    $$\tx = a + b\tw + \te$$
    where $\te$ has a zero mean, and $\te$ and $\tw$ are uncorrelated (easy to check from definitions) and hence independent (by normality).
    Thus,
    $$\mye[\E^{-\alpha \tw}\tx] = a\mye[\E^{-\alpha \tw}] + b\mye[\tw \E^{-\alpha \tw}]$$
    We can integrate by parts (using the fact that for the standard normal density $f$, we have $xf(x) = - f'(x)$) or look up in a table of integrals that
    $$\mye[\tw \E^{-\alpha \tw}] = (\mu_w-\alpha \sigma_w^2)\mye[\E^{-\alpha \tw}]$$
    We end up with 
    $$p = \frac{a + b \mu_w - \alpha b \sigma_w^2}{R_f} = \frac{\mye[\tx] - \alpha \cov(\tx,\tw)}{R_f}$$
    \end{frame}
\begin{frame}{Method 2 via Stein's Lemma}
Stein's Lemma: For any two joint normal random variables $\ty$ and $\tz$ and any continuously differentiable function $f$, 
$$\cov(\ty, f(\tz)) = \mye[f'(\tz)]\cov(\ty,\tz)$$
Intuition: suppose $f$ is linear: $f(z) = a + bz$. Then $f'=b$ and the lemma states that $\cov(\ty,a+b\tz) = b\cov(\ty,\tz)$, which we know.  What the lemma says is that we can approximate $f$ linearly by its derivative and replace $f'$ by its mean when $\ty$ and $\tz$ are joint normal.

Stein's Lemma implies
$$\cov(\tx,\E^{-\alpha \tw}) \ = \ -\alpha \mye[\E^{-\alpha w}]\cov(\tx,\tw)$$
Hence,
$$p \ = \ \frac{\mye[\tx] - \alpha \cov(\tx,\tw)}{R_f}$$
\end{frame}

\begin{frame}{Method 3}
We can show that, under the risk-neutral probability defined from $\tm$, the random variable $\tx$ is normally distributed with the same variance as under the original probability and with mean $\mye^*[\tx] = \mye[\tx] - \alpha \cov(\tx,\tw)$.

We end up with
$$p \ = \ \frac{\mye[\tx] - \alpha \cov(\tx,\tw)}{R_f}$$

To verify the distribution of $\tx$ under $\qr$, we calculate the moment-generating function
$$\mye^*[\E^{\lambda \tx}] = R_f \mye[\tm \E^{\lambda \tx}] = \frac{\mye[\E^{\lambda \tx -\alpha \tw}]}{\mye[\E^{-\alpha \tw}]}$$
\end{frame}

\begin{frame}[plain]
    We have
 \begin{align*}
     \mye[\E^{\lambda \tx -\alpha \tw}] &= \E^{\lambda a}\mye[\E^{(\lambda b-\alpha)\tw}]\mye[\E^{\lambda \te}]\\
     &= \E^{\lambda a}\E^{(\lambda b-\alpha)\mu_w + (\lambda b-\alpha)^2\sigma_w^2/2}\E^{\lambda^2\sigma^2_e/2}\\
     &= \E^{\lambda a+\lambda b\mu_w +\lambda^2b^2\sigma_w^2/2 - \lambda b \alpha \sigma_w^2 + \lambda^2\sigma_e^2/2}\mye[\E^{-\alpha \tw}]
 \end{align*} 
 Thus,
 \begin{align*}
     \mye^*[\E^{\lambda \tx}] & = \E^{\lambda a+\lambda b\mu_w +\lambda^2b^2\sigma_w^2/2 - \lambda b \alpha \sigma_w^2 + \lambda^2\sigma_e^2/2}\\
     &= \E^{\lambda (a + b \mu_w - b \alpha \sigma_w^2) + \lambda^2(b^2\sigma_w^2 + \sigma_e^2)/2}\\
     &= \E^{\lambda \big(\mu_x - \alpha \cov(\tx,\tw)\big) + \lambda^2\sigma_x^2/2}
     \end{align*}
Therefore, $\tx$ is normal with mean $\mye[\tx] - \alpha \cov(\tx,\tw)$ and standard deviation $\sigma_x$ under the RNP.
\end{frame}

\end{document}