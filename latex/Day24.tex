\documentclass[xcolor=dvipsnames,10pt]{beamer}

\mode<presentation> {\usetheme{Singapore}}
\usepackage{pgfpages}

%\setbeamercovered{transparent} 
\usepackage[english]{babel}
\usepackage[latin1]{inputenc}
\usepackage{times,amsfonts}
\usepackage[T1]{fontenc}
\setlength{\parskip}{\baselineskip}
% Or whatever. Note that the encoding and the font should match. If T1
% does not look nice, try deleting the line with the fontenc.

%\usecolortheme{sidebartab}
\setbeamertemplate{itemize item}[triangle]



\usepackage{calc}
\usepackage{environ}
\newcommand{\halfmargin}{0.0001\paperwidth}


\RequirePackage{booktabs,colortbl,ulem}

\usepackage{animate}
\RequirePackage{booktabs,colortbl,gensymb}
\setlength{\parskip}{\baselineskip}

\usepackage{calc}
\usepackage{environ}

% \newcommand{\halfmargin}{0.0001\paperwidth}


\NewEnviron{wideframe}[1][]{%
\begin{frame}{#1}
\makebox[\textwidth][c]{
\begin{minipage}{\dimexpr\paperwidth-\halfmargin-\halfmargin\relax}
\BODY
\end{minipage}}
\end{frame}
}


\DeclareMathOperator{\stdev}{stdev}
\DeclareMathOperator{\var}{var}
\DeclareMathOperator{\cov}{cov}
\DeclareMathOperator{\corr}{corr}
\DeclareMathOperator{\prob}{prob}
\DeclareMathOperator{\n}{n}
\DeclareMathOperator{\N}{N}
\DeclareMathOperator{\Cov}{Cov}

\newcommand{\hlf}{\frac{1}{2}}
\newcommand{\bi}{\begin{itemize}}
\newcommand{\ei}{\end{itemize}}
\newcommand{\im}{\item}
\newcommand{\D}{\mathrm{d}}
\newcommand{\E}{\mathrm{e}}
\newcommand{\mye}{\ensuremath{\mathsf{E}}}
\newcommand{\myreal}{\ensuremath{\mathbb{R}}}
\newcommand{\bq}{\begin{equation}}
\newcommand{\eq}{\end{equation}}
\newcommand{\eqdef}{\;\buildrel \text{d{}ef}\over = \;}
\newcommand{\xstar}{\buildrel *\over X}
\newcommand{\pmax}{p^{\text{max}}}
\newcommand{\qmax}{q^{\text{max}}}
\newcommand{\bfr}{\begin{frame}}
\newcommand{\bfrp}{\begin{frame}[plain]}
\newcommand{\efr}{\end{frame}}
\newcommand{\F}{\mathcal{F}}
\newcommand{\FF}{\mathbb{F}}
\newcommand{\ve}{\varepsilon}
\newcommand{\lh}{\hat{\lambda}}
\definecolor{mycolor}{gray}{0.8}
\definecolor{mymaincolor}{rgb}{0.6862745098039216,0.9333333333333333,0.9333333333333333}
\newcommand{\alr}[1]{\textcolor{blue}{#1}}
\definecolor{LightCyan}{rgb}{0.88,1,1}
\newcommand{\yel}{\cellcolor{yellow}}
\newcommand{\blue}{\cellcolor{SkyBlue}}
\newcommand{\gr}{\cellcolor{SpringGreen}}
\newcommand{\pink}{\cellcolor{pink}}
\newcommand{\apr}{\cellcolor{Apricot}}
\newcommand{\tve}{\tilde{\varepsilon}}
\newcommand{\tw}{\tilde{w}}
\newcommand{\ttth}{\tilde{\theta}}
\newcommand{\te}{\tilde{e}}
\newcommand{\ts}{\tilde{s}}
\newcommand{\tx}{\tilde{x}}
\newcommand{\ty}{\tilde{y}}
\newcommand{\tv}{\tilde{v}}
\newcommand{\tp}{\tilde{p}}
\newcommand{\tF}{\tilde{F}}
\newcommand{\tf}{\tilde{f}}
\newcommand{\tZ}{\tilde{Z}}
\newcommand{\ow}{\overline{w}}
\newcommand{\lb}{\left[}
\newcommand{\rb}{\right]}
\newcommand{\lp}{\left(}
\newcommand{\rp}{\right)}
\newcommand{\tm}{\tilde{m}}
\newcommand{\tc}{\tilde{c}}
\newcommand{\tz}{\tilde{z}}
\newcommand{\str}[1]{\textcolor{blue}{\sout{#1}}}
\newcommand{\tr}{\widetilde{R}}
\newcommand{\tR}{\widetilde{\mathbf{R}}}
\newcommand{\bms}{\begin{multline*}}
\newcommand{\ems}{\end{multline*}}
\newcommand{\bas}{\begin{align*}}
\newcommand{\eas}{\end{align*}}
\newcommand{\qr}{\mathbb{Q}}
\newcommand{\IMAGES}{/home/kerry/Dropbox/Images}
\newcommand{\tX}{\tilde{X}}
\newcommand{\tY}{\tilde{Y}}

\author{\vskip 0.5in \small Kerry Back \\BUSI 521--ECON 505\\ Rice University \\Spring 2022}
%\institute{Rice University\\ Spring 2019}
\date[]





\newcommand{\tu}{\tilde{u}}
\begin{document}
\title{\vskip 0.5in Day 24}
\subtitle{Option Pricing}

\begin{frame}
  \titlepage
\end{frame}

\section{Set-up}\subsection{}
\begin{frame}{Assumptions}
    single risky asset and single Brownian motion, constant risk-free rate
    
    no dividends
    
    geometric Browinan motion price process
    $$\frac{\D S}{S} = \mu\,\D t + \sigma\,\D B$$
    for constants $\mu$ and $\sigma$
    
    option that expires at some future date $T$
    
    European = can only be exercised at $T$ or American = can be exercised at any time prior to $T$
\end{frame}

\begin{frame}{Asset Price}
  $$\frac{\D S}{S} = \mu\,\D t + \sigma\,\D B$$
  \pause
  
  $$\D \log S = \left(\mu - \frac{1}{2}\sigma^2\right)\,\D t + \sigma\,\D B$$
  \pause
  
  $$\log S_T = \log S_0 + \left(\mu - \frac{1}{2}\sigma^2\right)T + \sigma B_T$$
  
  $$S_T = S_0 \E^{(\mu-\sigma^2/2)T + \sigma B_T}$$
  
  
\end{frame}
\begin{frame}{SDF Process}
$$\frac{\D M}{M} = - r\,\D t - \lambda\,\D B$$

$$\sigma\lambda = \mu-r \quad \Rightarrow \quad \lambda = \frac{\mu-r}{\sigma}$$

\pause
$$\D \log M = -r\,\D t - \frac{1}{2}\lambda^2\,\D t - \lambda\,\D B$$

\pause
$$\log M_T = \log M_0 -\left(r + \frac{1}{2}\lambda^2\right)T - \lambda B_T$$
\pause
$$M_T = \E^{-(r+\lambda^2/2)T - \lambda B_T}$$
\end{frame}

\section{Black-Scholes}\subsection{}
\begin{frame}{European Call Option}
Value of call at maturity is $\max(0,S_T-K)=(S_T-K)^+$

Let $A$ denote the event $S_T>K$ and let $1_A$ denote its zero-one indicator.  

The value of the call at maturity is
$$(S_T-K)1_A = S_T1_A - K1_A$$

The value of the call at date 0 is
$$\mye[M_TS_T1_A] - K\mye[M_T1_A]$$
\end{frame}

\begin{frame}[plain]
The event $A$ is the event $\log S_T > \log K$, which is the event
$$\log S_0 + \left(\mu - \frac{1}{2}\sigma^2\right)T + \sigma B_T > \log K$$
Let $\tz$ denote the standard normal $-B_T/\sqrt{T}$.  The event $A$ is the event
$$\log S_0 + \left(\mu - \frac{1}{2}\sigma^2\right)T - \sigma \sqrt{T}\tz > \log K$$
This is equivalent to
$$\tz < \bar{z} \eqdef \frac{\log S_0 - \log K + (\mu-\sigma^2/2)T}{\sigma\sqrt{T}}$$
Also,
$$S_T = S_0 \E^{(\mu-\sigma^2/2)T - \sigma\sqrt{T}\tz} \quad \text{and} \quad
M_T =  \E^{-(r+\lambda^2/2)T - \lambda \sqrt{T}\tz}$$
So, we're integrating exponentials of a standard normal over the region $(-\infty, \bar{z})$.


    
\end{frame}


\begin{frame}{Black-Scholes Formula}
Value of the call at date 0 is
$$S_0\N(d_1) - \E^{-rT}K\N(d_2)$$
where $\N$ is the standard normal cdf and
\begin{align*}
    d_1 &= \frac{\log S_0 - \log K + (r+\sigma^2/2)T}{\sigma\sqrt{T}}\\
    d_2 &= d_1 - \sigma\sqrt{T}
\end{align*}
    
\end{frame}

\begin{frame}{Risk-Neutral Probability}

The asset price process is also
$$\frac{\D S}{S} = r\,\D t + \sigma\,\D B^*$$
where $B^*$ is a Brownian motion under the risk-neutral probability.

The value of the call at date 0 is
$$\E^{-rT}\mye^*[S_T1_A] - \E^{-rT}K\mye^*(1_A) = \E^{-rT}\mye^*[S_T1_A] - \E^{-rT}KQ(A)$$

\end{frame}

\section{Fundamental PDE}\subsection{}
\begin{frame}{Fundamental PDE}
    Let $f(t,S_t)$ denote the value at $t$ of some future payoff (e.g., the value of $(S_T-K)1_A$). 
    
    It is convenient to use the risk-neutral probability.  The RN expected rate of return is the risk-free rate:
    $$\frac{\text{drift of $f$}}{f(t,S_t)} = r$$
    \pause
   \begin{align*}
       \D f &= f_t\,\D t + f_S\,\D S 
       + \frac{1}{2}f_{SS}(\D S)^2\\
       &= f_t\,\D t + f_S(rS\,\D t + \sigma S\,\D B^*)
       + \frac{1}{2}f_{SS}(S^2\sigma^2\,\D t)
   \end{align*}
   
   \pause
   So 
   $$\text{drift of $f$} = f_t + rSf_S + \frac{1}{2}\sigma^2S^2f_{SS}$$
\end{frame}

\begin{frame}[plain]
The fundamental PDE is
\begin{equation}\tag{$\star$}
f_t + rSf_S + \frac{1}{2}\sigma^2S^2f_{SS} = rf
\end{equation}

\pause

The Black-Scholes formula is
$$f(t,S_t) = S_t\N(d_1(t,S_t)) - \E^{-r(T-t)}K\N(d_2(t,S_t))$$
where
\begin{align*}
    d_1(t,S_t) &= \frac{\log S_t - \log K + (r+\sigma^2/2)(T-t)}{\sigma\sqrt{T-t}}\\
    d_2(t,S_t) &= d_1(t,S_t) - \sigma\sqrt{T-t}
\end{align*}
We can differentiate and verify that the Black-Scholes formula satisfies the fundamental PDE.  It also satisfies the boundary conditions $f(t,0)=0$, $f(t,\infty)=\infty$, and 
$\lim_{t\rightarrow T}f(t,S) = (S-K)^+$.
\end{frame}

\begin{frame}{Americans and Perpetuals}
For a finite-lived American option there is an optimal exercise boundary $s^*(t)$.  Exercise at the first time that $S_t=s^*(t)$.  Value satisfies the fundamental PDE in the ``inaction region'' and conditions at the boundary.

Have to find the boundary.  Called a free boundary problem.  Have to solve numerically.

For a time-stationary perpetual option, the value does not depend on~$t$.  Need to find a function $f(S_t)$.  The fundamental PDE becomes an ODE.  There is an analytic solution.  Appplicable in many timing games -- innovation races, etc. -- and corporate investment problems --- called real options.

Solution is derived from following.  Pick a number $s^*$ and let $\tau$ denote the hitting time of $s^*$ (first time $S_t=s^*$).  Value getting \$1 at $\tau$.  Value is $f(S_t) = \mye^*[\E^{-r(\tau-t)}\mid S_t]$.  Solve the ODE.
\end{frame}
 

 
\end{document}