\documentclass[10pt]{beamer}

\usetheme[progressbar=foot]{metropolis}
\usepackage{appendixnumberbeamer}

\usepackage{booktabs}
\usepackage[scale=2]{ccicons}

\usepackage{pgfplots}
\usepgfplotslibrary{dateplot}
\pgfplotsset{compat=1.18} 

\usepackage{xspace}
\usepackage{xcolor}

\DeclareMathOperator{\stdev}{stdev}
\DeclareMathOperator{\var}{var}
\DeclareMathOperator{\cov}{cov}
\DeclareMathOperator{\corr}{corr}
\DeclareMathOperator{\prob}{prob}
\DeclareMathOperator{\n}{n}
\DeclareMathOperator{\N}{N}
\DeclareMathOperator{\Cov}{Cov}

\newcommand{\hlf}{\frac{1}{2}}
\newcommand{\bi}{\begin{itemize}}
\newcommand{\ei}{\end{itemize}}
\newcommand{\im}{\item}
\newcommand{\D}{\mathrm{d}}
\newcommand{\E}{\mathrm{e}}
\newcommand{\mye}{\ensuremath{\mathsf{E}}}
\newcommand{\myreal}{\ensuremath{\mathbb{R}}}
\newcommand{\bq}{\begin{equation}}
\newcommand{\eq}{\end{equation}}
\newcommand{\eqdef}{\;\buildrel \text{d{}ef}\over = \;}
\newcommand{\xstar}{\buildrel *\over X}
\newcommand{\pmax}{p^{\text{max}}}
\newcommand{\qmax}{q^{\text{max}}}
\newcommand{\bfr}{\begin{frame}}
\newcommand{\bfrp}{\begin{frame}[plain]}
\newcommand{\efr}{\end{frame}}
\newcommand{\F}{\mathcal{F}}
\newcommand{\FF}{\mathbb{F}}
\newcommand{\ve}{\varepsilon}
\newcommand{\lh}{\hat{\lambda}}
\definecolor{mycolor}{gray}{0.8}
\definecolor{mymaincolor}{rgb}{0.6862745098039216,0.9333333333333333,0.9333333333333333}
\newcommand{\alr}[1]{\textcolor{blue}{#1}}
\definecolor{LightCyan}{rgb}{0.88,1,1}
\newcommand{\yel}{\cellcolor{yellow}}
\newcommand{\blue}{\cellcolor{SkyBlue}}
\newcommand{\gr}{\cellcolor{SpringGreen}}
\newcommand{\pink}{\cellcolor{pink}}
\newcommand{\apr}{\cellcolor{Apricot}}
\newcommand{\tve}{\tilde{\varepsilon}}
\newcommand{\tw}{\tilde{w}}
\newcommand{\ttth}{\tilde{\theta}}
\newcommand{\te}{\tilde{e}}
\newcommand{\ts}{\tilde{s}}
\newcommand{\tx}{\tilde{x}}
\newcommand{\ty}{\tilde{y}}
\newcommand{\tv}{\tilde{v}}
\newcommand{\tp}{\tilde{p}}
\newcommand{\tF}{\tilde{F}}
\newcommand{\tf}{\tilde{f}}
\newcommand{\tZ}{\tilde{Z}}
\newcommand{\ow}{\overline{w}}
\newcommand{\tm}{\tilde{m}}
\newcommand{\tc}{\tilde{c}}
\newcommand{\tz}{\tilde{z}}
\newcommand{\tr}{\widetilde{R}}
\newcommand{\tR}{\widetilde{\mathbf{R}}}
\newcommand{\bms}{\begin{multline*}}
\newcommand{\ems}{\end{multline*}}
\newcommand{\bas}{\begin{align*}}
\newcommand{\eas}{\end{align*}}
\newcommand{\qr}{\mathbb{Q}}
\newcommand{\tX}{\tilde{X}}
\newcommand{\tY}{\tilde{Y}}

\setbeamertemplate{frame footer}{BUSI 521/ECON 505, Spring 2024}

\title{Chapter 19: Perpetual Options}

\date{}
\author{Kerry Back\\ 
BUSI 521/ECON 505\\
Rice University}


\begin{document}

\maketitle


\begin{frame}{Set-up}
    \bi 
      \im Single risky asset with price $S$ and constant volatility $\sigma$, single Brownian motion, constant risk-free rate
    
      \im Dividend paid by risky asset in time period $\D t$ is $qS_t\,\D t$ for constant $q$ (``dividend yield'')
    
      \im Total return is
    $$\frac{\D S + qS\,\D t}{S} = \frac{\D S}{S} + q\,\D t$$
    \im Total expected return under RNP is risk-free rate, so
    $$\frac{\D S}{S} = (r-q)\,\D t + \sigma\,\D B^*$$
    for a risk-neutral Brownian motion $B^*$.
    \ei
 \end{frame}
 

 \begin{frame}{Perpetual Call} 
    \bi \im
 Perpetual call option with strike $K$
 
 \im Why exercise?  To capture the dividend.  But the asset price and dividend must be high enough before it is optimal to do so.
 
 \im An example of a strategy is to pick a number $x$ and exercise the first time $S_t$ gets up to $x$.  The optimal strategy will be of this type.
 
 \im The problem of finding the optimal exercise time is in the class of problems often called optimal stopping.
 \ei 
 \end{frame}
 
 \begin{frame}{Exercise Boundary}
    \bi \im
 We will first calculate the value if we exercise the first time $S_t$ gets up to $x$ for an arbitrary $x>S_0$.
 
 \im  Let $\tau = \inf\{t \mid S_t \ge x\}$.  This is called the hitting time of $x$.
 
 \im By the time-homogeneity of $S$, the value at any $t<\tau$ depends only on $S_t$.  Call it $f(S_t)$.  
     
 \im    More formally,
     $$f(s) = \mye^*[\E^{-r\tau}((x-K) \mid S_0=s] = \mye^*[\E^{-r(\tau-t)} (x-K)\mid S_t=s]$$
     \ei 
 \end{frame}
 \begin{frame}{Fundamental ODE}
     \bi \im 
     The fundamental ODE is 
     $$\frac{\text{drift$^*$ of $f$}}{f} = r$$
     which is
     $$(r-q)Sf' + \frac{1}{2}\sigma^2S^2f'' = rf$$
     \im Trying a power solution $f(S) = S^\gamma$, we see that $f$ satisfies the ODE if and only if
     $$(r-q)\gamma + \frac{1}{2}\sigma^2\gamma(\gamma-1) = r$$
     \im The quadratic formula shows that there are two real roots of this equation.  One is negative and the other is greater than 1.  
     \ei
     
 
 \end{frame}
     
 \begin{frame}{General Solution and Boundary Conditions}
 \bi \im 
 Let $\gamma=\,$ absolute value of negative root, and $\beta=\,$ positive root.  
 \im The general solution of the ODE is
 $$aS^{-\gamma} + bS^\beta$$
 for constants $a$ and $b$ that must be determined by boundary conditions.
 
 \im The value $f$ of the call exercised at the hitting time of $x$ satisfies $f(0)=0$ and $f(x) = x-K$.  The condition $f(0)=0$ implies $a=0$, and the condition $f(x)=x-K$ implies $b=(x-K)x^{-\beta}$.  
 
 \im The value of the call is
 $$f(S_t) = (x-K)\left(\frac{S_t}{x}\right)^\beta$$
  \ei    
 \end{frame}
 
 \begin{frame}{Optimal Stopping}
 \bi \im 
 To optimize, maximize $(x-K)\left(\frac{S_t}{x}\right)^\beta$ over $x$. 
 \im The factor $S_t^\beta$ is a positive constant and is irrelevant for determining the optimum, so we can maximize
 $$(x-K)x^{-\beta} = x^{1-\beta} - Kx^{-\beta}$$
 
 \im The FOC is
 $$(1-\beta)x^{-\beta} + \beta Kx^{-\beta-1}=0$$
 \im Equivalently,
 $$(1-\beta)x + \beta K=0$$
 So, 
 $$x^* = \frac{\beta}{\beta-1}K$$
     \ei 
 \end{frame}
 
 \begin{frame}{Perpetual Put}
 \bi \im
 Recall that the general solution of the ODe is $f(s) = as^{-\gamma} + bs^\beta$.
 
 \im For a put, we exercise the first time $S_t$ drops to a boundary $x$.  
 \im The boundary conditions for a put are $f(\infty)=0$, and $f(x)=K-x$.  The condition $f(\infty)=0$ implies $b=0$.  The condition $f(x)=K-x$ implies $a=(K-x)x^{\gamma}$.
 
 \im So, the put value is 
 $$f(S_t) = (K-x)\left(\frac{x}{S_t}\right)^\gamma$$
 \im The FOC for maximizing over $x$ is
 $$\gamma K x^{\gamma-1} - (1+\gamma)x^\gamma =0$$
 Maximizing over $x$ yields $x^*=\gamma K/(1+\gamma)$.
 \ei
     
 \end{frame}
 \end{document}

