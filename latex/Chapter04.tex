\documentclass[10pt]{beamer}

\usetheme[progressbar=foot]{metropolis}
\usepackage{appendixnumberbeamer}

\usepackage{booktabs}
\usepackage[scale=2]{ccicons}

\usepackage{pgfplots}
\usepgfplotslibrary{dateplot}
\pgfplotsset{compat=1.18} 

\usepackage{xspace}
\usepackage{xcolor}

\DeclareMathOperator{\stdev}{stdev}
\DeclareMathOperator{\var}{var}
\DeclareMathOperator{\cov}{cov}
\DeclareMathOperator{\corr}{corr}
\DeclareMathOperator{\prob}{prob}
\DeclareMathOperator{\n}{n}
\DeclareMathOperator{\N}{N}
\DeclareMathOperator{\Cov}{Cov}

\newcommand{\hlf}{\frac{1}{2}}
\newcommand{\bi}{\begin{itemize}}
\newcommand{\ei}{\end{itemize}}
\newcommand{\im}{\item}
\newcommand{\D}{\mathrm{d}}
\newcommand{\E}{\mathrm{e}}
\newcommand{\mye}{\ensuremath{\mathsf{E}}}
\newcommand{\myreal}{\ensuremath{\mathbb{R}}}
\newcommand{\bq}{\begin{equation}}
\newcommand{\eq}{\end{equation}}
\newcommand{\eqdef}{\;\buildrel \text{d{}ef}\over = \;}
\newcommand{\xstar}{\buildrel *\over X}
\newcommand{\pmax}{p^{\text{max}}}
\newcommand{\qmax}{q^{\text{max}}}
\newcommand{\bfr}{\begin{frame}}
\newcommand{\bfrp}{\begin{frame}[plain]}
\newcommand{\efr}{\end{frame}}
\newcommand{\F}{\mathcal{F}}
\newcommand{\FF}{\mathbb{F}}
\newcommand{\ve}{\varepsilon}
\newcommand{\lh}{\hat{\lambda}}
\definecolor{mycolor}{gray}{0.8}
\definecolor{mymaincolor}{rgb}{0.6862745098039216,0.9333333333333333,0.9333333333333333}
\newcommand{\alr}[1]{\textcolor{blue}{#1}}
\definecolor{LightCyan}{rgb}{0.88,1,1}
\newcommand{\yel}{\cellcolor{yellow}}
\newcommand{\blue}{\cellcolor{SkyBlue}}
\newcommand{\gr}{\cellcolor{SpringGreen}}
\newcommand{\pink}{\cellcolor{pink}}
\newcommand{\apr}{\cellcolor{Apricot}}
\newcommand{\tve}{\tilde{\varepsilon}}
\newcommand{\tw}{\tilde{w}}
\newcommand{\ttth}{\tilde{\theta}}
\newcommand{\te}{\tilde{e}}
\newcommand{\ts}{\tilde{s}}
\newcommand{\tx}{\tilde{x}}
\newcommand{\ty}{\tilde{y}}
\newcommand{\tv}{\tilde{v}}
\newcommand{\tp}{\tilde{p}}
\newcommand{\tF}{\tilde{F}}
\newcommand{\tf}{\tilde{f}}
\newcommand{\tZ}{\tilde{Z}}
\newcommand{\ow}{\overline{w}}
\newcommand{\tm}{\tilde{m}}
\newcommand{\tc}{\tilde{c}}
\newcommand{\tz}{\tilde{z}}
\newcommand{\tr}{\widetilde{R}}
\newcommand{\tR}{\widetilde{\mathbf{R}}}
\newcommand{\bms}{\begin{multline*}}
\newcommand{\ems}{\end{multline*}}
\newcommand{\bas}{\begin{align*}}
\newcommand{\eas}{\end{align*}}
\newcommand{\qr}{\mathbb{Q}}
\newcommand{\tX}{\tilde{X}}
\newcommand{\tY}{\tilde{Y}}

\setbeamertemplate{frame footer}{BUSI 521/ECON 505, Spring 2024}

\title{Chapter 4: Equilibrium and Efficiency}

\date{}
\author{Kerry Back\\ 
BUSI 521/ECON 505\\
Spring 2024\\
Rice University}


\begin{document}

%%%%%%%%%%%%%%%%%%%%%%%%%%%%%%%%%%%%%%%%%%%%%%%%%%%%%%%%%%%%%%%%%%%%%%%

\begin{frame}[plain]
    \titlepage
  \end{frame}
  
  %%%%%%%%%%%%%%%%%%%%%%%%%%%%%%%%%%%%%%%%%%%%%%%%%%%%%%%%%%%%%%%%%%%%%%%
  
  \section{Equilibrium}\subsection{}
  \bfr\frametitle{Competitive Equilibrium}
  \bi 
  \im $n$ assets, possibly including a risk-free asset, with values $\tx_i$.  
  \im Investors $h=1, \ldots, H$ have endowments of shares $\bar{\theta}_h \in \myreal^n$ at date 0.
  \im Choose portfolios $\theta_h \in \myreal^n$ subject to budget constraint
  $$p'\theta_h \ \le \ p'\bar{\theta}_h$$
  \im Maximize expected utility of date--1 wealth $\tw_h :=\sum_{i=1}^n \theta_{hi}\tx_i$.   
  \im Equilibrium is $(p,\theta^*_1, \ldots, \theta^*_H)$ such that $\theta^*_h$ is optimal for each investor $h$ given $p$, and markets clear:
  $$\sum_{h=1}^H \theta^*_h \ = \ \sum_{h=1}^H \bar{\theta}_h\,.$$
  \im  Existence?
  Optimality?  
  Equilibrium risk premia?
  \ei
  \end{frame}
  
  \bfr\frametitle{Arrow-Debreu Model}
  \bi 
  \im 
  $k$ states.  
  \im Assets are Arrow securities: pay 1 in single state and 0 otherwise.  
  \im Denote price vector by $q \in \myreal^k$.     
  \im Portfolio $\theta$ determines date--1 wealth as $w_j = \theta_j$ for $j=1, \ldots, k$.  In other words, the wealth vector $w \in \myreal^k$ is the portfolio.
  \im Existence: standard result
  \im Optimality: standard welfare theorems
  \ei 
  \end{frame}
  
  \begin{frame}
    \bi 
    \im 
  When is a competitive equilibrium in a securities market equivalent to an equilibrium in an Arrow-Debreu model?
  \im   Answer: if the securities market is complete.
  \im Asset prices $p =(p_1, \ldots, p_n)$ and state prices $(q_1, \ldots, q_k)$ correspond as $Xq=p$.
  \im  So, equilibria in complete markets are Pareto optimal.
  \ei
  \end{frame}
  
  \section{Efficiency}\subsection{}
  
  \bfr\frametitle{Pareto Optimum}
  \bi 
   \im Let $\tw_m$ denote end-of-period market wealth.
   \im   An allocation is $\tw_1, \ldots, \tw_H$ such that $\sum_h \tw_h = \tw_m$.
  \im   A Pareto optimum is an allocation such that any other allocation that makes at least one person better off also makes at least one person worse off.
  \im 
  A Pareto optimum solves a social planner's problem: for some weights $\lambda_1, \ldots, \lambda_H$, 
  $$\max \quad \sum_{h=1}^H \lambda_h\mye [u_h(\tw_h)] \quad \text{subject to} \quad \sum_{h=1}^H \tw_h = \tw_m\,.$$
  \ei 
  \end{frame}
  
  \bfr\frametitle{Social Planner's Problem}
  \bi 
  \im The resource constraint is a separate constraint for each state.  And, expected utility is additive across states.
  \im So, we can solve the maximization problem state-by-state.  
  \im   What does this mean?  Consider the problem $\max a + b$ subject to $a\le 3$ and $b\le 5$.  We can solve this as separate problems: $\max a$ s.t.\ $a \le 3$ and $\max b$ s.t.\ $b\le 5$.
  \im 
  In each state of the world $\omega$, the social planner solves
  $$\max \quad \sum_{h=1}^H \lambda_hu_h(\tw_h(\omega)) \quad \text{subject to} \quad \sum_{h=1}^H \tw_h(\omega) = \tw_m(\omega)\,.$$
  \im 
  Let $\tilde{\eta}(\omega)$ denote the Lagrange multiplier in state $\omega$.  
  \im Then, for all $h$,
  $$\lambda_h u_h'(\tw_h(\omega)) = \tilde{\eta}(\omega)\,.$$
  \ei
  \end{frame}
  
  \bfr\frametitle{Sharing Rules}
  \bi 
  \im At a Pareto optimum, there is equality of MRS's: for all individuals $h$ and $\ell$ and states $i$ and $j$,
  $$\frac{u_h'(\tw_h(\omega_i))}{u_h'(\tw_h(\omega_j))} = \frac{u_\ell'(\tw_\ell(\omega_i))}{u_\ell'(\tw_\ell(\omega_j))}$$
   \im If market wealth is higher in state~$i$ than in state~$j$, then at any Pareto optimum (assuming strict risk aversion) all investors have higher wealth in state~$i$ than in state~$j$: 
  \begin{align*}
  \widetilde{w}_h(\omega_i)>\widetilde{w}_h(\omega_j) & \Rightarrow \frac{u_h'(\widetilde{w}_h(\omega_i))}{u_h'(\widetilde{w}_h(\omega_j))} < 1 \\
  & \Rightarrow \frac{u_\ell'(\widetilde{w}_\ell(\omega_i))}{u_\ell'(\widetilde{w}_\ell(\omega_j))} <~1 \\
  &\Rightarrow \widetilde{w}_\ell(\omega_i)>\widetilde{w}_\ell(\omega_j)\,.
  \end{align*}
   \im This implies each investor's wealth is a function of market wealth.  The function is called a sharing rule.
  \ei 
  \end{frame}
  
  \bfr\frametitle{Example}
  \bi 
  \im 
  Suppose there are two risk-averse investors and two possible states of the world, with  $\widetilde{w}_m$ being the same in both states, say, $\widetilde{w}_m=6$, and with the two states being equally likely.  
  \im Can the allocation
  $$\widetilde{w}_1 = \begin{cases} 2 & \text{in state~1}\\
  4 & \text{in state~2}
  \end{cases}$$
  $$\widetilde{w}_2 = \begin{cases} 4 & \text{in state~1}\\
  2 & \text{in state~2}
  \end{cases}$$
  be Pareto optimal?
  \ei
  \end{frame}
  
  \section{LRT Utility}\subsection{}
  
  \bfr\frametitle{Sharing Rules with Linear Risk Tolerance}
  \bi 
  \im Assume $\tau_h(w) = A_h + B w$ with same cautiousness parameter $B\ge 0$ for all individuals.  Note $B>0$ implies DARA.  Then, either
  \bi
  \im Everyone has CARA utility: $-\E^{-\alpha_h w}$, or
  \im Everyone has shifted log utility:  $\log (w- \zeta_h)$, or
  \im Everyone has shifted power utility with $\rho>0$: 
  $$\frac{1}{1-\rho} (w-\zeta_h)^{1-\rho}$$
  \ei
  \im 
  In this case, Pareto optimal sharing rules are affine: $\tw_h = a_h + b_h \tw_m$.
  \ei 
  \end{frame}
  
  \begin{frame}
    \bi 
  \im Pareto optimal sharing rules with LRT utility and same cautiousness parameter are affine ($\tw_h = a_h + b_h \tw_m$) with
  \bi
  \im $\sum_{h=1}^H a_h = 0$, and
  \im $\sum_{h=1}^H b_h = 1$.
  \ei
  
  \im With CARA utility, $b_h = \tau_h / \sum_{j=1}^H \tau_j$.  
  
  \im With shifted log ($\rho=1$) or shifted power,
  $$b_h =  \frac{\lambda_h^{1/\rho }}{\sum_{j=1}^H \lambda_j^{1/\rho }}$$
  where the $\lambda$'s are the weights in the social planning problem. 
  \ei
  \end{frame}
  
  \bfr\frametitle{Proof for CARA Utility}
  \bi 
  \im
  Social planner's problem is (in each state of the world)
  $$\max \quad \sum_{h=1}^H \lambda_h \E^{-\alpha_h w_h} \quad \text{subject to} \quad \sum_{h=1}^H w_h = w$$
  where $w$ denotes the value of $\tw_m$ in the given state.
  
  \im FOC is:  $(\forall\, h) \; \lambda_h\alpha_h\E^{-\alpha_hw_h} = \eta$ where $\eta$ is the Lagrange multiplier (in the given state).   Set $\tau = \sum_h \tau_h$.   We have
  \begin{align*}
  w_h& =  -{\tau _h}\log \eta + \tau _h\log (\lambda_h\alpha_h)\label{3_wh}\\
   \Rightarrow \quad 
   w & = - \tau \log \eta + \sum_{\ell =1}^H \tau _\ell \log (\lambda_\ell \alpha_\ell )\\
   \Rightarrow \quad -\log \eta & =
  \frac{1}{\tau }w - \frac{1}{\tau }\sum_{\ell =1}^H \tau _\ell \log (\lambda_\ell \alpha_\ell )\\
   \Rightarrow \quad 
  w_h& =\frac{\tau _h}{\tau }w -\frac{\tau _h}{\tau }\sum_{\ell =1}^H \tau _\ell \log (\lambda_\ell \alpha_\ell ) + \tau _h\log(\lambda_h\alpha_h)
  \end{align*}
  \ei 
  \end{frame}
  
  \bfr\frametitle{Competitive Equilibria with LRT Utility}
  \bi
  \im Assume there is a risk-free asset.  Assume all investors have linear risk tolerance $\tau_h(w) = A_h + Bw$ with the same cautiousness parameter $B$.  Assume there are no $\ty_h$'s.
  \im The set of equilibrium prices does not depend on the distribution of wealth across investors.
  \bi
  \im Called Gorman aggregation
  \im Due to wealth expansion paths being parallel (Chapter 2)
  \ei
  \im Any Pareto optimal allocation can be implemented in the securities market.
  \bi
  \im Due to affine sharing rules, we only need the risk-free asset and market portfolio
  \im Example of two-fund separation (Chapter 2)
  \ei
  \im Any competitive equilibrium is Pareto optimal.
  \ei
  \end{frame}
  
  \end{document}
  
  \bfr\frametitle{Linear Risk Tolerance}
  If each $u_h$ has linear risk tolerance with the same cautiousness parameter, then $u$ has linear risk tolerance with the same cautiousness parameter.
  \bi
  \im If each investor has CARA utility, then the representative investor has CARA utility.
  \im If each investor has CRRA utility with risk aversion $\rho$, then the representative investor has CRRA utility with risk aversion $\rho$.
  \ei
  \end{frame}
  
  
  
  \bfr\frametitle{Marginal Rates of Substitution}
  Let $c_0$ and $\tilde{c}_1$ denote aggregate consumption.  At a Pareto optimum, each investor's MRS 
  $$\frac{\delta u_h'(\tilde{c}_{h1})}{u_h'(c_{h0})}$$
  equals the social planner's MRS
  $$\frac{\delta u'(\tilde{c}_{1})}{u'(c_{0})}$$
  Proof: \small
  \bi
  \im At a Pareto optimum, the envelope theorem implies, for each $h$, 
  $$u'(c_0) = \lambda_h u_h'(c_{h0})\,, \qquad u'(\tc_1) = \lambda_h u_h'(\tc_{h1})$$
  \im Dividing cancels the $\lambda_h$.
  \ei
  \end{frame}
  
  
  
  
  \section{Social Planner and SDF}\subsection{}
  
  \bfr\frametitle{Social Planner's Utility Function}
  Assume investors have time-additive utility with the same discount factor $\delta$.  Let $u_h$ denote the utility function of investor $h$.
  Given weights $\lambda_h$, define
  $$u(c) = \max \; \left\{\sum_{h=1}^H \lambda_h u_h(c_h) \mid \sum_{h=1}^H c_h = c\right\}$$
  \bi
  \im This is the social planner's utility function.
  \im The social planner's utility function $u$ is concave if each of the $u_h$ is concave.
  \ei
  \end{frame}
  
  \bfr\frametitle{Linear Risk Tolerance}
  If each $u_h$ has linear risk tolerance with the same cautiousness parameter, then $u$ has linear risk tolerance with the same cautiousness parameter.
  \bi
  \im If each investor has CARA utility, then the representative investor has CARA utility.
  \im If each investor has CRRA utility with risk aversion $\rho$, then the representative investor has CRRA utility with risk aversion $\rho$.
  \ei
  \end{frame}
  
  \bfr\frametitle{Proof for CARA Utility}
  Set $\tau = \sum_h \tau_h$.  FOC for social planning problem ($\tilde{\eta}$ is Lagrange multiplier for resource constraint): $(\forall\, h) \; \lambda_h\alpha_h\E^{-\alpha_h\widetilde{w}_h} = \widetilde{\eta}$
  \begin{align*}
   \quad & \Rightarrow \quad 
  \widetilde{w}_h =  -{\tau _h}\log \widetilde{\eta} + \tau _h\log (\lambda_h\alpha_h)\label{3_wh}\\
  & \Rightarrow \quad 
  \widetilde{w}_m = - \tau \log \widetilde{\eta} + \sum_{\ell =1}^H \tau _\ell \log (\lambda_\ell \alpha_\ell )\\
  & \Rightarrow \quad -\log \widetilde{\eta} =
  \frac{1}{\tau }\widetilde{w}_m - \frac{1}{\tau }\sum_{\ell =1}^H \tau _\ell \log (\lambda_\ell \alpha_\ell )\\
  & \Rightarrow \quad 
  \widetilde{w}_h =\frac{\tau _h}{\tau }\widetilde{w}_m -\frac{\tau _h}{\tau }\sum_{\ell =1}^H \tau _\ell \log (\lambda_\ell \alpha_\ell ) + \tau _h\log(\lambda_h\alpha_h)
  \end{align*}
  \end{frame}
  
  \bfr\frametitle{Marginal Rates of Substitution}
  Let $c_0$ and $\tilde{c}_1$ denote aggregate consumption.  At a Pareto optimum, each investor's MRS 
  $$\frac{\delta u_h'(\tilde{c}_{h1})}{u_h'(c_{h0})}$$
  equals the social planner's MRS
  $$\frac{\delta u'(\tilde{c}_{1})}{u'(c_{0})}$$
  Proof: \small
  \bi
  \im At a Pareto optimum, the envelope theorem implies, for each $h$, 
  $$u'(c_0) = \lambda_h u_h'(c_{h0})\,, \qquad u'(\tc_1) = \lambda_h u_h'(\tc_{h1})$$
  \im Dividing cancels the $\lambda_h$.
  \ei
  \end{frame}
  
  
  \bfr\frametitle{Pareto Optimal Competitive Equilibrium}
  At a Pareto optimal competitive equilibrium, the social planner's MRS is an SDF.  Proof:
  \bi
  \im By Pareto optimality, the social planner's MRS equals each investor's MRS.
  \im At a competitive equilibrium (due to each investor optimizing) each investor's MRS is an SDF.
  \ei 
  \end{frame}
  \end{document}
  

\end{document}