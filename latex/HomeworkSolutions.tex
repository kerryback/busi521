1.2, 1.3, 2.2, and 2.6
\documentclass[11pt]{article}
\RequirePackage{natbib,amsmath,amsthm,array,graphicx,footmisc,amsfonts,geometry,fancyvrb,chngcntr,minitoc}
\raggedbottom
\raggedright
\setlength{\parindent}{5ex}

\newcommand{\hiddensection}[1]{
	\stepcounter{section}
	\section*{\arabic{chapter}.\arabic{section}\hspace{1em}{#1}}
}

	
\counterwithin{table}{chapter}

\newenvironment{mypetit}{\centerline{\rule[0.2\baselineskip]{1in}{0.15mm}}\noindent\small}{}
\newenvironment{mypetitenum}{\centerline{\rule[0.2\baselineskip]{1in}{0.15mm}}\noindent\small}{}

\def\next{\vskip \baselineskip\noindent}
\newcommand{\mybox}[1]{\next\fbox{\parbox{4.55in}{#1}}\next}
\newcommand{\listtype[1]}{\renewcommand{\labelenumi}{#1}}
\newcommand{\vc}{^{\text{vec}}}
\newcommand{\mv}{_{\text{m}}}
\newcommand{\zb}{_{\text{z}}}
\newcommand{\cm}{_{\text{c}}}
\newcommand{\vct}{^{\text{vec}}}
\newcommand{\wt}[1]{\widetilde{#1}}
\newcommand{\tm}{\ensuremath{ t\!-\!1} }
\newcommand{\Tm}{\ensuremath{ T\!-\!1} }
\newcommand{\pr}{\ensuremath{\mathbb{P}}}
\newcommand{\qr}{\ensuremath{\mathbb{Q}}}
\newcommand{\mye}{\ensuremath{\mathsf{E}}}
\newcommand{\sr}{\ensuremath{\mathbb{S}}}
\newcommand{\fr}{\ensuremath{\mathbb{F}}}
\newcommand{\price}{\ensuremath{\mathcal{P}}}
\newcommand{\myreal}{\ensuremath{\mathbb{R}}}
\newcommand{\excise}[1]{\vskip 0.5\baselineskip \textit{#1} \vskip 0.5\baselineskip}
\newcommand{\CE}{\xi}
\newcommand{\sectspace}{\;\;\;\;}
\newcommand{\D}{\mathrm{d}}
\newcommand{\E}{\mathrm{e}}
\newcommand{\eqdef}{\;\buildrel \text{d{}ef}\over =\;}
\newcommand{\eqquest}{\;\buildrel \text{?}\over =\;}
\newcommand{\halfskip}{\vskip 0.5\baselineskip\noindent}
\newcommand{\onevector}{\iota}
\newcommand{\Rvector}{\mathbf{R}}
\newcommand{\sol}{\textbf{Solution:} \hspace{2ex}}
\newcommand{\be}{\begin{enumerate}\renewcommand{\labelenumi}{(\alph{enumi})}}
\newcommand{\ee}{\end{enumerate}}
\newcommand{\bq}{\begin{equation}}
\newcommand{\eq}{\end{equation}}

\theoremstyle{definition}
\newtheorem{prob}{}[chapter]

\newcommand{\teff}{\tau_{\text{eff}}}

\DeclareMathOperator{\var}{var} \DeclareMathOperator{\stdev}{stdev}
\DeclareMathOperator{\cov}{cov} \DeclareMathOperator{\corr}{corr}
\DeclareMathOperator{\M}{M} \DeclareMathOperator{\N}{N}
\DeclareMathOperator{\nd}{n} \DeclareMathOperator{\Cov}{Cov}
\DeclareMathOperator{\Prob}{prob} \DeclareMathOperator{\Var}{Var}
\DeclareMathOperator{\argmax}{argmax}
\DeclareMathOperator{\proj}{proj}
\DeclareMathOperator{\sign}{sign}

\newcommand{\ptext}[1]{\Prob(\text{#1})}

\geometry{verbose,letterpaper,tmargin=1in,bmargin=1.25in,lmargin=1in,rmargin=1in,headheight=0.2in,footskip=0.5in}
\renewcommand{\baselinestretch}{2}
\setlength{\headsep}{2\baselineskip}
\setlength{\footnotesep}{\baselineskip}

\newcommand{\bi}{\begin{itemize}}
\newcommand{\ei}{\end{itemize}}
\newcommand{\im}{\item}

\newcommand{\notes}[1]{
\addtocontents{toc}{\setcounter{tocdepth}{0}}
\addtocontents{minitoc}{\setcounter{tocdepth}{0}}
\section{\sectspace Notes and References}\label{#1}
\addtocontents{toc}{\protect\setcounter{tocdepth}{2}}
\addtocontents{minitoc}{\protect\setcounter{tocdepth}{2}}
}


\setlength{\headsep}{2\baselineskip}

\begin{document}
\VerbatimFootnotes
\noindent \textbf{Assignment 1: Exercise 1.2}\\
\noindent Consider a person with constant relative risk aversion~$\rho$ who has wealth~$w$.
\begin{enumerate}\renewcommand{\labelenumi}{(\alph{enumi})}
\item Suppose he faces a gamble in which he wins or loses some amount~$x$ with equal probabilities.  Derive a formula for the amount~$\pi$ that he would pay to avoid the gamble; that is, find~$\pi$ satisfying
$$u(w-\pi) = \frac{1}{2}u(w-x) + \frac{1}{2}u(w+x)$$
when~$u$ is log or power utility.  

\sol When $u$ is log utility, we have
$$\log (w-\pi) = \frac{1}{2}\log (w-x) + \frac{1}{2} \log (w+x) = \log \sqrt{(w-x)(w+x)}\,.$$
Exponentiating both sides and rearranging gives
$$\pi = w - \sqrt{(w-x)(w+x)}\,.$$
For power utility, we have
$$\frac{1}{1-\rho}(w-\pi)^{1-\rho} = \frac{1}{2}\cdot \frac{1}{1-\rho}(w-x)^{1-\rho} + \frac{1}{2}\cdot \frac{1}{1-\rho}(w+x)^{1-\rho}\,.$$
This implies
$$\pi = w - \left[\frac{1}{2}(w-x)^{1-\rho} + \frac{1}{2}(w+x)^{1-\rho}\right]^{1/(1-\rho)}\,.$$

\item Suppose instead that he is offered a gamble in which he loses~$x$ or wins~$y$ with equal probabilities.  Find the maximum possible loss~$x$ at which he would accept the gamble; that is, find~$x$ satisfying
$$u(w) = \frac{1}{2}u(w-x) + \frac{1}{2}u(w+y)$$
when~$u$ is log or power utility.

\sol When $u$ is log utility, we have
$$\log w = \frac{1}{2}\log (w-x) + \frac{1}{2} \log (w+y) = \log \sqrt{(w-x)(w+y)}\,.$$
Exponentiating and then squaring both sides and rearranging gives
$$x = \frac{wy}{w+y}\,.$$
For power utility, we have
$$\frac{1}{1-\rho}w^{1-\rho} = \frac{1}{2}\cdot \frac{1}{1-\rho}(w-x)^{1-\rho} + \frac{1}{2}\cdot \frac{1}{1-\rho}(w+y)^{1-\rho}\,.$$
This implies
$$x = w - \left[2w^{1-\rho} - (w+y)^{1-\rho}\right]^{1/(1-\rho)}\,.$$

\item Suppose  the person has wealth of \$100,000 and faces a gamble as in Part (a).  Use the answer in Part (a) to calculate the amount he would pay to avoid the gamble, for various values of~$\rho$ (say, between $0.5$ and $40$), and for $x=\text{\$100}$, $x=\text{\$1,000}$, $x=\text{\$10,000}$, and $x=\text{\$25,000}$.  
For large gambles, do large values of~$\rho$ seem reasonable?  What about small gambles?\\
\noindent\sol Set $\varepsilon = x/100000$.  From the solution to the previous exercise, the amount the person would pay is $100000\pi$, where, for log utility,
$$\pi = 1-\E^{0.5\log(1+\varepsilon)+0.5\log(1-\varepsilon)}\,,$$
and, for power utility,
$$\pi = 1-\left[0.5(1+\varepsilon)^{1-\rho}+0.5(1-\varepsilon)^{1-\rho}\right]^{\frac{1}{1-\rho}}\,.$$
This implies the following:
\begin{center}
	\begin{tabular}{llrrrr}
		$\rho$ & \quad & \text{\quad\$100}& \text{\quad\$1,000} & \text{\quad\$10,000}& \text{\quad\$25,000}\\
		\hline\\
		0.5 & \quad & \text{\quad\$0.03} & \text{\quad\$2.50} & \text{\quad\$251} & \text{\quad\$1,588} \\
		1  & \quad& \text{\quad\$0.05} & \text{\quad\$5} & \text{\quad\$501} & \text{\quad\$3,175} \\
		2  & \quad& \text{\quad\$0.10} & \text{\quad\$10} & \text{\quad\$1,000} & \text{\quad\$6,250} \\
		5  & \quad& \text{\quad\$0.25} & \text{\quad\$25} & \text{\quad\$2,434} & \text{\quad\$13,486} \\
		10  & \quad& \text{\quad\$0.50} & \text{\quad\$50} & \text{\quad\$4,424} & \text{\quad\$19,086} \\
		15  & \quad& \text{\quad\$0.75} & \text{\quad\$75} & \text{\quad\$5,826} & \text{\quad\$21,198} \\
		20  & \quad& \text{\quad\$1.00} & \text{\quad\$99} & \text{\quad\$6,763} & \text{\quad\$22,214} \\
		30  & \quad& \text{\quad\$1.50} & \text{\quad\$148} & \text{\quad\$7,832} & \text{\quad\$23,186} \\
		40  & \quad& \text{\quad\$2.00} & \text{\quad\$195} & \text{\quad\$8,387} & \text{\quad\$23,655}
	\end{tabular}
\end{center}

\noindent For the largest gamble, $\rho > 5$ (or, perhaps $\rho>2$) would seem unreasonable.  But, for $\rho \leq 5$, the premium for the \$100 gamble is \$0.25 or less, which may be too small.
   \item Suppose $\rho>1$, and the person is offered a gamble as in Part (b).  Show that he will reject the gamble no matter how large~$y$ is if
        $$\frac{x}{w} \geq~1 - 0.5^{1/(\rho-1)} \quad \Leftrightarrow \quad \rho \geq \frac{\log (0.5) + \log (1-x/w)}{\log (1-x/w)}\,.$$
        For example, with wealth of \$100,000, the person would reject a gamble in which he loses \$10,000 or wins~1 trillion dollars with equal probabilities when~$\rho$ satisfies this inequality for $x/w=0.1$.
        What values of~$\rho$ (if any) seem reasonable?\\
        \noindent\sol Given $1-\rho<0$, the person rejects the gamble if
        $$w^{1-\rho} < 0.5(w-x)^{1-\rho} + 0.5(w+y)^{1-\rho}\,.$$
        This is true for all $y>0$ if
        \begin{align*}
        w^{1-\rho} \leq 0.5(w-x)^{1-\rho} & \quad\Leftrightarrow\quad  w \geq 0.5^{\frac{1}{1-\rho}}(w-x)\\
        & \quad\Leftrightarrow\quad 0.5^{\frac{1}{1-\rho}}x \geq \left[0.5^{\frac{1}{1-\rho}}-1\right]w \\
        & \quad\Leftrightarrow\quad \frac{x}{w} \geq 1- 0.5^{\frac{1}{\rho-1}}\\
        & \quad\Leftrightarrow\quad 0.5^{\frac{1}{\rho-1}} \geq 1-\frac{x}{w} \\
        & \quad\Leftrightarrow\quad \frac{1}{\rho-1} \log (0.5) \geq \log (1-x/w)\\
        & \quad\Leftrightarrow\quad \frac{1}{\rho-1}  \leq \frac{\log (1-x/w)}{\log (0.5)}\\
        & \quad\Leftrightarrow\quad \rho - 1 \geq \frac{\log (0.5)}{\log (1-x/w)}\\
        & \quad\Leftrightarrow\quad \rho \geq \frac{\log (0.5)+\log (1-x/w)}{\log (1-x/w)}
        \end{align*}
        Thus, all gambles involving 1\% losses are rejected if $\rho\geq7 0$,  2\% losses  if $\rho\geq 36$, 10\% losses if $\rho\geq 7.6$, 25\% losses if $\rho\geq 3.5$, and 50\% losses if $\rho\geq 2$.  Surely, there should be some possible gain that would compensate someone for a 50\% chance of a 10\% loss, implying $\rho < 7.6$.  One could obviously argue for even smaller $\rho$.
        \end{enumerate}


%%%%%%%%%%%%%%%%%%%%%%%%%%%%%%%%%%%%%%%%%%%%%%%%%%%%%
%
%  1.3
%
\newpage\noindent \textbf{Assignment 1: Exercise 1.3}\\
\noindent 
  This exercise is a very simple version of a model of the bid-ask spread presented by \citet{Stoll_JF_1978}.
Consider an individual with constant absolute risk aversion~$\alpha$.  Assume~$\tilde{w}$ and~$\tilde{x}$ are joint normally distributed with means $\mu_w$ and $\mu_x$, variances $\sigma_w^2$ and $\sigma_x^2$ and correlation coefficient~$\rho$.
\begin{enumerate}\renewcommand{\labelenumi}{(\alph{enumi})}
\item Compute the maximum amount the individual would pay to obtain~$\tilde{w}$ when starting with~$\tilde{x}$; that is, compute $\text{BID}$ satisfying
$$\mye[u(\tilde{x})] = \mye[u(\tilde{x}+\tilde{w}-\text{BID})]\,.$$
\sol \noindent  Note that
$$\mye[u(\tilde{w})] = - \exp\left( - \alpha \mye[\tilde{w}] + \frac{1}{2}\alpha^2 \var(\tilde{w})\right)\,.$$
We have
$$\mye[u(\tilde{w}+\tilde{x}-\text{BID})] = - \exp\left( - \alpha \mye[\tilde{w}] -\alpha \mye[\tilde{x}] + \alpha \text{BID} + \frac{1}{2}\alpha^2 [\var(\tilde{w})+2\cov(\tilde{x},\tilde{w}) + \var(\tilde{x})]\right)\,.$$
Thus, BID satisfies
$$1 = \exp\left(-\alpha \mye[\tilde{x}] + \alpha \text{BID} + \frac{1}{2}\alpha^2 [2\cov(\tilde{x},\tilde{w}) + \var(\tilde{x})]\right)\,.$$
This implies
$$\text{BID} = \mye[\tilde{x}] - \alpha\cov(\tilde{x},\tilde{w}) - \frac{1}{2}\alpha \var(\tilde{x})=\mu_x-\alpha \rho\sigma_x\sigma_w-\frac{1}{2}\alpha\sigma_x^2\,.$$
\item Compute the minimum amount the individual would require to accept the payoff $-\tilde{w}$ when starting with~$\tilde{x}$; that is, compute $\text{ASK}$ satisfying
$$\mye[u(\tilde{x})] = \mye[u(\tilde{x}-\tilde{w}+\text{ASK})]\,.$$
\sol  We have
$$\mye[u(\tilde{w}-\tilde{x}+\text{ASK})] = - \exp\left( - \alpha \mye[\tilde{w}] +\alpha \mye[\tilde{x}] - \alpha \text{ASK} + \frac{1}{2}\alpha^2 [\var(\tilde{w})-2\cov(\tilde{x},\tilde{w}) + \var(\tilde{x})]\right)\,.$$
Thus, ASK satisfies
$$1 = \exp\left(\alpha \mye[\tilde{x}] - \alpha \text{ASK} + \frac{1}{2}\alpha^2 [-2\cov(\tilde{x},\tilde{w}) + \var(\tilde{x})]\right)\,.$$
This implies
$$\text{ASK} = \mye[\tilde{x}] - \alpha\cov(\tilde{x},\tilde{w}) + \frac{1}{2}\alpha \var(\tilde{x})=\mu_x-\alpha \rho\sigma_x\sigma_w+\frac{1}{2}\alpha\sigma_x^2\,.$$

%%%%%%%%%%%%%%%%%%%%%%%%%%%%%%%%%%%%%%%%%%%%%%%%%%%%%
%
%  2.2
%
\newpage \noindent \textbf{Assignment 1: Exercise 2.2}\\
\noindent 
	Suppose there is a risk-free asset and $n$ risky assets with payoffs $\tilde{x}_i$ and prices $p_i$.  Assume the vector $\tilde{x}=(\tilde{x}_1 \cdots \tilde{x}_n)'$ is normally distributed with mean $\mu_x$ and nonsingular covariance matrix $\Sigma_x$.  Let $p=(p_1 \cdots p_n)'$.  Suppose there is consumption at date 0 and consider an investor with initial wealth~$w_0$ and CARA utility at date~1:
	$$u_1(c) = - \E^{-\alpha c}\,.$$
	Let $\theta_i$ denote the number of shares the investor considers holding of asset~$i$ and set $\theta = (\theta_1 \cdots \theta_n)'$.
	The investor chooses consumption $c_0$ at date 0 and a portfolio $\theta$, producing wealth $(w_0-c_0 - \theta'p)R_f + \theta'\tilde{x}$ at date~1.
	\begin{enumerate}\renewcommand{\labelenumi}{(\alph{enumi})}
		\item Show that the optimal vector of share holdings is
		$$\theta = \frac{1}{\alpha }\Sigma_x^{-1}(\mu_x-R_fp)\,.$$\\
		\noindent\sol The investor chooses date--0 consumption $c_0$ and a portfolio $\theta$ of risky assets to maximize
		$$u(c_0) - \exp\left(-\alpha(w_0-c_0-\theta'p)R_f - \alpha \theta'\mu_x + \frac{1}{2}\alpha^2\theta'\Sigma_x\theta\right)\,.$$
		The optimal portfolio $\theta$ is the portfolio that maximizes
		$$ - R_fp'\theta +  \mu_x'\theta  - \frac{1}{2}\alpha\theta'\Sigma_x\theta\,.$$
		The first-order condition is
		$$-R_fp + \mu_x - \alpha\Sigma_x\theta = 0\,,$$
		with solution
		$$\theta = \frac{1}{\alpha}\Sigma_x^{-1}(\mu_x-R_fp)\,.$$
		\item Suppose all of the asset prices are positive, so we can define returns $\tilde{x}_i/p_i$. Explain why (2.33) (need a reference link) implies (2.22) (need a reference link). Note: This is another illustration of the absence of wealth effects. Neither date-0 wealth nor date-0 consumption affects the optimal portfolio for a CARA investor.\\
		\noindent\sol By definition $\tilde{R}_i=\tilde{x}_i/p_i$. Let $\mu_i=\mu_{x_i}/p_i$ be the mean of $\tilde{R}_i$ and $\Sigma_i=\Sigma_{x_i}/p_i^2$ be the covariance of it.\\
		Substituting in the equation of $\theta$ from part a we get:\\
		$$\phi_i=p_i\theta_i=\frac{1}{\alpha}\Sigma^{-1}_i(\mu-R_f\iota)$$\\
		Thus\\
		$$\phi=\frac{1}{\alpha}\Sigma^{-1}(\mu-R_f\iota)\,.$$
	
	
	%%%%%%%%%%%%%%%%%%%%%%%%%%%%%%%%%%%%%%%%%%%%%%%%%%%%%
%
%  2.6
%
\newpage
\noindent \textbf{Assignment 1: Exercise 2.6}\\
\noindent 
	Consider a utility function $v(c_0,c_1)$.
	The marginal rate of substitution (MRS) is defined to be the negative of the slope of an indifference curve and is equal to
	$$\text{MRS}(c_0,c_1) = \frac{\partial v(c_0,c_1)/\partial c_0}{\partial v(c_0,c_1)/\partial c_1}\,.$$
	The elasticity of intertemporal substitution is defined as
	$$\frac{\D \log (c_1/c_0)}{\D \log \text{MRS}(c_0,c_1)}\,,$$
	where the marginal rate of substitution is varied holding utility constant.
	Show that, if
	$$v(c_0,c_1) = \frac{1}{1-\rho}c_0^{1-\rho} + \frac{\delta}{1-\rho}c_1^{1-\rho}\,,$$
	then the EIS is $1/\rho$.
\end{prob}

\noindent\sol Holding utility constant implies
$$c_0^{-\rho}\,\D c_0 + \delta c_1^{-\rho},\D c_1 = 0\,,$$
so
$$- \frac{\D c_1}{\D c_0} = \frac{1}{\delta}\left(\frac{c_0}{c_1}\right)^{-\rho}\,.$$
This is the marginal rate of substitution.  Setting $x=c_1/c_0$, we have
$$\log \text{MRS} = - \log \delta + \rho \log x\,.$$
Hence,
$$\frac{\D \log \text{MRS}}{\D \log x} = \rho\,.$$
The elasticity of intertemporal substitution is the reciprocal $1/\rho$.	  
	\end{enumerate} 
	 


%%%%%%%%%%%%%%%%%%%%%%%%%%%%%%%%%%%%%%%%%%%%%%%%%%%%%
%
%  3.1
%
\noindent \textbf{Assignment 2: Exercise 3.1}\\
\noindent 
Assume there are two possible states of the world: $\omega_1$ and $\omega_2$.  There are two assets, a risk-free asset returning $R_f$ in each state, and a risky asset with initial price equal to~1 and date--1 payoff~$\tilde{x}$.  Let $R_d=\tilde{x}(\omega_1)$ and $R_u=\tilde{x}(\omega_2)$.  Assume without loss of generality that $R_u > R_d$.
\begin{enumerate}\renewcommand{\labelenumi}{(\alph{enumi})}
\item What inequalities between $R_f$, $R_d$ and $R_u$ are equivalent to the absence of arbitrage opportunities?\\
\noindent\sol The payoff of a zero-cost portfolio is $\phi(\tilde{R}-R_f)$ for some $\phi$.  For this to be nonnegative in both states and positive in one state, we must have either (i) $\phi>0$ and $R_u>R_d \geq R_f$ or (ii) $\phi<0$ and $R_f \geq R_u > R_d$.  Thus, a necessary and sufficient condition for the absence of arbitrage opportunities is that $R_u>R_f>R_d$.
\item Assuming there are no arbitrage opportunities, compute the unique vector of state prices,
and compute the unique risk-neutral probabilities of states $\omega_1$ and $\omega_2$.\\
\noindent\sol Let $q_d$ denote the state price of state $\omega_1$ and $q_u$ the state price of state $\omega_2$.  The state prices satisfy
\begin{align*}
q_dR_f + q_u R_f &=1\,,\\
q_d R_d + q_uR_u &=1\,.
\end{align*}
The unique solution to this system of equations is
$$q_d = \frac{R_u-R_f}{R_f(R_u-R_d)}\,,\quad \text{and}\quad q_u = \frac{R_f-R_d}{R_f(R_u-R_d)}\,.$$
The risk neutral probabilities are  $q_dR_f$ and $q_uR_f$.
\item Suppose another asset is introduced into the market that pays $\max(\tilde{x}-K,0)$ for some constant~$K$.  Compute the price at which this asset should trade, assuming there are no arbitrage opportunities.\\
\noindent\sol The asset should trade at
$q_u \max(x_u-K,0) + q_d\max(x_d-K,0)$, where $x_d$ denotes the value of $\tilde{x}$ in state~$1$ and $x_u$ the value of $\tilde{x}$ in state~$2$.
\end{enumerate}


%%%%%%%%%%%%%%%%%%%%%%%%%%%%%%%%%%%%%%%%%%%%%%%%%%%%%
%
%  3.2
%
\newpage\noindent \textbf{Assignment 2: Exercise 3.2}\\
\noindent 
 Assume there are three possible states of the world: $\omega_1$, $\omega_2$, and $\omega_3$.   Assume there are two assets: a risk-free asset returning $R_f$ in each state, and a risky asset with return $R_1$ in state $\omega_1$, $R_2$ in state $\omega_2$, and $R_3$ in state $\omega_3$.  Assume the probabilities are $1/4$ for state $\omega_1$, $1/2$ for state $\omega_2$, and $1/4$ for state~$\omega_3$.  Assume $R_f=1.0$, and $R_1=1.1$, $R_2 = 1.0$, and $R_3 = 0.9$.
\begin{enumerate}\renewcommand{\labelenumi}{(\alph{enumi})}
\item Prove that there are no arbitrage opportunities.\\
\noindent\sol Let $\tilde{R}$ denote the risky asset return.  A zero-cost portfolio has payoff $\phi(\tilde{R}-R_f)$ for some $\phi$.  This equals $0.1\phi$ in state 1, 0 in state 2, and $-0.1\phi$ in state 3.  Obviously, there is no $\phi$ such that $\phi(\tilde{R}-R_f)$ is nonnegative in all states and positive in some state.
\item Describe the one-dimensional family of state-price vectors $(q_1,q_2,q_3)$.\\
\noindent\sol State prices must satisfy
\begin{align*}
q_1 + q_2 + q_3 &=1\,\\
1.1 q_1 + q_2 + 0.9 q_3 &=1\,.
\end{align*}
Subtracting the top from the bottom shows that $q_3=q_1$ and substituting this into the first shows that $q_2 = 1-2 q_1$.  $q_1$ is arbitrary.
\item Describe the one-dimensional family of SDFs
$$\tilde{m}=(m_1, m_2, m_3)\,,$$
 where $m_i$ denotes the value of the SDF in state $\omega_i$.  Verify that $m_1=4$, $m_2=-2$, $m_3=4$ is an SDF.\\
 \noindent\sol Stochastic discount factors are given by
 $$m_1 = q_1/(1/4) = 4q_1\,, \quad m_2 = q_2/(1/2) = 2-4q_1\,, \quad m_3 = q_3/(1/4) = 4 q_1\,,$$
 with $q_1$ being arbitrary.  Taking $q_1=1$ yields $m_1=4$, $m_2=-2$, $m_3 = 4$.
\item Consider the formula
$$
\tilde{y}_p = \mye[\tilde{y}]+  \Cov(\tilde{X},\tilde{y})'\Sigma_x^{-1}( \tilde{X} -\mye[\tilde{X}])$$
for the projection of a random variable~$\tilde{y}$ onto the linear span of a constant and a random vector~$\tilde{X}$.  When the vector~$\tilde{x}$ has only one component~$\tilde{x}$ (is a scalar),  the formula simplifies to
$$\tilde{y}_p = \mye[\tilde{y}] + \beta (\tilde{x}-\mye[\tilde{x}])\,,$$
where
$$\beta = \frac{\cov(\tilde{x},\tilde{y})}{\var(\tilde{x})}\,.$$
Apply this formula with~$\tilde{y}$ being the SDF $m_1=4$, $m_2=-2$, $m_3=4$ and~$\tilde{x}$ being the risky asset return $R_1=1.1$, $R_2 = 1.0$, $R_3 = 0.9$ to compute the projection of the SDF onto the span of the risk-free and risky assets.\\
\noindent\sol We have $\mye[\tilde{R}]= 1$ and $\mye[\tilde{m}] = 1$ and
$$\cov(\tilde{R},\tilde{y}) = \frac{1}{4}(0.1)(3) + \frac{1}{2}(0)(-3) + \frac{1}{4}(-0.1)(3) = 0\,.$$
Thus, the projection is
$$\tilde{m}_p = \mye[\tilde{m}] = 1\,.$$
\item The projection in part (d) is by definition the payoff of some portfolio.  What is the portfolio?\\
\noindent\sol $\tilde{m}_p$ is the payoff of holding the risk-free asset.
\end{enumerate}



%%%%%%%%%%%%%%%%%%%%%%%%%%%%%%%%%%%%%%%%%%%%%%%%%%%%%
%
%  3.3
%
\newpage\noindent \textbf{Assignment 2: Exercise 3.3}\\
\noindent 
	Assume there is a risk-free asset. Let $\tilde{\textbf{R}}$ denote the vector of risky asset returns, let $\mu$ denote the mean of $\tilde{\textbf{R}}$, and let $\Sigma$ denote the covariance matrix of $\tilde{\textbf{R}}$. Let $\iota$ denote a vector of 1's. Derive the following formula for the SDF $\tilde{m}_p$ from the projection formula (3.32) (need reference link):\\
	$$\tilde{m}_p=\frac{1}{R_f}+\left( \iota-\frac{1}{R_f}\mu \right)'\Sigma^{-1}(\tilde{\textbf{R}}-\mu)\,.$$
\end{prob}
\noindent\sol From he projection formula we have:\\
$$\tilde{m}_p=\bar{m}_p+\cov(\tilde{m}_p,\tilde{\textbf{R}})\Sigma^{-1}(\tilde{\textbf{R}}-\mu)$$
Where $\bar{m}_p$ denote the mean of $\tilde{m}_p$. When a risk-free asset exists, the mean of an SDF is $1/R_f$, so $\bar{m}_p=1/R_f$.
$$\cov(\tilde{m}_p,\tilde{\textbf{R}})=\mye[(\tilde{m}_p-\bar{m}_p)(\tilde{\textbf{R}}-\mu)']=\mye[\tilde{m}_p\tilde{\textbf{R}}]'-\bar{m}_p\mu'=(\iota-\bar{m}_p\mu)'$$
Thus,
$$\tilde{m}_p=\frac{1}{R_f}+\left( \iota-\frac{1}{R_f}\mu \right)'\Sigma^{-1}(\tilde{\textbf{R}}-\mu)\,.$$

%%%%%%%%%%%%%%%%%%%%%%%%%%%%%%%%%%%%%%%%%%%%%%%%%%%%%
%
%  3.4
%
\newpage\noindent \textbf{Assignment 2: Exercise 3.4}\\
\noindent 
	Suppose two random vectors $\tilde{X}$ and $\tilde{Y}$ are joint normally distributed. Explain why the orthogonal projection (3.32) (need reference link) equals $\mye[\tilde{Y}|\tilde{X}]$ 
\end{prob}
\noindent\sol Let $\tilde{Z}_1\sim \mathcal{N}(\textbf{0},\mathcal{I}_{n\times n})$ and $\tilde{Z}_2\sim \mathcal{N}(\textbf{0},\mathcal{I}_{m\times m})$, we can write $\tilde{X}$ and $\tilde{Y}$ in the following form:\\
$$\tilde{X}=\bar{X}+\Sigma_X^{1/2}\tilde{Z}_1$$
$$\tilde{Y}=\bar{Y}+\Sigma_Y^{1/2}[\rho \tilde{Z}_1+(\mathcal{I}-\rho\rho')^{1/2}\tilde{Z}_2]$$
where $\rho=\cov(\tilde{Y})^{-1/2}\cov(\tilde{Y},\tilde{X})\cov(\tilde{X})^{-1/2}$. Thus, 
$$\mye[\tilde{Y}|\tilde{X}]=\mye[\bar{Y}+\Sigma_Y^{1/2}[\rho \tilde{Z}_1+(\mathcal{I}-\rho\rho')^{1/2}\tilde{Z}_2]|\tilde{X}]=\bar{Y}+\Sigma_Y^{1/2}[\rho \Sigma_X^{-1/2}(\tilde{X}-\bar{X})+(\mathcal{I}-\rho\rho')^{1/2}\mye[\tilde{Z}_2|\tilde{X}]]$$
So we have:
$$\mye[\tilde{Y}|\tilde{X}]=\bar{Y}+\Sigma_Y^{1/2}\rho \Sigma_X^{-1/2}(\tilde{X}-\bar{X})=\bar{Y}+\cov(\tilde{Y},\tilde{X})\Sigma_X^{-1}(\tilde{X}-\bar{X})$$ 


\newpage\noindent \textbf{Assignment 3: Exercise 4.1}\\
\noindent 
	Suppose there are~$n$ risky assets with normally distributed payoffs $\tilde{x}_i$. Assume all investors have CARA utility and no labor income $\tilde{y}_h$. Define $\alpha $ to be the aggregate absolute risk aversion as in Section~\ref{s_uf_1}. Assume there is a risk-free asset in zero net supply. Let $\bar{\theta} = (\bar{\theta}_1 \cdots \bar{\theta}_n)'$ denote the vector of supplies of the n risky assets. Let~$\mu$ denote the mean and~$\Sigma$ the covariance matrix of the vector $\tilde{X}=(\tilde{x}_1 \cdots \tilde{x}_n)'$ of asset payoffs. Assume~$\Sigma$ is nonsingular. Suppose the utility functions of investor $h$ are
	$$u_0(c)=-e^{-\alpha_hc} $$ and $$ u_1(c)=-\delta_h e^{-\alpha_hc}$$ 
	Let $\bar{c}_0$ denote the aggregate endowment $\sum_{h=1}^H y_{h0}$ at date 0. 
	\begin{enumerate}\renewcommand{\labelenumi}{(\alph{enumi})}
		\item Use the results of Exercise (2.2) (need reference link) on the optimal demands for the risky assets to show that the equilibrium price vector is
		$$p = \frac{1}{R_f}\left(\mu - \alpha \Sigma \bar{\theta}\right)$$\\
		\noindent\sol From Exercise~\ref{pc_prob_theta}, the optimal portfolio of investor $h$ is
		$$\theta_h =  \frac{1}{\alpha_h }\Sigma^{-1}(\mu-R_fp)\,.$$
		Thus, the aggregate demand for risky assets is
		$$\left(\sum_{h=1}^H\frac{1}{\alpha_h}\right)\Sigma^{-1}(\mu-R_fp) = \frac{1}{\alpha}\Sigma^{-1}(\mu-R_fp)\,.$$
		Market clearing implies
		$$\bar{\theta} = \frac{1}{\alpha}\Sigma^{-1}(\mu-R_fp) \quad \Rightarrow \quad p = \frac{1}{R_f}(\mu-\alpha\Sigma\bar{\theta})\,.$$
		\item Interpret the risk adjustment vector $\alpha\Sigma\bar{\theta}$ in (4.24) (need reference link), explaining in economic terms why a large element of this vector implies an asset has a low price relative to its expected payoff.\\
		\noindent\sol For a fixed expected payoff, the equilibrium price of an asset is lower when the variance of the asset is larger. Moreover, with a higher amount of risk aversion coefficient the effect of variance on price is higher. 
		\item Assume $\delta_h$ is the same for all $h$ (denote the common value by $\delta$). Use the market-clearing condition for the date-0 consumption good to deduce that the equilibrium risk-free return is
		$$R_f=\frac{1}{\delta}exp\left( \alpha\left( \bar{\theta}'\mu-c_0\right) -\frac{1}{2}\alpha^2\bar{\theta}'\Sigma \bar{\theta}\right) $$
		\noindent\sol To derive $R_f$ we need to clear the market for date--0 consumption (or the market for the risk-free asset).  Investor $h$ chooses $c_0$ and $\theta$ to maximize
		\begin{multline*}
		-\exp(-\alpha _hc_0)-\delta_h \mye\bigg[\exp\big(-\alpha _h[(w_{h0}-c_0-p'\theta)R_f + \theta'\tilde{x}]\big)\bigg]\\ = -\exp(-\alpha _hc_0)-\delta_h \exp\big(-\alpha _h(w_{h0}-c_0-p'\theta)R_f\big)\mye\bigg[\exp\big(-\alpha _h \theta'\tilde{x}\big)\bigg]\,.
		\end{multline*}
		Substituting the optimal portfolio $\theta=\theta_h$ yields
		$$\mye\bigg[\exp\big(-\alpha _h \theta'\tilde{x}\big)\bigg] = \exp\left(-(\mu-R_fp)'\Sigma^{-1}\mu + \frac{1}{2}(\mu-R_fp)'\Sigma^{-1}(\mu-R_fp)\right)\,,$$
		and substituting $p=\frac{1}{R_f}(\mu - \alpha \Sigma\bar{\theta})$ yields
		$$\mye\bigg[\exp\big(-\alpha _h \theta'\tilde{x}\big)\bigg] = \exp\left(-\alpha\bar{\theta}'\mu + \frac{\alpha^2}{2}\bar{\theta}'\Sigma\theta\right)\,,$$
		Thus, the first-order condition for $c_{h0}$ is
		$$
		\alpha_h\exp(-\alpha _hc_{h0}) = \delta_h\alpha_hR_f\exp\big(-\alpha _h(w_{h0}-c_{h0}-p'\theta_h)R_f\big)\exp\left(-\alpha\bar{\theta}'\mu + \frac{\alpha^2}{2}\bar{\theta}'\Sigma\theta\right) \,.
		$$
		Dividing by $\alpha_h$, taking logs and rearranging yields
		$$c_{h0} = -\frac{\log \delta_h}{\alpha_h} - \frac{\log R_f}{\alpha_h} + (w_{h0}-c_{h0}-p'\theta_h)R_f +\frac{\alpha}{\alpha_h}\bar{\theta}'\mu - \frac{\alpha^2}{2 \alpha_h}\bar{\theta}'\Sigma\bar{\theta}\,.$$
		By assumption, $\delta$ is the same for all $h$, so
		$$\frac{\log \delta}{\alpha}  = \sum_{h=1}^H \frac{\log \delta_h}{\alpha_h}\,.$$
		Thus, aggregate demand for the consumption good can be expressed as
		$$\sum_{h=1}^H c_{h0} = -\frac{\log \delta}{\alpha} - \frac{\log R_f}{\alpha} + R_f\sum_{h=1}^H(w_{h0}-c_{h0}-p'\theta_h)  + \bar{\theta}'\mu - \frac{\alpha}{2 }\bar{\theta}'\Sigma\bar{\theta}\,.$$
		By market clearing for the risky assets, $\sum_{h=1}^H c_{h0}=\bar{c}$ if and only if $\sum_{h=1}^H w_{h0}-c_{h0}-p'\theta_h=0$.  Thus, each of these is equivalent to
		$$\bar{c}_0 = -\frac{\log \delta}{\alpha} - \frac{\log R_f}{\alpha}   + \bar{\theta}'\mu - \frac{\alpha}{2 }\bar{\theta}'\Sigma\bar{\theta}\,.$$
		The solution of this is
		$$R_f = \frac{1}{\delta}\exp\left\{\alpha \left(\bar{\theta}'\mu - \bar{c}_0\right) - \frac{1}{2}\alpha ^2\bar{\theta}'\Sigma\bar{\theta}\right\}\,.$$
		\item Explain in economic terms why the risk-free return (4.25) (need reference link) is higher when $\bar{\theta}'\mu$ is higher and lower when $\delta$, $\bar{c}_0$, or $\bar{\theta}'\Sigma \bar{\theta}$ is higher.\\
		\noindent\sol $\bar{\theta}'\mu$ is expected aggregate date--1 consumption.  Investors prefer to smooth consumption over time, so when $\bar{\theta}'\mu$ is larger, they wish to borrow to consume more at date~0.  The risk-free return must rise to offset this inclination to borrow.  The reverse is true when $\bar{c}_0$ is larger.  When $\delta$ is higher, investors do not discount the future as much, and hence wish to save to finance date--1 consumption.  The risk-free return must fall to offset this inclination to save.  $\bar{\theta}'\Sigma\bar{\theta}$ is the variance of aggregate date--1 consumption.  When it is larger, there is more risk, and investors expected date--1 utilities are smaller.  They wish to transfer wealth from date~0 to date~1 in this circumstance, and the risk-free return must fall to offset that desire.

	\end{enumerate}	
	
\newpage\noindent \textbf{Assignment 3: Exercise 5.4}\\
\noindent 
	Show that $\mye[\tilde{R}^2] \geq \mye[\tilde{R}_p^2]$ for every return~$\tilde{R}$ (thus, $\tilde{R}_p$ is the minimum second-moment return).  The returns having a given second moment~$a$ are the returns satisfying $\mye[\tilde{R}^2]=a$, which is equivalent to
	$$\var(\tilde{R})+\mye[\tilde{R}]^2 = a\,;$$
	thus, they plot on the circle $x^2 + y^2 = a$ in (standard deviation, mean) space. Use the fact that $\tilde{R}_p$ is the minimum second-moment return to illustrate graphically that $\tilde{R}_p$ must be on the inefficient part of the frontier, with and without a risk-free asset (assuming $\mye[\tilde{R}_p]>0$ in the absence of a risk-free asset).
\end{prob}

\noindent\sol Using Facts 1, 2 and 8,
$$\mye[\tilde{R}^2] = \mye[(\tilde{R_p}+b\tilde{e}_p + \tilde{\varepsilon})^2] = \mye[\tilde{R}_p^2] + b^2 \mye[\tilde{e}_p^2] + \mye[\tilde{\varepsilon}^2] \geq \mye[\tilde{R}_p^2]\,.$$
With a risk-free asset, the cone intersects the vertical axis at $R_f>0$, and the point on the cone closest to the origin is on the lower part.  In the absence of a risk-free asset, the assumption $\mye[\tilde{R}_p]>0$ implies that global minimum variance portfolio has a positive expected return (use the definition of $b\mv$ and Facts 16 and 17 --- which imply $1-\mye[\tilde{e}_p] >0$ --- to deduce this).  Thus, the point on the hyperbola closest to the origin  must be on the lower part of the hyperbola.

	\end{document}
\newpage\noindent \textbf{Assignment 2: Additional Exercise}\\
\noindent 
Suppose there are two risky assets with expected rates of return equal to 8\% and 16\%, standard deviations equal to 25\% and 35\%, and a correlation equal to 30\%. 

    (a)  Compute and plot the mean-variance frontier.

    (b)  Compute the global minimum-variance portfolio.  What is its expected rate of return and standard deviation?
\end{document}