\documentclass[10pt]{beamer}

\usetheme[progressbar=foot]{metropolis}
\usepackage{appendixnumberbeamer}

\usepackage{booktabs}
\usepackage[scale=2]{ccicons}

\usepackage{pgfplots}
\usepgfplotslibrary{dateplot}
\pgfplotsset{compat=1.18} 

\usepackage{xspace}
\usepackage{xcolor}

\DeclareMathOperator{\stdev}{stdev}
\DeclareMathOperator{\var}{var}
\DeclareMathOperator{\cov}{cov}
\DeclareMathOperator{\corr}{corr}
\DeclareMathOperator{\prob}{prob}
\DeclareMathOperator{\n}{n}
\DeclareMathOperator{\N}{N}
\DeclareMathOperator{\Cov}{Cov}

\newcommand{\hlf}{\frac{1}{2}}
\newcommand{\bi}{\begin{itemize}}
\newcommand{\ei}{\end{itemize}}
\newcommand{\im}{\item}
\newcommand{\D}{\mathrm{d}}
\newcommand{\E}{\mathrm{e}}
\newcommand{\mye}{\ensuremath{\mathsf{E}}}
\newcommand{\myreal}{\ensuremath{\mathbb{R}}}
\newcommand{\bq}{\begin{equation}}
\newcommand{\eq}{\end{equation}}
\newcommand{\eqdef}{\;\buildrel \text{d{}ef}\over = \;}
\newcommand{\xstar}{\buildrel *\over X}
\newcommand{\pmax}{p^{\text{max}}}
\newcommand{\qmax}{q^{\text{max}}}
\newcommand{\bfr}{\begin{frame}}
\newcommand{\bfrp}{\begin{frame}[plain]}
\newcommand{\efr}{\end{frame}}
\newcommand{\F}{\mathcal{F}}
\newcommand{\FF}{\mathbb{F}}
\newcommand{\ve}{\varepsilon}
\newcommand{\lh}{\hat{\lambda}}
\definecolor{mycolor}{gray}{0.8}
\definecolor{mymaincolor}{rgb}{0.6862745098039216,0.9333333333333333,0.9333333333333333}
\newcommand{\alr}[1]{\textcolor{blue}{#1}}
\definecolor{LightCyan}{rgb}{0.88,1,1}
\newcommand{\yel}{\cellcolor{yellow}}
\newcommand{\blue}{\cellcolor{SkyBlue}}
\newcommand{\gr}{\cellcolor{SpringGreen}}
\newcommand{\pink}{\cellcolor{pink}}
\newcommand{\apr}{\cellcolor{Apricot}}
\newcommand{\tve}{\tilde{\varepsilon}}
\newcommand{\tw}{\tilde{w}}
\newcommand{\ttth}{\tilde{\theta}}
\newcommand{\te}{\tilde{e}}
\newcommand{\ts}{\tilde{s}}
\newcommand{\tx}{\tilde{x}}
\newcommand{\ty}{\tilde{y}}
\newcommand{\tv}{\tilde{v}}
\newcommand{\tp}{\tilde{p}}
\newcommand{\tF}{\tilde{F}}
\newcommand{\tf}{\tilde{f}}
\newcommand{\tZ}{\tilde{Z}}
\newcommand{\ow}{\overline{w}}
\newcommand{\tm}{\tilde{m}}
\newcommand{\tc}{\tilde{c}}
\newcommand{\tz}{\tilde{z}}
\newcommand{\tr}{\widetilde{R}}
\newcommand{\tR}{\widetilde{\mathbf{R}}}
\newcommand{\bms}{\begin{multline*}}
\newcommand{\ems}{\end{multline*}}
\newcommand{\bas}{\begin{align*}}
\newcommand{\eas}{\end{align*}}
\newcommand{\qr}{\mathbb{Q}}
\newcommand{\tX}{\tilde{X}}
\newcommand{\tY}{\tilde{Y}}

\setbeamertemplate{frame footer}{BUSI 521/ECON 505, Spring 2024}

\title{Chapter 15: Continuous-Time Topics}

\date{}
\author{Kerry Back\\ 
BUSI 521/ECON 505\\
Rice University}


\begin{document}

\maketitle


\section{Risk Neutral Probabilities}
\subsection{}


\bfr\frametitle{Risk-Neutral Probabilities in Continuous Time}
 \bi 
 \im Consider $T<\infty$.
 \im Let $R$ denote the money market account price with $R_0=1$.  
 \im Let~$M$ be an SDF process. Assume $MR$ is a martingale so $$\mye[M_TR_T] \ = \ R_0 \ = \ 1$$   

 \im Define
$$
\qr (A) \ = \  \mye\left[M_TR_T1_A\right]
$$
for each event~$A$ that is distinguishable at date~$T$, where $1_A=1$ when the state of the world is in~$A$ and~0 otherwise. 

\im It follows  that $\qr$ is a probability (measure) and
$$
\mye^*[X_T] \ = \  \mye\left[M_TR_TX_T\right]
$$
for any random variable~$X_T$ depending on date--$T$ information, where $\mye^*$ denotes expectation with respect to $\qr $.  
\ei
\end{frame}


\bfr\frametitle{Risk-Neutral Valuation}
\bi 
\im Let~$W$ be such  that $MW$ is a martingale under the physical probability.  Because we changed the probability using $MR$, a theorem in probability theory tells us that
$$\frac{MW}{MR}$$
is a $\qr$--martingale.

\im So, $W/R$ is a $\qr $--martingale.  Thus,
$$
W_t \ = \  R_t\mye^*_t\left[\frac{W_T}{R_T}\right] \ = \  \mye^*_t\left[\exp\left(-\int_t^T r_u\,\D u\right)W_T\right]\,.
$$
\im In other words, asset values are expected discounted values, taking expectations with respect to the risk neutral probability and discounting at the instantaneous risk-free rate. 

\im It follows that expected returns under the RNP equal the risk-free rate.
\ei 
\end{frame}




\bfr\frametitle{Girsanov's Theorem}
\bi 
\im Let $M$ be an SDF process with 
$$\frac{\D M}{M} \ = \ - r\,\D t - \lambda'\,\D B$$
Here, $r$ and $\lambda$ can be stochastic processes.
\im Define the risk-neutral probability $\qr$ using the martingale $MR$.

\im The vector $B$ is not a vector of Brownian motions under $\qr$ 
\bi
\im Its drift is nonzero.  
\im But, we still have quadratic variation $(\D B)(\D B)' = I\,\D t$, so it is ``close'' to being a vector of Brownian motions.
\ei
\im Girsanov's theorem states that $B^*$ defined by 
 $B^*_0=0$ and
$$\D B^* \ = \ \D B + \lambda\,\D t$$
is a vector of independent Brownian motions under the risk-neutral probability $\qr$.
\ei 
\end{frame}


\bfr\frametitle{Asset Returns under a Risk-Neutral Probability}
\bi 
\im 
 Recall that the vector of asset returns is
$$\frac{\D S}{S} \ = \ \mu\,\D t + \sigma\,\D B$$
\im Define $\D B^* = \D B + \lambda\,\D t$.
 Substitute to obtain
\begin{align*}
\frac{\D S}{S} &\ = \ \mu\,\D t + \sigma\,(\D B^* - \lambda\,\D t)\\
&\ = \ (\mu - \sigma\lambda)\,\D t + \sigma\,\D B^*\\
&\ = \ r\iota\,\D t + \sigma\,\D B^*
\end{align*}
\ei
\end{frame}


\section{Fundamental PDE}\subsection{}
\begin{frame}{Fundamental PDE}
    \bi 
    \im Suppose $S$ is a univariate GBM that is a dividend-reinvested price: $\D S/S = \mu\,\D t + \sigma\,\D B = r\,\D t + \sigma\,\D B^*$.
  \im  Let $f(t,S_t)$ denote the value at $t$ of some at $T>t$ that depends on $S_T$.
    
  \im   The risk-neutral expected rate of return is the risk-free rate, so
    $$\frac{\text{drift of $f$ under RNP}}{f(t,S_t)} = r$$
    \pause
   \begin{align*}
       \D f &= f_t\,\D t + f_S\,\D S 
       + \frac{1}{2}f_{SS}(\D S)^2\\
       &= f_t\,\D t + f_S(rS\,\D t + \sigma S\,\D B^*)
       + \frac{1}{2}f_{SS}(S^2\sigma^2\,\D t)
   \end{align*}
   
   \pause
   \im So 
   $$\text{drift of $f$ under RNP} = f_t + rSf_S + \frac{1}{2}\sigma^2S^2f_{SS}$$
   \ei 
\end{frame}

\begin{frame}[plain]
The fundamental PDE is
\begin{equation}\tag{$\star$}
f_t + rSf_S + \frac{1}{2}\sigma^2S^2f_{SS} = rf
\end{equation}
\end{frame}

\end{document}