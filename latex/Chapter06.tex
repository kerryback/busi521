\documentclass[10pt]{beamer}

\usetheme[progressbar=foot]{metropolis}
\usepackage{appendixnumberbeamer}

\usepackage{booktabs}
\usepackage[scale=2]{ccicons}
\usepackage{comment}
\usepackage{pgfplots}
\usepgfplotslibrary{dateplot}
\pgfplotsset{compat=1.18} 

\usepackage{xspace}
\usepackage{xcolor}

\DeclareMathOperator{\stdev}{stdev}
\DeclareMathOperator{\var}{var}
\DeclareMathOperator{\cov}{cov}
\DeclareMathOperator{\corr}{corr}
\DeclareMathOperator{\prob}{prob}
\DeclareMathOperator{\n}{n}
\DeclareMathOperator{\N}{N}
\DeclareMathOperator{\Cov}{Cov}

\newcommand{\hlf}{\frac{1}{2}}
\newcommand{\bi}{\begin{itemize}}
\newcommand{\ei}{\end{itemize}}
\newcommand{\im}{\item}
\newcommand{\D}{\mathrm{d}}
\newcommand{\E}{\mathrm{e}}
\newcommand{\mye}{\ensuremath{\mathsf{E}}}
\newcommand{\myreal}{\ensuremath{\mathbb{R}}}
\newcommand{\bq}{\begin{equation}}
\newcommand{\eq}{\end{equation}}
\newcommand{\eqdef}{\;\buildrel \text{d{}ef}\over = \;}
\newcommand{\xstar}{\buildrel *\over X}
\newcommand{\pmax}{p^{\text{max}}}
\newcommand{\qmax}{q^{\text{max}}}
\newcommand{\bfr}{\begin{frame}}
\newcommand{\bfrp}{\begin{frame}[plain]}
\newcommand{\efr}{\end{frame}}
\newcommand{\F}{\mathcal{F}}
\newcommand{\FF}{\mathbb{F}}
\newcommand{\ve}{\varepsilon}
\newcommand{\lh}{\hat{\lambda}}
\definecolor{mycolor}{gray}{0.8}
\definecolor{mymaincolor}{rgb}{0.6862745098039216,0.9333333333333333,0.9333333333333333}
\newcommand{\alr}[1]{\textcolor{blue}{#1}}
\definecolor{LightCyan}{rgb}{0.88,1,1}
\newcommand{\yel}{\cellcolor{yellow}}
\newcommand{\blue}{\cellcolor{SkyBlue}}
\newcommand{\gr}{\cellcolor{SpringGreen}}
\newcommand{\pink}{\cellcolor{pink}}
\newcommand{\apr}{\cellcolor{Apricot}}
\newcommand{\tve}{\tilde{\varepsilon}}
\newcommand{\tw}{\tilde{w}}
\newcommand{\ttth}{\tilde{\theta}}
\newcommand{\te}{\tilde{e}}
\newcommand{\ts}{\tilde{s}}
\newcommand{\tx}{\tilde{x}}
\newcommand{\ty}{\tilde{y}}
\newcommand{\tv}{\tilde{v}}
\newcommand{\tp}{\tilde{p}}
\newcommand{\tF}{\tilde{F}}
\newcommand{\tf}{\tilde{f}}
\newcommand{\tZ}{\tilde{Z}}
\newcommand{\ow}{\overline{w}}
\newcommand{\tm}{\tilde{m}}
\newcommand{\tc}{\tilde{c}}
\newcommand{\tz}{\tilde{z}}
\newcommand{\tr}{\widetilde{R}}
\newcommand{\tR}{\widetilde{\mathbf{R}}}
\newcommand{\bms}{\begin{multline*}}
\newcommand{\ems}{\end{multline*}}
\newcommand{\bas}{\begin{align*}}
\newcommand{\eas}{\end{align*}}
\newcommand{\qr}{\mathbb{Q}}
\newcommand{\tX}{\tilde{X}}
\newcommand{\tY}{\tilde{Y}}

\setbeamertemplate{frame footer}{BUSI 521/ECON 505, Spring 2024}

\title{Chapter 6: Factor Models}

\date{}
\author{Kerry Back\\ 
BUSI 521/ECON 505\\
Spring 2024\\
Rice University}


\begin{document}

\maketitle




\section{Factors}\subsection{}

\begin{frame}[plain]
    \bi 
    \im Project (regress) returns $\tr_i$ on random variables $\tf_1, \ldots, \tf_k$ (the same set of $f_j$'s for each $i$):

    $$\tr_i = \alpha_i + \sum_{j=1}^k \beta_{ij} \tf_j + \varepsilon_i$$
    \im We say that the $\tf_1, \ldots, \tf_k$ are pricing factors (or a factor model holds with the $\tf_j$'s as factors) if there exist constants $\lambda_1, \ldots, \lambda_k$ such that
    $$(\forall\; i) \quad \mye[\tr_i] - R_f = \sum_{j=1}^k \lambda_j \beta_{ij}$$
    \im When there is a factor model, only the risks $\beta_{ij}\tf_j$ are priced (contribute to the risk premium).  The $\lambda_j$'s are called prices of risk.
    \ei\end{frame}

    \begin{frame}[plain]
        \bi 
    \im The vector of $\beta$'s is
    $$\cov(\tilde F)^{-1} \cov(\tilde F, \tr_i)$$
    where $F = (\tf_1, \ldots, \tf_k)'$, so
    $$\mye[\tr_i] - R_f = \lambda'\cov(\tilde F)^{-1} \cov(\tilde F, \tr_i)$$
    where $\lambda = (\lambda_1, \ldots, \lambda_k)'$.
    \im Set $\gamma = \cov(\tilde F)^{-1}\lambda$.  We also have
    $$\mye[\tr_i] - R_f = \gamma' \cov(\tilde F, \tr_i)$$
    \im So, we can use either betas or covariances for computing risk premia.
\ei 
\end{frame}


\bfr\frametitle{Risk Adjustments}
\bi 
\im Start with 
$$\mye[\tr_i]-R_f \ = \ \lambda' \Cov(\tF)^{-1}\Cov(\tF,\tr_i)$$
and set $\tr = \tx_i/p_i$ and solve for $p_i$:
$$p_i \ = \ \frac{\mye[\tx_i] - \lambda'\Cov(\tF)^{-1}\Cov(\tF,\tx_i)}{R_f}$$

\im Or, leave $\tr_i$ in the covariance and solve for $p_i$ as
$$p_i \ = \ \frac{\mye[\tx_i]}{R_f + \lambda'\Cov(\tF)^{-1}\Cov(\tF,\tr_i)}$$
\ei 
\end{frame}

\begin{frame}{SDFs}
    \bi 
    \im An SDF is a factor: for all returns,
$$\mye[\tr]-R_f \ = \ -R_f\cov(\tm,\tr) \ = \ -R_f \var(\tm) \cdot \frac{\cov(\tm,\tr)}{\var(\tm)} \ = \ \lambda \cdot \beta$$

\im Factors are SDFs: if $\tF$ is a vector of factors, then there exists $a$ and $b$ such that $a+b'\tF$ is an SDF.
\ei 
\end{frame}

\begin{frame}{Projecting Factors on Returns}
    \bi 
    \im If we do an orthogonal projection onto the span of asset returns, we get a portfolio payoff but not necessarily a return (the cost of the portfolio may not equal 1).

    \im If we want returns, we can instead project on the space of excess returns (the payoffs of zero-cost portfolios; equivalently, the linear span of the excess returns $\tr_i-R_f$).  

    \im For every $\tz$ in the space of excess returns, $R_f + \tz$ is a return.
\ei 
\end{frame}

\begin{frame}{Projections of Factors are Factors}
\bi 
\im Given a vector $\tF$ of factors, project any or all of the factors onto the span of the excess returns and a constant.  We can replace each such factor $\tf_i$ with the excess return $\tz_i$ in its projection.  

\im Why?  Denote the projection of $\tf_i$ by $a_i + \tz_i$ with residual $\te_i$, where $\tz_i$ is an excess return.  
  \bi 
  \im The residual has zero mean and is orthogonal to each excess return, so it has zero covariance with each excess return.  \im Therefore, for each return $\tr$,  
\begin{align*}
\cov(\tr,\tf_i) \ = \ \cov(\tr,a+ \tz_i + \te_i) \ &= \ \cov(\tr,\tz_i) + \cov(\tr,\te_i)\\
&= \ \cov(\tr,\tz_i) + \cov(\tr-R_f,\te_i)\\
&   = \ \cov(\tr,\tz_i)
\end{align*}
\ei 
\im The covariances are unchanged.  Betas change, so the prices of risk $\lambda_i$ change.
\ei 
\end{frame}

\begin{frame}{Frontier Returns}
    \bi 
    \im A frontier return is a factor: $\pi = (1/\delta)\Sigma^{-1}(\mu - R_f\iota)$ implies that risk premia are proportional to covariances with the return of $\pi$.

    \im Linear combinations of projections of factors are frontier returns: 
    \bi 
    \im Project each factor onto the space of excess returns, so we have factors  that are excess returns.  
    \im A vector $\tZ$ of excess returns forms a factor model if and only if $R_f + \gamma'\tZ$ is a frontier return for some $\gamma \in \myreal^k$.  
    \ei
\ei 
\end{frame}

\begin{frame}{Prices of Risk}
    \bi 
    \im We often try excess returns as factors empirically -- for example, the market excess return $\tr_m - R_f$.  
    \im Suppose there is a factor model in which one or more of the factors is an excess return.

    \im The price of risk $\lambda_i$ for each excess-return factor is the mean of the excess return (will prove for CAPM; general proof is similar).

    \im If a factor is a return, its price of risk is its risk premium (the mean of the excess return).

    \im For non-return factors, the price of risk is a free parameter.
\ei 
\end{frame}


\section{CAPM}\subsection{}

\begin{frame}{CAPM}
    \bi 
\im We have shown that risk premia are proportional to covariances with a frontier portfolio.  
\im Suppose the market is a frontier portfolio and let $\tr_m$ denote the market return.  

\im Then, there exists $\gamma$ (the constant of proportionality) such that for all returns $\tr$,
$$\mye[\tr] - R_f \ = \ \gamma \cov(\tr_m,\tr) \ = \ \gamma \var(\tr_m,\tr)\frac{\cov(\tr_m,\tr)}{\var(\tr_m)} \ = \ \lambda \beta$$
\im To compute $\lambda$, apply this to $\tr=\tr_m$ to get
$$\mye[\tr_m]-R_f \ = \ \lambda \beta \ = \ \lambda$$
because the beta of the projection of $\tr_m$ on $\tr_m$ equals 1.  
\im Thus, for all returns $\tr$,
$$\mye[\tr] - R_f \ = \ \frac{\cov(\tr_m,\tr)}{\var(\tr_m)}\big(\mye[\tr_m] - R_f \big)$$ 
\ei 
\end{frame}

\begin{frame}{Risk Adjustment with the CAPM}
    \bi 
\im Under the CAPM, for an asset with date--1 value $\tx$ and date--0 price $p$, we have
$$p \ = \ \frac{\mye[\tx]}{R_f + \big(\mye[\tr_m] - R_f \big)\beta}$$
where
$$\beta \ = \ \frac{\cov(\tr_m,\tr)}{\var(\tr_m)}$$
\im In practice, estimate $\beta$ from company stock returns and the market return.  Apply this formula to value the cash flow $\tx$ from a new project. Invest if $p$ is greater than the cost of the project.
\ei 
\end{frame}

\begin{frame}{CAPM, Marginal Utilities, and SDFs}
    \bi 
\im Suppose there is an investor who optimally holds the market portfolio, so date--1 wealth is $w\tr_m$.  
\im There is an SDF proportional to the investor's marginal utility $u'(w\tr_m)$, so, for all assets,
$$\mye[\tr]-R_f \ = \ \gamma \cov(u'(w\tr_m),\tr)$$
for a constant $\gamma$ (that is the same for all assets).

\im How can we go from $\cov(u'(w\tr_m),\tr)$ to $\cov(\tr_m,\tr)$?
\pause
\begin{enumerate}
\im With normal returns, use Stein's lemma: for joint normals $\tx$ and $\ty$ and a function $f$,
$$\cov(\tx, f(\ty)) \ = \ \mye[f'(\ty)] \cov(\tx, \ty)$$
\im With quadratic utility, marginal utility is affine in wealth: $u'(w\tr)m) = a + b w\tr_m$, so
$$\cov(u'(w\tr_m),\tr) \ = \ b w \cov(\tr_m,\tr)$$
\end{enumerate}
\ei 
\end{frame}

\bfr\frametitle{Testing the CAPM}
\bi
\im Regress excess returns on the market excess return:
$$\tr-R_f = \alpha + \beta(\tr_m-R_f) + \tve$$
\im The regression implies
$$\mye[\tr] -R_f = \alpha + \beta \big(\mye[\tr_m]-R_f\big)$$
\im The CAPM states that $\alpha=0$, so we can test the null $\alpha=0$ for each return.  We can also test the joint null hypothesis $(\alpha_1, \ldots, \alpha_n)' = 0$ for a set of $n$ assets (Gibbons-Ross-Shanken test).
\im Usually we use portfolio returns rather than individual stock returns when testing the CAPM---diversification improves accuracy of $\beta$ estimates.
\im We can follow the same procedure to test any factor model in which the factors are excess returns.
\ei
\end{frame}

\section{Jensen's Alpha}\subsection{}

\begin{frame}{Seeking Alpha}
\bi 
\im Suppose we conjecture a factor model with excess returns as factors
$$\mye[\tr_i] - R_f =  \mye[\tF]'\Cov(\tf)^{-1}\Cov(\tF,\tr_i)\,.$$
\im We test it by running regressions
$$\tr - R_f = \alpha + \beta'\tF + \tve$$
and testing if $\alpha=0$.

\im If a return or portfolio or manager has a positive $\alpha$, it has out-performed, relative to what the model predicts (seeking alpha).
\ei 
\end{frame}

%%%%%%%%%%%%%%%%%%%%%%%%%%%%%%%%%%%%%%%%%%%%%%%%%%%%%%%%%

\begin{frame}{Jensen's Alpha}
    \bi 
\im A standard way to evaluate a portfolio manager is to compute alpha relative to a benchmark (large cap benchmark for large cap manager, small cap benchmark for small cap manager, bond benchmark for bond manager, \ldots).

\im Compared to holding the benchmark, you can improve mean-variance efficiency by adding some of a positive-alpha return.  

\im This is called Jensen's alpha.
\ei 
\end{frame}

%%%%%%%%%%%%%%%%%%%%%%%%%%%%%%%%%%%%%%%%%%%%%%%%%%%%%%%%%

\begin{frame}{$\alpha + \varepsilon$ }
\bi 
\im Let $\tr_b$ denote the benchmark return.  Run the regressions
$$\tr_i - R_f \ = \ \alpha_i + \beta_i (\tr_b-R_f) + \tve_i$$

Consider the excess returns
$$\tr_i - R_f -\beta_i (\tr_b-R_f) \ = \alpha_i + \tve_i$$

\im You can get $\alpha_i$ at zero cost by taking on the residual risk $\tve_i$.  Is it worthwhile to take on this risk?
\ei 
\end{frame}

%%%%%%%%%%%%%%%%%%%%%%%%%%%%%%%%%%%%%%%%%%%%%%%%%%%%%%%%%

\begin{frame}{Utility Improvements}
    \bi 
\im Consider an investor who holds the benchmark and consider a return with a positive alpha.

\im Will she get a utility improvement by adding some of $\alpha_i + \tve_i$?  

\im Suppose she changes from $\tr_b$ to $\tr^* = \tr_b + \lambda(\alpha_i+\tve_i)$ for some (possibly small) $\lambda$.

\im We can do a Taylor series expansion to get
$$\mye[u(\tr^*)] \ \approx \ \mye[u(\tr_b)] + \lambda \mye[u'(\tr_b)(\alpha_i+\tve_i)]$$
\im What is the sign of $\mye[u'(\tr_b)(\alpha_i+\tve_i)]$? 
\bi
\im \pause If returns are normally distributed?
\im \pause If utility is quadratic?
\ei
\ei 
\end{frame}

%%%%%%%%%%%%%%%%%%%%%%%%%%%%%%%%%%%%%%%%%%%%%%%%%%%%%%%%%

\begin{frame}{Mean-Variance Improvements}
    \bi 
\im Let's add some of $\alpha_i+\tve_i$ to $\tr_b$ without changing the mean.

\im To keep the mean constant, we'll subtract some of $\tr_b-\tr_f$.  \im The new return is
$$\tr^* \ = \ \tr_b + \lambda(\alpha_i+\tve_i) - \frac{\lambda\alpha_i}{\mye[\tr_b]-R_f}(\tr_b-\tr_f)$$

\im The new variance is
$$\left(1-\frac{\lambda\alpha_i}{\mye[\tr_b]-R_f}\right)^2\var(\tr_b) + \lambda^2\var(\tve_i)$$
\im Is this larger or smaller than $\var(\tr_b)$?
\ei 
\end{frame}

%%%%%%%%%%%%%%%%%%%%%%%%%%%%%%%%%%%%%%%%%%%%%%%%%%%%%%%%%

\begin{frame}{Factor Models and Jensen's Alpha}
\bi 
\im Suppose we conjecture a factor model with excess returns as factors.

\im Suppose we run the regression
$$\tr_i - R_f = \alpha_i + \beta_i'\tF + \tve_i$$
and decide that a return has a positive alpha.

\im Then adding some of the return will generate utility or mean-variance improvements for an investor who holds 
$$R_b \ = \ R_f + \beta_i'\tF$$
\ei 
\end{frame}

%%%%%%%%%%%%%%%%%%%%%%%%%%%%%%%%%%%%%%%%%%%%%%%%%%%%%%%%%

\section{Fama-French}\subsection{}

\bfr{Fama-French-Carhart Model}
\bi 
 \im Fama-French (1993): SMB and HML in addition to the market return.
\bi
\im SMB = Small Minus Big
\im HML = High book-to-market (value) Minus Low book-to-market (growth)
\ei

\im Carhart (1997): UMD = Up Minus Down

\im Fama-French (2015): RMW and CMA
\bi
\im RMW = Robust (high profits) Minus Weak (low profits)
\im CMA = Conservative (low growth) Minus Aggressive (high growth)
\ei

\im See Hou-Xue-Zhang (2015) also.
\ei 
\end{frame}

\begin{frame}
\bi 
\im Basic idea of CAPM: only market risk should earn a risk premium.  \im We measure market risk by the beta in the projection
$$\tr_i - R_f \ = \ \alpha_i + \beta_i(\tr_m-R_f) + \tve_i$$
\im The CAPM says the risk premium is
$$R_f + \beta_i\big(\mye[\tr_m]-R_f\big)$$
\im The CAPM says the projection residual $\tve_i$ doesn't earn a risk premium.

\im Other models (for example, Fama-French): systematic components of the  residual may earn risk premia, meaning correlations with SMB, HML, RMW, and CMA (and/or UMD, percent change in oil price, \ldots).
\ei 
\end{frame}


\begin{frame}{Construction of Characteristics}
\bi 
\im Portfolios formed at end of June each year

\im Size = market cap at end of June

\im Momentum = compound return over first 11 of prior 12 months

\im Book-to-market, operating profitability, and asset growth all from prior calendar-year annual report (so from end of prior December if fiscal year ends in December and earlier otherwise)
\bi
\im 

Book-to-market = book equity divided by market cap at end of December.  Book equity = shareholders' equity + deferred taxes - preferred stock.

\im Operating profitability = (sales - cogs - sg\&a - interest) / (book equity + minority interests) at end of December

\im Asset growth = percent change in assets
\ei
\ei 
\end{frame}


\begin{frame}{Size and Book-to-Market Sorted Portfolios}

Average monthly returns of value-weighted portfolios 1970-2020.  25 groups formed each month by intersecting 5 size groups and 5 book-to-market groups (based on NYSE quintiles).

\begin{center}
\begin{tabular}{lrrrrr}
\toprule
{} \\
bm &   BM1 &   BM2 &   BM3 &   BM4 &   BM5 \\
size &       &       &       &       &       \\
\midrule
ME1  &  0.59 &  1.17 &  1.12 &  1.32 &  1.37 \\
ME2  &  0.90 &  1.19 &  1.20 &  1.24 &  1.31 \\
ME3  &  0.97 &  1.19 &  1.12 &  1.22 &  1.36 \\
ME4  &  1.07 &  1.06 &  1.09 &  1.17 &  1.22 \\
ME5  &  0.96 &  0.99 &  1.01 &  0.88 &  1.05 \\
\bottomrule
\end{tabular}

\end{center}
\end{frame}

\begin{frame}{Size and Momentum Sorted Portfolios}

Average monthly returns of value-weighted portfolios 1970-2020.  25 groups formed each month by intersecting 5 size groups and 5 momentum groups (based on NYSE quintiles).

\begin{center}
\begin{tabular}{lrrrrr}
\toprule
{} \\
mom & PRIOR1 & PRIOR2 & PRIOR3 & PRIOR4 & PRIOR5 \\
size &        &        &        &        &        \\
\midrule
ME1  &   0.46 &   1.00 &   1.23 &   1.36 &   1.66 \\
ME2  &   0.60 &   1.07 &   1.19 &   1.34 &   1.54 \\
ME3  &   0.71 &   1.00 &   1.10 &   1.15 &   1.49 \\
ME4  &   0.64 &   1.03 &   1.10 &   1.19 &   1.39 \\
ME5  &   0.59 &   0.96 &   0.89 &   0.99 &   1.18 \\
\bottomrule
\end{tabular}

\end{center}
\end{frame}

\begin{frame}{Size and Investment Sorted Portfolios}

Average monthly returns of value-weighted portfolios 1970-2020.  25 groups formed each month by intersecting 5 size groups and 5 investment groups (based on NYSE quintiles).

\begin{center}
\begin{tabular}{lrrrrr}
\toprule
{} \\
inv &  INV1 &  INV2 &  INV3 &  INV4 &  INV5 \\
size &       &       &       &       &       \\
\midrule
ME1  &  1.30 &  1.30 &  1.32 &  1.16 &  0.67 \\
ME2  &  1.23 &  1.25 &  1.28 &  1.24 &  0.87 \\
ME3  &  1.26 &  1.26 &  1.16 &  1.16 &  0.96 \\
ME4  &  1.15 &  1.17 &  1.14 &  1.16 &  1.00 \\
ME5  &  1.18 &  1.02 &  0.96 &  0.96 &  0.89 \\
\bottomrule
\end{tabular}

\end{center}
\end{frame}

\begin{frame}{Size and Profitability Sorted Portfolios}

Average monthly returns of value-weighted portfolios 1970-2020.  25 groups formed each month by intersecting 5 size groups and 5 profitability groups (based on NYSE quintiles).

\begin{center}
\begin{tabular}{lrrrrr}
\toprule
{} \\
prof &   OP1 &   OP2 &   OP3 &   OP4 &   OP5 \\
size &       &       &       &       &       \\
\midrule
ME1  &  0.97 &  1.30 &  1.23 &  1.29 &  1.20 \\
ME2  &  1.03 &  1.15 &  1.19 &  1.18 &  1.33 \\
ME3  &  1.01 &  1.11 &  1.14 &  1.15 &  1.31 \\
ME4  &  1.00 &  1.07 &  1.09 &  1.12 &  1.21 \\
ME5  &  0.74 &  0.83 &  0.90 &  0.93 &  1.01 \\
\bottomrule
\end{tabular}

\end{center}
\end{frame}

\bfr{Construction of FF Factors}
\bi
\im All of the factors are excess returns (long minus short).
\im Split stocks into small and big (NYSE 50th percentile).
\im Split by the characteristic (book-to-market, momentum, profitability, or investment) as 30\%,40\%,30\%.  \im Intersect the sorts to form 6 groups.  HML, UMD, RMW, and CMA are all defined as:
$$\frac{1}{2}\text{TopBig} + \frac{1}{2}\text{TopSmall} - \frac{1}{2}\text{BottomBig} - \frac{1}{2}\text{BottomSmall}$$
\im SMB $\,=\frac{1}{2}\text{TopSmall} + \frac{1}{2}\text{BottomSmall} - \left(\frac{1}{2}\text{TopBig} - \frac{1}{2}\text{BottomBig}\right)$
\im There is an SMB for each sort (value/growth, profitability, asset growth).  Average the three SMBs.
\ei
\end{frame}

\section{APT}\subsection{}

\bfr\frametitle{Statistical Factor Models}
\bi 
\im A factor pricing model explains \alert{risk premia} in terms of covariances or betas with respect to a set of factors.

 \im A statistical factor model explains \alert{covariances} in terms of covariances with respect to a set of factors.

 \im The residual in the projection of a return $\tr$ onto the span of a constant and $\tX = (\tx_1, \ldots, \tx_k)'$ is
\bq\tag{$\star$}
\tve \eqdef \tr - \left\{\overline{R} 
+ \Cov(\widetilde{X},\tr)'\Cov(\widetilde{X})^{-1} [\tX - \overline{X}]\right\}
\eq

 \im We say that returns satisfy a \alert{statistical factor model} with factors $\tX$ if the residuals of the returns are pairwise \alert{uncorrelated}.  The residuals are called idiosyncratic risks.
 \ei 
\end{frame}

\bfr\frametitle{}
\bi 
 \im For any return $\tr_i$, let $\beta_i$ denote the column vector of multiple regression betas:
$$\beta_i \ = \ \Cov(\widetilde{X})^{-1}\Cov(\widetilde{X},\tr_i)$$

 \im Equation ($\star$) is $\tr_i = \overline{R}_i + \beta_i'[\tX - \overline{X}] + \tve_i$.

 \im Because the $\tve$'s are mutually uncorrelated (and orthogonal to the factors),
$$\cov(\tr_i,\tr_j) \ = \ \cov(\beta_i'[\tX - \overline{X}] + \tve_i,\beta_j'[\tX - \overline{X}] + \tve_j] = \beta_i'\Cov(X)\beta_j$$

 \im Thus, the covariance between any two returns depends only on their exposures to the common factors (their betas).  Those exposures are called systematic risks.

 \im Simplest model: market model.  Only factor is the market return.  All stocks tend to go up and down with the market.  Other risks are uncorrelated.
\ei 
\end{frame}

\bfr\frametitle{Arbitrage Pricing Theory}
\bi 
 \im The APT asserts that statistical factors are also pricing factors (approximately).

 \im Intuition: In a statistical factor model, only systematic risks should earn risk premia, because idiosyncratic risks can be diversified away.  
 \im Thus, risk premia should depend on betas.
\ei 
\end{frame}

\bfr\frametitle{APT with No Idiosyncratic Risks}
\bi 
 \im To get intuition about the APT, assume there are no idiosyncratic risks: $\tr_i = \overline{R}_i + \beta_i'[\tX - \overline{X}]$.  

 \im Assume there is an SDF (this is the arbitrage part of the APT).  Then
\begin{align*}
 1 = \mye[\tm\tr] \ &= \ \mye[\tm]\overline{R} + \beta'\mye[\tm(\tX - \overline{X})]\\
 &= \ \mye[\tm]\overline{R} + \beta'\Cov(\tX,\tm)
 \end{align*}
 
  \im So
 $$\mye[\tr] \ = \ \frac{1}{\mye[\tm]} - \frac{1}{\mye[\tm]}\beta'\Cov(\tX,\tm)$$
 
  \im Thus, the vector of prices of risk is
 $$\lambda \eqdef - \frac{1}{\mye[\tm]}\Cov(\tX,\tm)$$
\ei 
\end{frame}

\bfr\frametitle{APT with Idiosyncratic Risks}
\bi 
 \im Following the same algebra with $\tve \neq 0$ yields
$$\mye[\tr] \ = \ \frac{1}{\mye[\tm]} +\beta'\lambda - \frac{\mye[\tm\tve]}{\mye[\tm]}$$

 \im Call
$$\delta \eqdef - \frac{\mye[\tm\tve]}{\mye[\tm]}$$
the pricing error.

 \im Assume there is an infinite number of assets $i = 1, 2, \ldots$ and an SDF.  We can show that
$\sum_{i=1}^\infty \delta_i^2 < \infty$.

 \im Thus, for any given number $\epsilon>0$, most (all but a finite number) of the pricing errors are small (satisfy
$|\delta_i|\ < \epsilon$).
\ei 
\end{frame}


\end{document}
\begin{comment}
    \begin{frame}{Best Linear Approximations of Random Variables}
    \bi
    Given two (finite-variance) random variables $\tx$ and $\ty$, there exist $\alpha$ and $\beta$ such that $\alpha + \beta \tx$ is the best linear (affine) approximation of $\ty$ in the sense of minimizing the mean squared error:
    $$ \mye[(\ty - \alpha - \beta \tx)^2]\,.$$
    The solution is $\beta = \cov(\tx,\ty)/\var(\tx)$ and 
    $$\alpha = \mye[\ty] - \beta \mye[\tx]$$
    Defining the error (residual) $\tve = \ty - \alpha - \beta \tx$ for the optimal $\alpha$ and $\beta$, we have  $\mye[\tve]= 0$ and $\cov(\tve,\tx)=0$.
    
    An equivalent definition of $\alpha$ and $\beta$ is that they are the unique pair such that
    $$\ty \ = \ \alpha + \beta \tx + \tve$$
    with $\mye[\tve]= 0$ and $\cov(\tve,\tx)=0$.
    \ei 
    \end{frame}
    
    
    \begin{frame}{Orthogonal Projection}
        \bi 
    The phrase ``best linear approximation'' has a vector-space interpretation.  Let $\mathcal{L}^2$ denote the space of finite-variance random variables.  Define an inner product by $\langle \tx,\ty \rangle = \mye[\tx\ty]$ and orthogonality by $\mye[\tx\ty] = 0$.  
    
    Given a random variable $\tx \in \mathcal{L}^2$, let $\mathcal{M} = \{ a + b \tx \mid (a,b) \in \myreal^2\}$.  The set $\mathcal{M}$ is the subspace of $\mathcal{L}^2$ that is spanned by a constant and $\tx$.  
    
    Given another random variable $\ty \in \mathcal{L}^2$, the best linear approximation $\alpha + \beta \ty$ defined on the previous slide is the orthogonal projection of $\ty$ onto $\mathcal{M}$.  It is the closest point in $\mathcal{M}$ to $\ty$ in the sense of the norm (consistent with the inner product) $\parallel \tz \parallel = \sqrt{\mye[\tz^2]}$.
    \ei
    \end{frame}
    
    \begin{frame}{A Different Orthogonal Projection}
    \bi 
    There always exists $\alpha$ and $\beta$ such that
    $$\ty \ = \ \alpha + \beta \tx + \tve$$
    with $\mye[\tve]= 0$ and $\cov(\tve,\tx)=0$.
    However, it is not generally true that $\mye[\tve \mid \tx] = 0$.
    
    Define 
    $$\mathcal{N} = \{f(\tx) \mid f \text{\,is a function such that $f(\tx) \in \mathcal{L}^2$}\}$$  Then, $\mathcal{M} \subset \mathcal{N}$.  There also exists an orthogonal projection of $\ty$ onto $\mathcal{N}$.  This projection is the random variable $\mye[\ty \mid \tx]$.  If we define the error $\te = \ty - \mye[\ty \mid \tx]$, then we have $\mye[\te \mid \tx] = 0$.
    
    If $\mye[\ty \mid \tx] \neq \alpha + \beta \tx$, then $\mye[\ty \mid \tx]$ is closer to $\ty$ in the sense of the norm defined on the previous slide.
    \ei 
    \end{frame}
    
    \end{comment}