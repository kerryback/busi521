\documentclass[xcolor=dvipsnames,10pt]{beamer}

\mode<presentation> {\usetheme{Singapore}}
\usepackage{pgfpages}

%\setbeamercovered{transparent} 
\usepackage[english]{babel}
\usepackage[latin1]{inputenc}
\usepackage{times,amsfonts}
\usepackage[T1]{fontenc}
\setlength{\parskip}{\baselineskip}
% Or whatever. Note that the encoding and the font should match. If T1
% does not look nice, try deleting the line with the fontenc.

%\usecolortheme{sidebartab}
\setbeamertemplate{itemize item}[triangle]



\usepackage{calc}
\usepackage{environ}
\newcommand{\halfmargin}{0.0001\paperwidth}


\RequirePackage{booktabs,colortbl,ulem}

\usepackage{animate}
\RequirePackage{booktabs,colortbl,gensymb}
\setlength{\parskip}{\baselineskip}

\usepackage{calc}
\usepackage{environ}

% \newcommand{\halfmargin}{0.0001\paperwidth}


\NewEnviron{wideframe}[1][]{%
\begin{frame}{#1}
\makebox[\textwidth][c]{
\begin{minipage}{\dimexpr\paperwidth-\halfmargin-\halfmargin\relax}
\BODY
\end{minipage}}
\end{frame}
}


\DeclareMathOperator{\stdev}{stdev}
\DeclareMathOperator{\var}{var}
\DeclareMathOperator{\cov}{cov}
\DeclareMathOperator{\corr}{corr}
\DeclareMathOperator{\prob}{prob}
\DeclareMathOperator{\n}{n}
\DeclareMathOperator{\N}{N}
\DeclareMathOperator{\Cov}{Cov}

\newcommand{\hlf}{\frac{1}{2}}
\newcommand{\bi}{\begin{itemize}}
\newcommand{\ei}{\end{itemize}}
\newcommand{\im}{\item}
\newcommand{\D}{\mathrm{d}}
\newcommand{\E}{\mathrm{e}}
\newcommand{\mye}{\ensuremath{\mathsf{E}}}
\newcommand{\myreal}{\ensuremath{\mathbb{R}}}
\newcommand{\bq}{\begin{equation}}
\newcommand{\eq}{\end{equation}}
\newcommand{\eqdef}{\;\buildrel \text{d{}ef}\over = \;}
\newcommand{\xstar}{\buildrel *\over X}
\newcommand{\pmax}{p^{\text{max}}}
\newcommand{\qmax}{q^{\text{max}}}
\newcommand{\bfr}{\begin{frame}}
\newcommand{\bfrp}{\begin{frame}[plain]}
\newcommand{\efr}{\end{frame}}
\newcommand{\F}{\mathcal{F}}
\newcommand{\FF}{\mathbb{F}}
\newcommand{\ve}{\varepsilon}
\newcommand{\lh}{\hat{\lambda}}
\definecolor{mycolor}{gray}{0.8}
\definecolor{mymaincolor}{rgb}{0.6862745098039216,0.9333333333333333,0.9333333333333333}
\newcommand{\alr}[1]{\textcolor{blue}{#1}}
\definecolor{LightCyan}{rgb}{0.88,1,1}
\newcommand{\yel}{\cellcolor{yellow}}
\newcommand{\blue}{\cellcolor{SkyBlue}}
\newcommand{\gr}{\cellcolor{SpringGreen}}
\newcommand{\pink}{\cellcolor{pink}}
\newcommand{\apr}{\cellcolor{Apricot}}
\newcommand{\tve}{\tilde{\varepsilon}}
\newcommand{\tw}{\tilde{w}}
\newcommand{\ttth}{\tilde{\theta}}
\newcommand{\te}{\tilde{e}}
\newcommand{\ts}{\tilde{s}}
\newcommand{\tx}{\tilde{x}}
\newcommand{\ty}{\tilde{y}}
\newcommand{\tv}{\tilde{v}}
\newcommand{\tp}{\tilde{p}}
\newcommand{\tF}{\tilde{F}}
\newcommand{\tf}{\tilde{f}}
\newcommand{\tZ}{\tilde{Z}}
\newcommand{\ow}{\overline{w}}
\newcommand{\lb}{\left[}
\newcommand{\rb}{\right]}
\newcommand{\lp}{\left(}
\newcommand{\rp}{\right)}
\newcommand{\tm}{\tilde{m}}
\newcommand{\tc}{\tilde{c}}
\newcommand{\tz}{\tilde{z}}
\newcommand{\str}[1]{\textcolor{blue}{\sout{#1}}}
\newcommand{\tr}{\widetilde{R}}
\newcommand{\tR}{\widetilde{\mathbf{R}}}
\newcommand{\bms}{\begin{multline*}}
\newcommand{\ems}{\end{multline*}}
\newcommand{\bas}{\begin{align*}}
\newcommand{\eas}{\end{align*}}
\newcommand{\qr}{\mathbb{Q}}
\newcommand{\IMAGES}{/home/kerry/Dropbox/Images}
\newcommand{\tX}{\tilde{X}}
\newcommand{\tY}{\tilde{Y}}

\author{\vskip 0.5in \small Kerry Back \\BUSI 521--ECON 505\\ Rice University \\Spring 2022}
%\institute{Rice University\\ Spring 2019}
\date[]





\newcommand{\tu}{\tilde{u}}
\begin{document}
\title{\vskip 0.5in Day 19}
\subtitle{Continuous-Time Securities Markets}

\begin{frame}
  \titlepage
\end{frame}

\section{Market Model}
\subsection{}

\bfr\frametitle{Securities Market Model}
 Money market account has price $R$ with $\D R/R = r\,\D t$. 
 $n$ locally risky assets with dividend-reinvested prices $S_i$.
$\mu=\,$ vector of $n$ stochastic processes $\mu_i$.
 $\sigma=n \times k$ matrix of stochastic processes, $\Sigma = \sigma\sigma'$.
 $B =\,$ vector of $k$ independent Brownian motions.  $k \geq n$. 
 
$$\D S/S \ \eqdef \ \begin{pmatrix}\D S_{1t}/S_{1t} \\ \vdots \\ \D S_{nt}/S_{nt} \end{pmatrix}\ = \  \mu_t\,\D t + \sigma_t\,\D B_t $$
This means that, for $i=1,\ldots,n$,
$$\frac{\D S_{it}}{S_{it}} \ = \  \mu_{it}\,\D t + \sum_{j=1}^k \sigma_{ijt}\,\D B_{jt}$$
\end{frame}

\bfr\frametitle{Intertemporal Budget Constraint}
 Let $\phi_i$ denote the amount of the consumption good invested in risky asset~$i$.
 Let $W=\,$ wealth, $C=\,$ consumption, $Y=\,$ labor income.
 The intertemporal budget constraint is
$$\D W = (Y-C)\,\D t + \theta'\,\D S + (W-\theta'S)r\,\D t$$
where $\theta=(\theta_1,\ldots,\theta_n)'$ denotes share holdings.
Setting $\phi_i = \theta_iS_i$, we obtain
$$\D W = (Y-C)\,\D t + \phi'\,(\D S/S) + (W-\phi'\iota)r\,\D t$$
Equivalently,
$$\D W = (Y-C)\,\D t + rW\,\D t + \phi'(\D S/S - r\iota)\,\D t$$
Equivalently,
$$\D W = (Y-C)\,\D t + rW\,\D t+\phi'(\mu-r\iota)\,\D t + \phi'\sigma\,\D B$$
\end{frame}

\begin{frame}
 Assuming $W>0$, we can define $\pi=\phi/W$ and write the intertemporal budget constraint as
$$\D W = (Y-C)\,\D t + rW\,\D t+W\pi'(\mu-r\iota)\,\D t + W\pi'\sigma\,\D B$$
Equivalently,
$$\frac{\D W}{W} = \frac{Y-C}{W}\,\D t + r\,\D t+\pi'(\mu-r\iota)\,\D t + \pi'\sigma\,\D B$$
If $Y=C=0$, the wealth process is said to be self financing.
\end{frame}


\section{SDF Processes}
\subsection{}

\bfr\frametitle{Stochastic Discount Factor Processes}
Define a stochastic process ~$M$ to be an SDF process if 
\bi
\im $M_0=1$
\im $M_t>0$ for all~$t$ with probability~1
\im $MR$ is a local martingale, where $R$ denotes the price of the money market account,
\im $MS_i$ is a local martingale, for $i=1,\ldots,n$, where the $S_i$ are the dividend-reinvested asset prices.
\ei
`Local martingale' means zero drift (no $\D t$ part).  We can show that \alert{if $M$ is an SDF process and $W$ is a self-financing wealth process, then $MW$ is a local martingale}.
\end{frame}

\bfr\frametitle{Dynamics of SDF Processes}
 We can show the following:
A stochastic process $M>0$ with $M_0=1$ is an SDF process if and only if
$$\frac{\D M}{M} \ = \ - r\,\D t - \lambda'\,\D B$$
for a stochastic process $\lambda$ satisfying
$$\sigma\lambda = \mu-r\iota$$
 Notice that
\begin{align*} (\D S/S)\left(\frac{\D M}{M} \right) \ & = \ -(\sigma\,\D B)(\lambda'\,\D B)\\
 \ & = \ -(\sigma\,\D B)(\D B)'\lambda  \ = \ -\sigma\lambda\,\D t  \ = \ - (\mu-r\iota)\,\D t
 \end{align*}
  So,
 $$ (\mu-r\iota)\,\D t \ = \ -(\D S/S)\left(\frac{\D M}{M} \right)$$
 \end{frame}

\bfr\frametitle{Factor Pricing and Prices of Risk}
 As in a single-period model, an SDF process is a pricing factor.  We read the equation 
$$ (\mu-r\iota)\,\D t \ = \ -(\D S/S)\left(\frac{\D M}{M} \right)$$
as saying \alert{the risk premium of each asset is minus its covariance with the SDF process}.

 Notice that
$$-(\D S/S)\left(\frac{\D M}{M} \right) \ = \ -(\D S/S)(-\lambda'\,\D B) 
\ = \ \sum_{j=1}^k \lambda_j (\D S/S)(\D B_j)$$
 We call $\lambda$ the vector of `prices of risk.' Each $\lambda_j$ is the `price' for the covariance with $B_j$.
 If there is only a single risky asset, then $\lambda = (\mu-r)/\sigma$, which is the Sharpe ratio of the risky asset.

\end{frame}

\bfr\frametitle{Projections of SDF Processes}
 One solution $\lambda$ of the equation $\sigma\lambda = \mu-r\iota$ is 
$$\lambda_p \eqdef \sigma'(\sigma\sigma')^{-1}(\mu-r\iota) = \sigma'\Sigma^{-1}(\mu-r\iota)$$
 For this solution,
\begin{align*}
\lambda_p'\,\D B &= (\mu-r\iota)'\Sigma^{-1}\sigma\,\D B\\
&= \pi'\sigma\,\D B
\end{align*}
for $\pi = \Sigma^{-1}(\mu-r\iota)$ (the log-optimal portfolio).  Thus, it is spanned by the assets.

 Every solution $\lambda$ of the equation $\sigma\lambda = \mu-r\iota$ is of the form
$$\lambda = \lambda_p + \zeta$$
where $\zeta$ is orthogonal to the assets in the sense that $\sigma\zeta=0$.
\end{frame}

\section{Valuation}\subsection{}

\bfr\frametitle{Valuation}
 If $MW$ is a martingale, for $u>t$,
$$M_tW_t = \mye_t[M_uW_u] \quad \Leftrightarrow \quad W_t = \mye_t\left[\frac{M_u}{M_t}W_u\right]$$
 Under another martingale assumption, if $(C,\pi,W)$ satisfy the intertemporal budget constraint with $Y=0$,
then
$$W_t = \mye_t \left[\int_t^u \frac{M_\tau}{M_t}C_\tau\,\D \tau + \frac{M_u}{M_t}W_u\right]$$
 In particular, for an asset with price process $P$ and dividend process $D$,
$$P_t = \mye_t \left[\int_t^u \frac{M_\tau}{M_t}D_\tau\,\D \tau + \frac{M_u}{M_t}P_u\right]$$
\end{frame}

\section{Euler Equation}\subsection{}
\bfr\frametitle{Euler Equation}
 The Euler equation is a necessary and sufficient condition for optimality for the investor with time-additive utility
$$\mye \int_0^\infty \E^{-\delta t}u(C_t)\,\D t$$
 The Euler equation is that the MRS
$$\frac{\E^{-\delta t}u'(C_t)}{u'(C_0)}$$
is an SDF process.

\end{frame}

\section{Representative Investor}\subsection{}
\begin{frame}{CRRA Representative Investor}
Applying the Euler equation for a CRRA representative investor, we have
$$M_t = \E^{-\delta t}\left(\frac{C_t}{C_0}\right)^{-\rho}$$
(``Assuming there is no bubble in the price of the market portfolio'') the price is
$$P_t = \mye_t \int_t^\infty \E^{-\delta(\tau-t)}\left(\frac{C_\tau}{C_t}\right)^{-\rho}C_\tau\,\D \tau$$
So, the price-dividend ratio is
$$\frac{P_t}{C_t} =  \int_t^\infty \E^{-\delta(\tau-t)}\mye_t\left[\left(\frac{C_\tau}{C_t}\right)^{1-\rho}\right]\,\D \tau$$
\end{frame}

\begin{frame}{IID Consumption Growth}
Assume
$$\frac{\D C}{C} = \alpha\,\D t + \gamma'\,\D B$$
for constant $\alpha$ and $\gamma$ (geometric Brownian motion).  Then
$$\D \log C = \left(\alpha - \frac{1}{2}\gamma'\gamma\right)\,\D t + \gamma'\,\D B$$
This implies
$$C_\tau = C_t \E^{(\alpha - \gamma'\gamma/2)(\tau-t) + \gamma'(B_\tau-B_t)}$$
The exponent is normal with mean $(\tau-t)(\alpha-\gamma'\gamma/2)$ and variance $(\tau-t)\gamma'\gamma$.
We can easily calculate 
$$\mye_t\left[\left(\frac{C_\tau}{C_t}\right)^{1-\rho}\right]$$ as $\E^{-\eta(\tau-t)}$ for a constant $\eta$ and then, assuming $\eta>0$, integrate to get 
$$\frac{P_t}{C_t} = \frac{1}{\delta+\eta}$$
\end{frame}

\section{RN Probs}
\subsection{}


\bfr\frametitle{Risk-Neutral Probabilities in Continuous Time}
 Consider $T<\infty$.
 Let $R$ denote the money market account price with $R_0=1$.  
 Let~$M$ be an SDF process. Assume $MR$ is a martingale so $$\mye[M_TR_T] \ = \ R_0 \ = \ 1$$   

 Define
$$
\qr (A) \ = \  \mye\left[M_TR_T1_A\right]
$$
for each event~$A$ that is distinguishable at date~$T$, where $1_A=1$ when the state of the world is in~$A$ and~0 otherwise. 

 It follows  that $\qr$ is a probability (measure) and
$$
\mye^*[X_T] \ = \  \mye\left[M_TR_TX_T\right]
$$
for any random variable~$X_T$ depending on date--$T$ information, where $\mye^*$ denotes expectation with respect to $\qr $.  
\end{frame}


\bfr\frametitle{Risk-Neutral Valuation}
 Let~$W$ be such  that $MW$ is a martingale under the physical probability.  Because we changed the probability using $MR$, a theorem in probability theory tells us that
$$\frac{MW}{MR}$$
is a $\qr$--martingale.

 So, $W/R$ is a $\qr $--martingale.  Thus,
$$
W_t \ = \  R_t\mye^*_t\left[\frac{W_T}{R_T}\right] \ = \  \mye^*_t\left[\exp\left(-\int_t^T r_u\,\D u\right)W_T\right]\,.
$$
 In other words, asset values are expected discounted values, taking expectations with respect to the risk neutral probability and discounting at the instantaneous risk-free rate. 

It follows that expected returns under the RNP equal the risk-free rate.

\end{frame}




\bfr\frametitle{Girsanov's Theorem}
 Let $M$ be an SDF process with 
$$\frac{\D M}{M} \ = \ - r\,\D t - \lambda'\,\D B$$
Here, $r$ and $\lambda$ can be stochastic processes.
 Define the risk-neutral probability $\qr$ using the martingale $MR$.

 The vector $B$ is not a vector of Brownian motions under $\qr$ 
\bi
\im Its drift is nonzero.  
\im But, we still have quadratic variation $(\D B)(\D B)' = I\,\D t$, so it is ``close'' to being a vector of Brownian motions.
\ei
 Girsanov's theorem states that $B^*$ defined by 
 $B^*_0=0$ and
$$\D B^* \ = \ \D B + \lambda\,\D t$$
is a vector of independent Brownian motions under the risk-neutral probability $\qr$.
\end{frame}


\bfr\frametitle{Asset Returns under a Risk-Neutral Probability}
 Recall that the vector of asset returns is
$$\frac{\D S}{S} \ = \ \mu\,\D t + \sigma\,\D B$$
 Define $\D B^* = \D B + \lambda\,\D t$.
 Substitute to obtain
\begin{align*}
\frac{\D S}{S} &\ = \ \mu\,\D t + \sigma\,(\D B^* - \lambda\,\D t)\\
&\ = \ (\mu - \sigma\lambda)\,\D t + \sigma\,\D B^*\\
&\ = \ r\iota\,\D t + \sigma\,\D B^*
\end{align*}
\end{frame}

\end{document}