\documentclass[10pt]{beamer}

\usetheme[progressbar=foot]{metropolis}
\usepackage{appendixnumberbeamer}

\usepackage{booktabs}
\usepackage[scale=2]{ccicons}

\usepackage{pgfplots}
\usepgfplotslibrary{dateplot}
\pgfplotsset{compat=1.18} 

\usepackage{xspace}
\usepackage{xcolor}

\DeclareMathOperator{\stdev}{stdev}
\DeclareMathOperator{\var}{var}
\DeclareMathOperator{\cov}{cov}
\DeclareMathOperator{\corr}{corr}
\DeclareMathOperator{\prob}{prob}
\DeclareMathOperator{\n}{n}
\DeclareMathOperator{\N}{N}
\DeclareMathOperator{\Cov}{Cov}

\newcommand{\hlf}{\frac{1}{2}}
\newcommand{\bi}{\begin{itemize}}
\newcommand{\ei}{\end{itemize}}
\newcommand{\im}{\item}
\newcommand{\D}{\mathrm{d}}
\newcommand{\E}{\mathrm{e}}
\newcommand{\mye}{\ensuremath{\mathsf{E}}}
\newcommand{\myreal}{\ensuremath{\mathbb{R}}}
\newcommand{\bq}{\begin{equation}}
\newcommand{\eq}{\end{equation}}
\newcommand{\eqdef}{\;\buildrel \text{d{}ef}\over = \;}
\newcommand{\xstar}{\buildrel *\over X}
\newcommand{\pmax}{p^{\text{max}}}
\newcommand{\qmax}{q^{\text{max}}}
\newcommand{\bfr}{\begin{frame}}
\newcommand{\bfrp}{\begin{frame}[plain]}
\newcommand{\efr}{\end{frame}}
\newcommand{\F}{\mathcal{F}}
\newcommand{\FF}{\mathbb{F}}
\newcommand{\ve}{\varepsilon}
\newcommand{\lh}{\hat{\lambda}}
\definecolor{mycolor}{gray}{0.8}
\definecolor{mymaincolor}{rgb}{0.6862745098039216,0.9333333333333333,0.9333333333333333}
\newcommand{\alr}[1]{\textcolor{blue}{#1}}
\definecolor{LightCyan}{rgb}{0.88,1,1}
\newcommand{\yel}{\cellcolor{yellow}}
\newcommand{\blue}{\cellcolor{SkyBlue}}
\newcommand{\gr}{\cellcolor{SpringGreen}}
\newcommand{\pink}{\cellcolor{pink}}
\newcommand{\apr}{\cellcolor{Apricot}}
\newcommand{\tve}{\tilde{\varepsilon}}
\newcommand{\tw}{\tilde{w}}
\newcommand{\ttth}{\tilde{\theta}}
\newcommand{\te}{\tilde{e}}
\newcommand{\ts}{\tilde{s}}
\newcommand{\tx}{\tilde{x}}
\newcommand{\ty}{\tilde{y}}
\newcommand{\tv}{\tilde{v}}
\newcommand{\tp}{\tilde{p}}
\newcommand{\tF}{\tilde{F}}
\newcommand{\tf}{\tilde{f}}
\newcommand{\tZ}{\tilde{Z}}
\newcommand{\ow}{\overline{w}}
\newcommand{\tm}{\tilde{m}}
\newcommand{\tc}{\tilde{c}}
\newcommand{\tz}{\tilde{z}}
\newcommand{\tr}{\widetilde{R}}
\newcommand{\tR}{\widetilde{\mathbf{R}}}
\newcommand{\bms}{\begin{multline*}}
\newcommand{\ems}{\end{multline*}}
\newcommand{\bas}{\begin{align*}}
\newcommand{\eas}{\end{align*}}
\newcommand{\qr}{\mathbb{Q}}
\newcommand{\tX}{\tilde{X}}
\newcommand{\tY}{\tilde{Y}}

\setbeamertemplate{frame footer}{BUSI 521/ECON 505, Spring 2024}

\title{Chapter 8: Dynamic Securities Markets}

\date{}
\author{Kerry Back\\ 
BUSI 521/ECON 505\\
Spring 2024\\
Rice University}


\begin{document}

\maketitle


\bfr\frametitle{Assets and Returns}
\bi \im Dates $t=0,1,2,\ldots$.  No tildes anymore for random things.  Information grows over time as random variables are observed.
 \im 
$D_{it}=\,$ dividend of asset $i$ at date $t$.
 Ex-dividend price $P_{it}>0$.  
 \im 
Return from $t$ to $t+1$ is
$$R_{i,t+1} := \frac{P_{i,t+1}+D_{i,t+1}}{P_{it}}$$

\im Risk-free return from $t$ to $t+1$ is $R_{f,t+1}$.  Known at $t$ (so risk-free from $t$ to $t+1$) but maybe not known until $t$ (randomly evolving interest rates).
\ei
\end{frame}

\begin{frame}{Iterated Expectations}
    \bi 
    \im Let $\mye_t$ denote expectation given information at date $t$.
    \im Assume information is nondecreasing over time.
    \im For any $s<t<u$ and random variable $X_u$ known at date $u$,
    $$\mye_s[X_u] = \mye_s\bigg[ \mye_t[X_u]\bigg]$$
    \ei
 \end{frame}

\section{SDFs}\subsection{}

\begin{frame}{One-Period SDFs}

\bi 
\im SDF at $t$ for pricing at $t+1$ is a r.v. $Z_{t+1}$ depending on date $t+1$ information such that
$$\mye_t[Z_{t+1}R_{i, t+1}] = 1$$
for all assets $i$.
\im Equivalently, price at $t$ of any portfolio payoff $X_{t+1}$ at $t+1$ is
$$\mye_t[Z_{t+1}X_{t+1}]$$
\im With no uncertainty or with risk neutrality,
$$Z_{t+1} = \frac{1}{R_{f, t+1}} := \frac{1}{1+r_{f, t+1}}$$
\ei
\end{frame}

\begin{frame}{Two-Period SDFs}
\bi 
\im How to price at $t-1$ a payoff at $t+1$?
\pause 
\im Same as price at $t-1$ of enough dollars at $t$ to buy the payoff.  So, is price at $t-1$ of 
$$\mye_t[Z_{t+1}X_{t+1}]$$
\pause 
\im So price at $t-1$ is
$$\mye_{t-1}\bigg[Z_t\mye_t[Z_{t+1}X_{t+1}]\bigg] = \mye_{t-1}\bigg[\mye_t[Z_tZ_{t+1}X_{t+1}]\bigg] = \mye_{t-1}\bigg[Z_tZ_{t+1}X_{t+1}\bigg]$$
\im We're compounding discount factors. 
\im With no uncertainty, price is 
$$\frac{X_{t+1}}{(1+r_{f,t})(1+r_{f,t+1})}$$
\ei
\end{frame}

\begin{frame}{SDF Process}
    \bi 
    \im Define $M$ by compounding discount factorrs:
    $$M_{t} := Z_1 \times Z_2 \times \cdots \times Z_{t}$$
    \im Set $M_0=1$.
    \im Price at date 0 of a payoff $X_t$ at date $t$ is $\mye[M_tX_t]$.
    \pause 
    \im Price at date $s<t$ of payoff $X_t$ at date $t$ is
    $$\mye_s[Z_{s+1}\cdots Z_tX_t] = \mye_s \left[\frac{Z_1 \cdots Z_t}{Z_1\cdots Z_s}X_t\right] = \mye_s \left[\frac{M_t}{M_s}X_t\right]$$
    \ei
\end{frame}

\section{Factor Model}\subsection{}

\begin{frame}{Dynamic Factor Model}
\bi 
\im From
$$ 1 = \mye_t\left[\frac{M_{t+1}}{M_t}R_{i,t+1}\right]$$
we get
$$ 1 = \frac{\mye_t[R_{i,t+1}]}{R_{f,t+1}} + \cov_t \left(\frac{M_{t+1}}{M_t},R_{i,t+1}\right)$$
\im So 
$$\mye_t[R_{i,t+1}] - R_{f,t+1} = - R_{f,t+1}\cov_t \left(\frac{M_{t+1}}{M_t},R_{i,t+1}\right)$$
\ei 
\end{frame}


\section{Portfolio Choice}\subsection{}

\begin{frame}{Portfolio Choice}
\bi \im 
 Stack returns into an $n$--vector $R_{t+1}$.  One may be risk-free (return $\,= R_{f,t+1}$).

 \im Investor chooses consumption $C_t$ and a portfolio $\pi_t \in \myreal^n$. $\iota'\pi_t=1$.  Labor income $Y_t$.

\im Suppose investor seeks to maximize
$$\sum_{t=0}^\infty \delta^t u(C_t)$$
 Wealth (actually financial wealth) $W$ satisfies the \alert{intertemporal budget constraint}
$$W_{t+1} = (W_t-C_t)\pi_t'R_{t+1} + Y_{t+1}$$
\ei
 \end{frame}



\bfr\frametitle{Euler Equation}
\bi \im 
A necessary condition for consumption/investment optimality is that, for all dates~$t$ and assets~$i$,
$$\mye_t\left[\frac{\delta u'(C_{t+1})}{u'(C_t)}R_{i,t+1}\right] = 1$$
 \im This is called the Euler equation.  It is derived by the same logic as in a single-period model.

 \im The Euler equation is equivalent to: 
$$M_t := \frac{\delta^t u'(C_t)}{u'(C_0)}$$
is an SDF process.
\im The one-period SDFs are one-period marginal rates of substitution:
$$\frac{M_{t+1}}{M_t} = \frac{\delta u'(C_{t+1})}{u'(C_t)}$$
\ei 
\end{frame}




\section{Equity Premium Puzzle}
\subsection{}

\bfr\frametitle{Representative Investor and SDF Process}
\bi \im 
Let $C$ denote aggregate consumption.    \im Assume there is a representative investor with CRRA utility and risk aversion $\rho$.  \im Then, the one-period SDF is
$$\frac{M_{t+1}}{M_t} = \delta \left(\frac{C_{t+1}}{C_t}\right)^{-\rho}$$

\im The SDF process is
$$M_t = \delta^t \left(\frac{C_{t}}{C_0}\right)^{-\rho}$$
\ei 
\end{frame}

\begin{frame}{Market Price-Dividend Ratio}
\bi 
\im Define the market portfolio as the claim to future consumption.  
\im Consumption is then the dividend of the market portfolio.  Assume consumption growth $C_{t+1}/C_t$ is iid lognormal.
\im 
The ex-dividend date--$t$ price of the market portfolio is
$$
   P_t := \mye_t \sum_{u=t+1}^\infty \frac{M_u}{M_t}C_u  = \mye_t \sum_{u=t+1}^\infty \delta^{u-t}\left(\frac{C_u}{C_t}\right)^{-\rho}C_u$$
\im So, the price-dividend ratio is
\begin{align*}
    \frac{P_t}{C_t} & = \mye_t \sum_{u=t+1}^\infty \delta^{u-t}\left(\frac{C_u}{C_t}\right)^{1-\rho}\\
    & = \mye \sum_{u=1}^\infty \delta^{u}\left(\frac{C_u}{C_0}\right)^{1-\rho}
\end{align*}
\ei 
\end{frame}

\begin{frame}[plain]
\bi 
\im Assume $\log C_{t+1} = \log C_t + \mu + \sigma \varepsilon_{t+1}$ for iid standard normals $\varepsilon$.  

\im Then 
$$\log C_u = \log C_0 + u\mu + \sigma \sum_{n=1}^u \varepsilon_n$$
\im Hence,
\begin{align*}
    \mye \left[\left(\frac{C_u}{C_0}\right)^{1-\rho}\right] &= \mye \left[\exp \left((1-\rho)\left\{u\mu + \sigma \sum_{n=1}^u \varepsilon_n\right\}\right)\right]\\
 &=\exp \left((1-\rho)u\mu + \frac{1}{2}(1-\rho)^2u\sigma^2 \right) \\
 &= \left(\E^{(1-\rho)\mu + (1-\rho)^2\sigma^2/2} \right)^u 
\end{align*}
\ei 
\end{frame}

\begin{frame}[plain]
\bi 
\im 
So, the price-dividend ratio is
$$ \sum_{u=1}^\infty \left(\delta\E^{(1-\rho)\mu + (1-\rho)^2\sigma^2/2} \right)^u = \frac{\nu_1}{1-\nu_1}$$
where
$$\nu_1 = \delta \mye\left[\left(\frac{C_1}{C_0}\right)^{1-\rho}\right] = \delta\E^{(1-\rho)\mu + (1-\rho)^2\sigma^2/2}$$
provided $\nu_1<1$.  
\im 
This is the same $\nu_1$ we saw in Chapter 7.  
\im Everything else---risk-free return, expected market return, log equity premium, equity premium puzzle---is exactly the same as in Chapter 7.
\ei 
\end{frame}

\section{Risk-Neutral Probability}\subsection{}
\begin{frame}{Risk-Neutral Probability}
\bi \im Consider an arbitrary finite (possibly large) horizon $T$.  
\im Consider an event $A$ that can be distinguished by date $T$ (at date $T$, you know whether $A$ happened or not).
\im Define 
    $$Q(A) = \mye[R_{f1}\cdots R_{fT}M_T1_A]$$ 
\im Then $Q$ is a probability measure.
\im Define $\mye^*$ as expectation with respect to $Q$.  Then for all assets $i$ and dates $t$,
    $$\mye^*_t[R_{i,t+1}] = R_{f,t+1}$$
        \im And, the price at $t$ of a payoff $X_{t+1}$ at date $t+1$ is
    $$\frac{\mye^*_t[X_{t+1}]}{1+r_{f,t+1}}$$
    \ei 
\end{frame}

\section{Martingales}\subsection{}

\begin{frame}{Martingales}
    \bi 
    \im A martingale is a sequence of random variables $Y$ such that $Y_s = \mye_s[Y_t]$ for all $s<t$.
    \im Equivalently, $E_s[Y_t-Y_s] = 0$.
    \im Consider any payoff at date $u$ with value $V_t$ at date $t$.  Then
    \begin{enumerate}
        \im The sequence $M_tV_t$ is a martingale (up to $u$).
        \im The sequence
        $$\frac{V_t}{(1+r_{f1}) \cdots (1+r_{ft})}$$
        is a $Q$-martingale.
    \end{enumerate}
    \ei
 \end{frame}

\section{Testing}\subsection{}
\begin{frame}{Testing Conditional Models}
Suppose we have a model for an SDF.  Call the model value $\hat M$.  We want to test whether
\begin{equation}\label{0}\tag{$\star$}
(\forall\;t, i) \qquad \mye_t[\hat M_{t+1}R_{i,t+1}] = 1
\end{equation}
Let $I_t$ be any variable observed at $t$.  Multiply both sides by $I_t$ and rearrange as:
$$(\forall\;t, i) \qquad \mye_t[I_t\hat M_{t+1}R_{i,t+1}] -I_t =0$$
Take the expectation and use the law of iterated expectations to obtain
\begin{equation}\label{1}\tag{$\star\star$}
(\forall\;t, i) \qquad \mye[I_t(\hat M_{t+1}R_{i,t+1} -1)] =0
\end{equation}
The conditional model \eqref{0} implies the unconditional moment condition \eqref{1} for every \alert{instrument} $I$.  If we reject the unconditional moment conditions, then we reject the model. 
\end{frame}
\end{document}


\section{Rare Disasters}
\subsection{}

\bfr\frametitle{Rare Disasters}
\bi
\im Assume a representative investor with CRRA utility.  Assume IID consumption growth.
\im From Chapter 10, the market return is
$$R_{m,t+1} = \frac{1}{\nu}\left(\frac{C_{t+1}}{C_t}\right)$$
where
$$\nu = \delta \mye\left[\left(\frac{C_{t+1}}{C_t}\right)^{1-\rho}\right]$$
\im The risk-free return is
$$R_f = \frac{1}{\delta \mye\left[\left(\frac{C_{t+1}}{C_t}\right)^{-\rho}\right]}$$
\im We will not assume lognormal consumption growth.  Instead, we assume a fatter left tail (rare disasters).  The goal is to see what this implies for the equity premium and risk-free rate.
\ei
\end{frame}

\bfr\frametitle{Consumption Model}
\bi
\im Assume 
$$
\Delta \log C_{t+1} = \mu + \sigma \xi_{t+1} + \chi_{t+1}\log (1-b_{t+1})\,,
$$
where~$\xi$,~$\chi$, and~$b$ are independent sequences of IID random variables, with each $\xi_t$ having the standard normal distribution, each $\chi_t$ having a Bernoulli distribution
$$\chi_{t} = \begin{cases} 0 & \text{with probability $1-p$}\\
1 & \text{with probability~$p$}
\end{cases}$$
for some fixed $0<p<1$, and each $b_t$ being distributed on the interval $(0,1)$.  Barro (QJE, 2006) takes $p=2$\%.
\im When $\chi_{t+1}=1$,  
$$\frac{C_{t+1}}{C_t} = \E^{\mu + \sigma \xi_{t+1}}(1-b_{t+1})\,.$$
Thus, $b_{t+1}$ is the fraction by which consumption falls when $\chi_{t+1}=1$ (which occurs with probability $p$).  
\ei
\end{frame}

\bfr\frametitle{Calculations}
\bi
\im To calculate the expected market return and risk-free return, we have to calculate
$$\mye\left[\left(\frac{C_{t+1}}{C_t}\right)^{\gamma}\right]$$
for $\gamma = 1$, $\gamma=1-\rho$, and $\gamma=-\rho$.
\im For the expected market return, $\gamma=1$ is in the numerator and $\gamma=1-\rho$ is in the denominator.
\im For the risk-free return $\gamma=-\rho$ is in the denominator.
\ei
\end{frame}

\bfr\frametitle{Calculations cont.}
\bi
\im By the independence of the normal shock $\xi$ from the indicator $\chi$ and loss rate $b$,
\begin{align*}
\mye\left[\left(\frac{C_{t+1}}{C_t}\right)^\gamma\right] &=\mye\left[\E^{\gamma \Delta \log C_{t+1}}\right]\\
&=\mye\left[\E^{\gamma \mu + \gamma \sigma \xi_{t+1}}\right]\mye\left[\E^{\gamma\chi_{t+1}\log(1-b_{t+1})}\right]
\end{align*}
\im The first factor is $\E^{\gamma \mu + \gamma^2\sigma^2/2}$.  The second factor is
$$
(1-p)\E^0+p\mye\left[\E^{\gamma \log (1-b_{t+1})}\right]=1-p+p\mye\left[(1-b_{t+1})^\gamma\right]\,.
$$\ei
\end{frame}

\bfr\frametitle{Expected Market Return and Risk-Free Return}
\bi
\im The first factor in the formula for $\mye\left[\left(C_{t+1}/C_t\right)^{\gamma}\right]$ is the same as in a lognormal model.  
\im Consequently, the expected market return is the same as in a lognormal model except for an additional factor
$$
 \frac{1-p+p\mye\left[1-b_{t+1}\right]}{1-p+p\mye\left[(1-b_{t+1})^{1-\rho}\right]}
 $$
 \im Also, the risk-free return is the same as in a lognormal model except for an additional factor
 $$
 \frac{1}{1-p+p\mye\left[(1-b_{t+1})^{-\rho}\right]}
 $$
 \ei
\end{frame}

\bfr\frametitle{Expected Market Return and Risk-Free Return cont.}
\bi
\im Both factors have numerators equal to 1 or smaller and denominators larger than 1 (when $\rho>1$), so both factors are less than 1.
\im Consequently, the expected market return and risk-free return are lower in the rare disasters model than in the lognormal model.
\im Usually, the risk-free return shrinks more, so the equity premium is higher and the risk-free return is lower than in a lognormal model.
 \im The left tail of the distribution of $b$ is important and controversial.   
 \ei
\end{frame}

\end{document}

\bfr\frametitle{Exercises}
Graded: 8.1, 11.2

Recommended: 8.2
\end{frame}
\end{document}



\bfr\frametitle{Ruling Out Ponzi Schemes}
 An investor with an infinite horizon will borrow and continually roll over the debt if possible, meaning $W_T \rightarrow -\infty$.

 In fact, the investor would take $W$ to $-\infty$ immediately.  To rule this out, we need constraints in addition to the intertemporal budget constraint.  

 We could require one of the following:
\begin{enumerate}
\im No borrowing: $C_t \leq W_t$
\im Negligible debt at infinity: $\lim_{T \rightarrow} \mye [M_TW_T] \geq 0$
\im $W$ bounded below.  If $\lim_{T \rightarrow} \mye [M_T] = 0$, then this implies (2).
\end{enumerate}
\end{frame}



\section{Gordon Growth Model}
\subsection{}

\bfr\frametitle{Gordon Growth Model}
\bi
\im Asset dividend process $D$.  We want to find a formula for its price $P$.  
\im Assume there is a strictly positive SDF process $M$ (and no bubble), so
\begin{align}
P_t &= \sum_{u=t+1}^\infty \mye_t \left[ \frac{M_u}{M_t}D_u\right]\notag\\
& = D_t\sum_{u=t+1}^\infty \mye_t \left[ \left(\frac{M_u}{M_t}\right)\left( \frac{D_u}{D_t}\right)\right] \notag\\
&= D_t \sum_{u=t+1}^\infty \mye_t \left[ \prod_{i=t}^{u-1}\left(\frac{M_{i+1}}{M_i}\right)\left(\frac{D_{i+1}}{D_i}\right)\right]\,. \label{1}\tag{$\star$}
\end{align}
\im The last line uses, for example,
$$\frac{M_u}{M_t} = \frac{M_{t+1}}{M_t}\cdot\frac{M_{t+2}}{M_{t+1}}\cdots \frac{M_u}{M_{u-1}}$$

\ei
\end{frame}

\bfr\frametitle{Gordon Growth Model cont.}
\bi
\im Assume the pair $(M_{t+1}/M_t, D_{t+1}/D_t)$ is an IID vector process.  This gives us a simple formula for $P/D$.
\im Set
$$
\nu = \mye\left[\left(\frac{M_{i+1}}{M_i}\right)\left(\frac{D_{i+1}}{D_i}\right)\right]\,.
$$
Assume $\nu<1$.  
\im By the IID hypothesis, the expected product in \eqref{1} is the product of the expectations, and each expectation equals~$\nu$, so
$$
\frac{P_t}{D_t} = \sum_{u=t+1}^\infty \prod_{i=t}^{u-1} \mye\left[\left(\frac{M_{i+1}}{M_i}\right)\left(\frac{D_{i+1}}{D_i}\right)\right]= \sum_{u=t+1}^\infty \nu^{u-t} = \frac{\nu}{1-\nu}\,.
$$ 
\ei
\end{frame}

\bfr\frametitle{Gordon Growth Model cont.}
\bi
\im The asset return is proportional to dividend growth.  Because $P/D = \nu/(1-\nu)$, the asset return is
\begin{align*}
R_{t+1} = \frac{D_{t+1} + P_{t+1}}{P_{t}} &=  \frac{1+P_{t+1}/D_{t+1}}{P_t/D_t}\left(\frac{D_{t+1}}{D_t}\right)\notag\\
&=\frac{1}{\nu}\left(\frac{D_{t+1}}{D_t}\right)\,.
\end{align*}
\im We can use this to solve for $\nu$ in terms of observables: Take expectations on both sides and rearrange as
$$\nu = \frac{\mye[D_{t+1}/D_t]}{\mye[R_{t+1}]}$$
\im This gives the Gordon formula for the price-dividend ratio:
\bq\label{2}\tag{$\star\star$}
\frac{P_t}{D_t} = \frac{\mye[D_{t+1}/D_t]}{\mye[R_{t+1}] - \mye[D_{t+1}/D_t]}\,.
\eq
\ei
\end{frame}

\end{document}