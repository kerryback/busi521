\documentclass[11pt]{article}
\RequirePackage{natbib,amsmath,amsthm,array,graphicx,footmisc,amsfonts,geometry,fancyvrb,chngcntr,minitoc}
\raggedbottom
\raggedright
\setlength{\parindent}{5ex}

\newcommand{\hiddensection}[1]{
	\stepcounter{section}
	\section*{\arabic{chapter}.\arabic{section}\hspace{1em}{#1}}
}

	
\counterwithin{table}{chapter}

\newenvironment{mypetit}{\centerline{\rule[0.2\baselineskip]{1in}{0.15mm}}\noindent\small}{}
\newenvironment{mypetitenum}{\centerline{\rule[0.2\baselineskip]{1in}{0.15mm}}\noindent\small}{}

\def\next{\vskip \baselineskip\noindent}
\newcommand{\mybox}[1]{\next\fbox{\parbox{4.55in}{#1}}\next}
\newcommand{\listtype[1]}{\renewcommand{\labelenumi}{#1}}
\newcommand{\vc}{^{\text{vec}}}
\newcommand{\mv}{_{\text{m}}}
\newcommand{\zb}{_{\text{z}}}
\newcommand{\cm}{_{\text{c}}}
\newcommand{\vct}{^{\text{vec}}}
\newcommand{\wt}[1]{\widetilde{#1}}
\newcommand{\tm}{\ensuremath{ t\!-\!1} }
\newcommand{\Tm}{\ensuremath{ T\!-\!1} }
\newcommand{\pr}{\ensuremath{\mathbb{P}}}
\newcommand{\qr}{\ensuremath{\mathbb{Q}}}
\newcommand{\mye}{\ensuremath{\mathsf{E}}}
\newcommand{\sr}{\ensuremath{\mathbb{S}}}
\newcommand{\fr}{\ensuremath{\mathbb{F}}}
\newcommand{\price}{\ensuremath{\mathcal{P}}}
\newcommand{\myreal}{\ensuremath{\mathbb{R}}}
\newcommand{\excise}[1]{\vskip 0.5\baselineskip \textit{#1} \vskip 0.5\baselineskip}
\newcommand{\CE}{\xi}
\newcommand{\sectspace}{\;\;\;\;}
\newcommand{\D}{\mathrm{d}}
\newcommand{\E}{\mathrm{e}}
\newcommand{\eqdef}{\;\buildrel \text{d{}ef}\over =\;}
\newcommand{\eqquest}{\;\buildrel \text{?}\over =\;}
\newcommand{\halfskip}{\vskip 0.5\baselineskip\noindent}
\newcommand{\onevector}{\iota}
\newcommand{\Rvector}{\mathbf{R}}
\newcommand{\sol}{\textbf{Solution:} \hspace{2ex}}
\newcommand{\be}{\begin{enumerate}\renewcommand{\labelenumi}{(\alph{enumi})}}
\newcommand{\ee}{\end{enumerate}}
\newcommand{\bq}{\begin{equation}}
\newcommand{\eq}{\end{equation}}

\theoremstyle{definition}
\newtheorem{prob}{}[chapter]

\newcommand{\teff}{\tau_{\text{eff}}}

\DeclareMathOperator{\var}{var} \DeclareMathOperator{\stdev}{stdev}
\DeclareMathOperator{\cov}{cov} \DeclareMathOperator{\corr}{corr}
\DeclareMathOperator{\M}{M} \DeclareMathOperator{\N}{N}
\DeclareMathOperator{\nd}{n} \DeclareMathOperator{\Cov}{Cov}
\DeclareMathOperator{\Prob}{prob} \DeclareMathOperator{\Var}{Var}
\DeclareMathOperator{\argmax}{argmax}
\DeclareMathOperator{\proj}{proj}
\DeclareMathOperator{\sign}{sign}

\newcommand{\ptext}[1]{\Prob(\text{#1})}

\geometry{verbose,letterpaper,tmargin=1in,bmargin=1.25in,lmargin=1in,rmargin=1in,headheight=0.2in,footskip=0.5in}
\renewcommand{\baselinestretch}{2}
\setlength{\headsep}{2\baselineskip}
\setlength{\footnotesep}{\baselineskip}

\newcommand{\bi}{\begin{itemize}}
\newcommand{\ei}{\end{itemize}}
\newcommand{\im}{\item}

\newcommand{\notes}[1]{
\addtocontents{toc}{\setcounter{tocdepth}{0}}
\addtocontents{minitoc}{\setcounter{tocdepth}{0}}
\section{\sectspace Notes and References}\label{#1}
\addtocontents{toc}{\protect\setcounter{tocdepth}{2}}
\addtocontents{minitoc}{\protect\setcounter{tocdepth}{2}}
}


\setlength{\headsep}{2\baselineskip}

\begin{document}
\VerbatimFootnotes
\documentclass[english,12pt]{amsart}
\RequirePackage{geometry,amsmath,graphicx,babel}

%%%%%%%%%%%%%%%%%%%%%%%%%%%%%%%%%%%%%%%%%%%%%%%%%%%%%%%%%%%%%%%%%%%%%%%%%%%%%%%%%%%%%%%%%%

\geometry{verbose,letterpaper,tmargin=1in,bmargin=1in,lmargin=1.125in,rmargin=1.125in,headheight=0.5in,footskip=0.5in}
\setlength{\parskip}{\bigskipamount}
\setlength{\parindent}{0pt}
\newcommand{\head}[1]{\vskip 0.5\baselineskip\underline{\textbf{#1}}\vskip 0.25\baselineskip}
\newcommand{\tophead}[1]{\underline{\textbf{#1}}\vskip 0.5\baselineskip}
\usepackage{hyperref}

\begin{document}
\begin{minipage}[t]{0.5\textwidth}
\vspace*{-1in}
BUSI 521: Asset Pricing Theory \\ ECON 505: Financial Economics I\\Spring 2022
\end{minipage}%
\begin{minipage}[t]{0.5\textwidth}
\raggedleft
%\vspace*{-0.5in}
\includegraphics[scale=1]{rice.jpg}
\end{minipage}%
\begin{center}
\vspace*{-0.65in}
\rule{\textwidth}{0.4pt}
\end{center}

As usual, collaboration on this assignment is not permitted.
\begin{enumerate}
\item What is the equilibrium in the Grossman-Stiglitz model when informed investors are risk averse and uninformed investor are risk neutral?
\item 
\item Do Exercise 8.1, parts (b)--(e).  The answer to part (a) is as follows.  Under the given assumptions, the return on the risky asset from $t$ to $t+1$ is $R_h := (k+1)\lambda_h/k$ when $D_{t+1}/D_t=\lambda_h$ and the return is $R_\ell := (k+1)\lambda_\ell/k$ when $D_{t+1}/D_t=\lambda_\ell$.  The necessary and sufficient condition for the absence of arbitrage opportunities is $R_h > R_f > R_\ell$.
\end{enumerate}

\end{document}