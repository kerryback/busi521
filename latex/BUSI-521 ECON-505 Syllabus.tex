\documentclass[english,12pt]{amsart}
\RequirePackage{geometry,amsmath,graphicx,babel}

%%%%%%%%%%%%%%%%%%%%%%%%%%%%%%%%%%%%%%%%%%%%%%%%%%%%%%%%%%%%%%%%%%%%%%%%%%%%%%%%%%%%%%%%%%

\geometry{verbose,letterpaper,tmargin=1in,bmargin=1in,lmargin=1.125in,rmargin=1.125in,headheight=0.5in,footskip=0.5in}
\setlength{\parskip}{\bigskipamount}
\setlength{\parindent}{0pt}
\newcommand{\head}[1]{\vskip 0.5\baselineskip\underline{\textbf{#1}}\vskip 0.25\baselineskip}
\newcommand{\tophead}[1]{\underline{\textbf{#1}}\vskip 0.5\baselineskip}
\usepackage{hyperref}

\begin{document}
\begin{minipage}[t]{0.5\textwidth}
\vspace*{-1in}
BUSI 521: Asset Pricing Theory \\ ECON 505: Financial Economics I\\Spring 2022
\end{minipage}%
\begin{minipage}[t]{0.5\textwidth}
\raggedleft
%\vspace*{-0.5in}
\includegraphics[scale=1]{rice.jpg}
\end{minipage}%
\begin{center}
\vspace*{-0.65in}
\rule{\textwidth}{0.4pt}
\end{center}


%%%%%%%%%%%%%%%%%%%%%%%%%%%%%%%%%%%%%%%%%%%%%%%%%%%%%%%%%%%%%%%%%%%%%%%%%%%%


\head{Meeting Details}
 MW 2:00-3:15\\
 Room 317, McNair Hall\\
\url{https://jgsb.zoom.us/j/96260714073?pwd=N3lvREVjY0tKUlhuSlg5SmVsVmhMZz09}

\head{Instructor}
Kerry Back\\
J. Howard Creekmore Professor of Finance and Professor of Economics\\
Room 325, McNair Hall\\
\verb!kerryback@gmail.com!

\head{Textbook}
 \textit{Asset Pricing and Portfolio Choice Theory,} Oxford University Press, 2nd Edition, 2017. 

\head{Course Description}

This course is an introduction to asset pricing and portfolio choice theory with a little on empirics.  This is the foundation for the `investments' branch of finance.  Understanding how assets are priced is also important for issuing entities, like corporations, so asset pricing is also part of the foundation for corporate finance.  Of course, prices are determined by supply and demand.  We take supply (a topic in corporate finance) as given in this course and study demand (portfolio choice).  Mostly, we'll work in the neoclassical framework, but we'll also touch on models of asymmetric information and heterogeneous beliefs.

Uncertainty and time are the two key elements of portfolio choice.  We will start with single-period models and then move to dynamic models in both discrete and continuous time.   We'll develop the theory of dynamic programming in continuous time and use it to study portfolio choice and some corporate investment decisions.  Dynamic programming and other aspects of the mathematics of uncertainty in continuous time are useful in other areas of economics and finance as well.

\newpage
\head{Schedule}

\begin{enumerate}\renewcommand{\labelenumi}{\arabic{enumi}.}
\item Utility and risk aversion
\item Portfolio choice
\item Stochastic discount factors 
\item More stochastic discount factors
\item Mean-variance analysis 
\item More mean-variance analysis 
\item Factor models
\item More factor models
\item Extracting factors empirically
\item Testing factor models
\item General equilibrium
\item Representative investors and the equity premium puzzle
\item MIDTERM EXAM (February 23)
\item Rational expectations and informational efficiency
\item Heterogeneous beliefs
\item Dynamic models
\item Futures
\item Brownian motion and stochastic calculus
\item Stochastic discount factors in continuous-time models
\item Dynamic programming in continuous time
\item Portfolio choice and ICAPM
\item Options
\item Fundamental PDE
\item Real options
\item REVIEW
\end{enumerate}

\head{Grading}
Grades will be based 20\% on biweekly individual homework assignments, 30\% on the midterm exam, and 50\% on the final exam.  


\head{Honor Code}
The Rice University Honor Code applies to all work in this course. The intent of the Honor Code in general and specifically in this course is to ensure that each student claims and receives credit for their own efforts. The intent is not to limit the valuable exchange of ideas through discussion among fellow students. The atmosphere at Rice University must be one of academic and personal integrity.  Any suspected violations of the Honor Code are submitted to the Rice University Honor Council.


\head{Disability Accommodations}
Any student with a disability requiring accommodations in this class should contact me after contacting the Disabled Student Services office. 

\end{document}
\end{document}