
\mode<presentation> {\usetheme{Singapore}}
\usepackage{pgfpages}

%\setbeamercovered{transparent} 
\usepackage[english]{babel}
\usepackage[latin1]{inputenc}
\usepackage{times,amsfonts}
\usepackage[T1]{fontenc}
\setlength{\parskip}{\baselineskip}
% Or whatever. Note that the encoding and the font should match. If T1
% does not look nice, try deleting the line with the fontenc.

%\usecolortheme{sidebartab}
\setbeamertemplate{itemize item}[triangle]



\usepackage{calc}
\usepackage{environ}
\newcommand{\halfmargin}{0.0001\paperwidth}


\RequirePackage{booktabs,colortbl,ulem}

\usepackage{animate}
\RequirePackage{booktabs,colortbl,gensymb}
\setlength{\parskip}{\baselineskip}

\usepackage{calc}
\usepackage{environ}

% \newcommand{\halfmargin}{0.0001\paperwidth}


\NewEnviron{wideframe}[1][]{%
\begin{frame}{#1}
\makebox[\textwidth][c]{
\begin{minipage}{\dimexpr\paperwidth-\halfmargin-\halfmargin\relax}
\BODY
\end{minipage}}
\end{frame}
}


\DeclareMathOperator{\stdev}{stdev}
\DeclareMathOperator{\var}{var}
\DeclareMathOperator{\cov}{cov}
\DeclareMathOperator{\corr}{corr}
\DeclareMathOperator{\prob}{prob}
\DeclareMathOperator{\n}{n}
\DeclareMathOperator{\N}{N}
\DeclareMathOperator{\Cov}{Cov}

\newcommand{\hlf}{\frac{1}{2}}
\newcommand{\bi}{\begin{itemize}}
\newcommand{\ei}{\end{itemize}}
\newcommand{\im}{\item}
\newcommand{\D}{\mathrm{d}}
\newcommand{\E}{\mathrm{e}}
\newcommand{\mye}{\ensuremath{\mathsf{E}}}
\newcommand{\myreal}{\ensuremath{\mathbb{R}}}
\newcommand{\bq}{\begin{equation}}
\newcommand{\eq}{\end{equation}}
\newcommand{\eqdef}{\;\buildrel \text{d{}ef}\over = \;}
\newcommand{\xstar}{\buildrel *\over X}
\newcommand{\pmax}{p^{\text{max}}}
\newcommand{\qmax}{q^{\text{max}}}
\newcommand{\bfr}{\begin{frame}}
\newcommand{\bfrp}{\begin{frame}[plain]}
\newcommand{\efr}{\end{frame}}
\newcommand{\F}{\mathcal{F}}
\newcommand{\FF}{\mathbb{F}}
\newcommand{\ve}{\varepsilon}
\newcommand{\lh}{\hat{\lambda}}
\definecolor{mycolor}{gray}{0.8}
\definecolor{mymaincolor}{rgb}{0.6862745098039216,0.9333333333333333,0.9333333333333333}
\newcommand{\alr}[1]{\textcolor{blue}{#1}}
\definecolor{LightCyan}{rgb}{0.88,1,1}
\newcommand{\yel}{\cellcolor{yellow}}
\newcommand{\blue}{\cellcolor{SkyBlue}}
\newcommand{\gr}{\cellcolor{SpringGreen}}
\newcommand{\pink}{\cellcolor{pink}}
\newcommand{\apr}{\cellcolor{Apricot}}
\newcommand{\tve}{\tilde{\varepsilon}}
\newcommand{\tw}{\tilde{w}}
\newcommand{\ttth}{\tilde{\theta}}
\newcommand{\te}{\tilde{e}}
\newcommand{\ts}{\tilde{s}}
\newcommand{\tx}{\tilde{x}}
\newcommand{\ty}{\tilde{y}}
\newcommand{\tv}{\tilde{v}}
\newcommand{\tp}{\tilde{p}}
\newcommand{\tF}{\tilde{F}}
\newcommand{\tf}{\tilde{f}}
\newcommand{\tZ}{\tilde{Z}}
\newcommand{\ow}{\overline{w}}
\newcommand{\lb}{\left[}
\newcommand{\rb}{\right]}
\newcommand{\lp}{\left(}
\newcommand{\rp}{\right)}
\newcommand{\tm}{\tilde{m}}
\newcommand{\tc}{\tilde{c}}
\newcommand{\tz}{\tilde{z}}
\newcommand{\str}[1]{\textcolor{blue}{\sout{#1}}}
\newcommand{\tr}{\widetilde{R}}
\newcommand{\tR}{\widetilde{\mathbf{R}}}
\newcommand{\bms}{\begin{multline*}}
\newcommand{\ems}{\end{multline*}}
\newcommand{\bas}{\begin{align*}}
\newcommand{\eas}{\end{align*}}
\newcommand{\qr}{\mathbb{Q}}
\newcommand{\IMAGES}{/home/kerry/Dropbox/Images}
\newcommand{\tX}{\tilde{X}}
\newcommand{\tY}{\tilde{Y}}

\author{\vskip 0.5in \small Kerry Back \\BUSI 521--ECON 505\\ Rice University \\Spring 2022}
%\institute{Rice University\\ Spring 2019}
\date[]






\title{\vskip 0.5in Days 7--8}
\subtitle{Factor Models\\
}


\begin{document}

%%%%%%%%%%%%%%%%%%%%%%%%%%%%%%%%%%%%%%%%%%%%%%%%%%%%%%%%%%%%%%%%%%%%%%%

\begin{frame}[plain]
  \titlepage
\end{frame}

%%%%%%%%%%%%%%%%%%%%%%%%%%%%%%%%%%%%%%%%%%%%%%%%%%%%%%%%%%%%%%%%%%%%%%%


\section{Projections}\subsection{}

\begin{frame}{Best Linear Approximations of Random Variables}

Given two (finite-variance) random variables $\tx$ and $\ty$, there exist $\alpha$ and $\beta$ such that $\alpha + \beta \tx$ is the best linear (affine) approximation of $\ty$ in the sense of minimizing the mean squared error:
$$ \mye[(\ty - \alpha - \beta \tx)^2]\,.$$
The solution is $\beta = \cov(\tx,\ty)/\var(\tx)$ and 
$$\alpha = \mye[\ty] - \beta \mye[\tx]$$
Defining the error (residual) $\tve = \ty - \alpha - \beta \tx$ for the optimal $\alpha$ and $\beta$, we have  $\mye[\tve]= 0$ and $\cov(\tve,\tx)=0$.

An equivalent definition of $\alpha$ and $\beta$ is that they are the unique pair such that
$$\ty \ = \ \alpha + \beta \tx + \tve$$
with $\mye[\tve]= 0$ and $\cov(\tve,\tx)=0$.
\end{frame}


\begin{frame}{Orthogonal Projection}
The phrase ``best linear approximation'' has a vector-space interpretation.  Let $\mathcal{L}^2$ denote the space of finite-variance random variables.  Define an inner product by $\langle \tx,\ty \rangle = \mye[\tx\ty]$ and orthogonality by $\mye[\tx\ty] = 0$.  

Given a random variable $\tx \in \mathcal{L}^2$, let $\mathcal{M} = \{ a + b \tx \mid (a,b) \in \myreal^2\}$.  The set $\mathcal{M}$ is the subspace of $\mathcal{L}^2$ that is spanned by a constant and $\tx$.  

Given another random variable $\ty \in \mathcal{L}^2$, the best linear approximation $\alpha + \beta \ty$ defined on the previous slide is the orthogonal projection of $\ty$ onto $\mathcal{M}$.  It is the closest point in $\mathcal{M}$ to $\ty$ in the sense of the norm (consistent with the inner product) $\parallel \tz \parallel = \sqrt{\mye[\tz^2]}$.
\end{frame}

\begin{frame}{A Different Orthogonal Projection}
There always exists $\alpha$ and $\beta$ such that
$$\ty \ = \ \alpha + \beta \tx + \tve$$
with $\mye[\tve]= 0$ and $\cov(\tve,\tx)=0$.
However, it is not generally true that $\mye[\tve \mid \tx] = 0$.

Define 
$$\mathcal{N} = \{f(\tx) \mid f \text{\,is a function such that $f(\tx) \in \mathcal{L}^2$}\}$$  Then, $\mathcal{M} \subset \mathcal{N}$.  There also exists an orthogonal projection of $\ty$ onto $\mathcal{N}$.  This projection is the random variable $\mye[\ty \mid \tx]$.  If we define the error $\te = \ty - \mye[\ty \mid \tx]$, then we have $\mye[\te \mid \tx] = 0$.

If $\mye[\ty \mid \tx] \neq \alpha + \beta \tx$, then $\mye[\ty \mid \tx]$ is closer to $\ty$ in the sense of the norm defined on the previous slide.
\end{frame}


\begin{frame}{Projecting onto the Span of Multiple Variables}

Suppose we're given a vector $\tX = (\tx_1 \, \tx_2 \, \cdots \tx_n)'$ of finite-variance random variables.  The orthogonal projection of $\ty$ onto the span of a constant and the $\tx_i$ is
$$\alpha + \beta'\tx$$
where
$$\beta = \Cov(\tX)^{-1}\Cov(\tX,\ty)$$
and $\alpha = \mye[\ty] - \mye[\beta'\tX]$. 

Substituting for $\alpha$ and letting $\tve$ denote the residual, we have
$$\ty - \mye[\ty] \ = \ \beta'\big(\tX - \mye[\tX]\big) + \tve$$
where $\mye[\tve]=0$ and $\Cov(\tX,\tve)=0$.

Furthermore, $\beta'\tX$ is the linear combination of the $\tx_i$ that is most highly correlated with $\ty$.
\end{frame}

\begin{frame}{Projections that are Returns}
If we do an orthogonal projection onto the span of asset returns, we get a portfolio payoff but not necessarily a return (the cost of the portfolio may not equal 1).

If we want returns, we can instead project on the space of excess returns (the payoffs of zero-cost portfolios; equivalently, the linear span of the excess returns $\tr-R_f$).  

For every $\tz$ in the space of excess returns, $R_f + \tz$ is a return.
\end{frame}

\section{Factors}\subsection{}

\begin{frame}{Factor Models}

We say that there is a factor model for asset returns with factors $\tF = (\tf_1 \, \tf_2 \, \cdots \tf_k)'$ if the risk premium of each asset is determined by its betas with respect to the factors.

More precisely, we say there is a factor model if the risk premium is a linear combination of the betas, with the linear coefficients being the same for all assets.

In other words, there is a factor model if there exists $\lambda \in \myreal^k$ such that, for each asset,
$$\mye[\tr]-R_f \ = \ \lambda' \Cov(\tF)^{-1}\Cov(\tF,\tr)$$

This means that, for determining risk premia, we can measure risks as betas with respect to the factors.  We call the $\lambda_i$'s prices of risk.
\end{frame}

\bfr\frametitle{Risk Adjustments}
Start with 
$$\mye[\tr]-R_f \ = \ \lambda' \Cov(\tF)^{-1}\Cov(\tF,\tr)$$
and set $\tr = \tx/p$ and solve for $p$:
$$p \ = \ \frac{\mye[\tx] - \lambda'\Cov(\tF)^{-1}\Cov(\tF,\tx)}{R_f}$$

Or, leave $\tr$ in the covariance and solve for $p$ as
$$p \ = \ \frac{\mye[\tx]}{R_f + \lambda'\Cov(\tF)^{-1}\Cov(\tF,\tr)}$$


\end{frame}

\begin{frame}{SDFs}
An SDF is a factor: for all returns,
$$\mye[\tr]-R_f \ = \ -R_f\cov(\tm,\tr) \ = \ -R_f \var(\tm) \cdot \frac{\cov(\tm,\tr)}{\var(\tm)} \ = \ \lambda \cdot \beta$$

Factors are SDFs: if $\tF$ is a vector of factors, then there exists $a$ and $b$ such that $a+b'\tF$ is an SDF.
\end{frame}

\begin{frame}{Projections of Factors are Factors}

Given a vector $\tF$ of factors, project any or all of the factors onto the span of the excess returns and a constant.  We can replace each such factor $\tf_i$ with the excess return $\tz_i$ in its projection.  

Why?  Denote the projection of $\tf_i$ by $a_i + \tz_i$ with residual $\te_i$, where $\tz_i$ is an excess return.  The residual has zero mean and is orthogonal to each excess return, so it has zero covariance with each excess return.  Therefore, for each return $\tr$,  
\begin{align*}
\cov(\tr,\tf_i) \ = \ \cov(\tr,a+ \tz_i + \te_i) \ &= \ \cov(\tr,\tz_i) + \cov(\tr,\te_i)\\
&= \ \cov(\tr,\tz_i) + \cov(\tr-R_f,\te_i)\\
&   = \ \cov(\tr,\tz_i)
\end{align*}
The covariances are unchanged, so risk premia are linear in covariances.  Betas change, so the prices of risk $\lambda_i$ change.
\end{frame}

\begin{frame}{Frontier Returns}
A frontier return is a factor: $\pi = (1/\delta)\Sigma^{-1}(\mu - R_f\iota)$ implies that risk premia are proportional to covariances with the return of $\pi$.

Projections of factor are frontier returns: Project each factor onto the space of excess returns, so we have factors  that are excess returns.  A vector $\tZ$ of excess returns forms a factor model if and only if $R_f + \gamma'\tZ$ is a frontier return for some $\gamma \in \myreal^k$.  
\end{frame}

\begin{frame}{Prices of Risk}
We often try excess returns as factors empirically -- for example, the market excess return $\tr_m - R_f$.  Suppose there is a factor model in which one or more of the factors is an excess return.

The price of risk $\lambda_i$ for each excess-return factor is the mean of the excess return (will prove for CAPM; general proof is similar).

If a factor is a return, its price of risk is its risk premium (the mean of the excess return).

For non-return factors, the price of risk is a free parameter.
\end{frame}

\section{CAPM}\subsection{}

\begin{frame}{CAPM}
We have shown that risk premia are proportional to covariances with a frontier portfolio.  Suppose the market is a frontier portfolio and let $\tr_m$ denote the market return.  

Then, there exists $\gamma$ (the constant of proportionality) such that for all returns $\tr$,
$$\mye[\tr] - R_f \ = \ \gamma \cov(\tr_m,\tr) \ = \ \gamma \var(\tr_m,\tr)\frac{\cov(\tr_m,\tr)}{\var(\tr_m)} \ = \ \lambda \beta$$
To compute $\lambda$, apply this to $\tr=\tr_m$ to get
$$\mye[\tr_m]-R_f \ = \ \lambda \beta \ = \ \lambda$$
because the beta of the projection of $\tr_m$ on $\tr_m$ equals 1.  Thus, for all returns $\tr$,
$$\mye[\tr] - R_f \ = \ \frac{\cov(\tr_m,\tr)}{\var(\tr_m)}\big(\mye[\tr_m] - R_f \big)$$ 
\end{frame}

\begin{frame}{Risk Adjustment with the CAPM}
Under the CAPM, for an asset with date--1 value $\tx$ and date--0 price $p$, we have
$$p \ = \ \frac{\mye[\tx]}{R_f + \big(\mye[\tr_m] - R_f \big)\beta}$$
where
$$\beta \ = \ \frac{\cov(\tr_m,\tr)}{\var(\tr_m)}$$
In practice, estimate $\beta$ from company stock returns and the market return.  Apply this formula to value the cash flow $\tx$ from a new project. Invest if $p$ is greater than the cost of the project.
\end{frame}

\begin{frame}{CAPM, Marginal Utilities, and SDFs}
Suppose there is an investor who optimally holds the market portfolio, so date--1 wealth is $w\tr_m$.  There is an SDF proportional to the investor's marginal utility $u'(w\tr_m)$, so, for all assets,
$$\mye[\tr]-R_f \ = \ \gamma \cov(u'(w\tr_m),\tr)$$
for a constant $\gamma$ (that is the same for all assets).

How can we go from $\cov(u'(w\tr_m),\tr)$ to $\cov(\tr_m,\tr)$?
\pause
\begin{enumerate}
\im With normal returns, use Stein's lemma: 
$$\cov(u'(w\tr_m,\tr) \ = \ \mye[u''(w\tr_m)] w \cov(\tr_m,\tr)$$
\im With quadratic utility, marginal utility is affine in wealth: $u'(w\tr)m) = a + b w\tr_m$, so
$$\cov(u'(w\tr_m,\tr) \ = \ b w \cov(\tr_m,\tr)$$
\end{enumerate}
\end{frame}

\bfr\frametitle{Testing the CAPM}
\bi
\im Regress excess returns on the market excess return:
$$\tr-R_f = \alpha + \beta(\tr_m-R_f) + \tve$$
\im The regression implies
$$\mye[\tr] -R_f = \alpha + \beta \big(\mye[\tr_m]-R_f\big)$$
\im The CAPM states that $\alpha=0$, so we can test the null $\alpha=0$ for each return.  We can also test the joint null hypothesis $(\alpha_1, \ldots, \alpha_n)' = 0$ for a set of $n$ assets (Gibbons-Ross-Shanken test).
\im Usually we use portfolio returns rather than individual stock returns when testing the CAPM---diversification improves accuracy of $\beta$ estimates.
\im We can follow the same procedure to test any factor model in which the factors are excess returns.
\ei
\end{frame}

\section{Jensen's Alpha}\subsection{}

\begin{frame}{Seeking Alpha}

Suppose we conjecture a factor model with excess returns as factors
$$\mye[\tr] - R_f =  \Cov(\tX,\tr)'\Cov(\tX)^{-1}\mye[\tX]\,.$$
We test it by running regressions
$$\tr - R_f = \alpha + \beta'\tX + \tve$$
and testing if $\alpha=0$.

If a return or portfolio or manager has a positive $\alpha$, it has out-performed, relative to what the model predicts (seeking alpha).
\end{frame}

%%%%%%%%%%%%%%%%%%%%%%%%%%%%%%%%%%%%%%%%%%%%%%%%%%%%%%%%%

\begin{frame}{Jensen's Alpha}
A standard way to evaluate a portfolio manager is to compute alpha relative to a benchmark (large cap benchmark for large cap manager, small cap benchmark for small cap manager, bond benchmark for bond manager, \ldots).

Compared to holding the benchmark, you can improve mean-variance efficiency by adding some of a positive-alpha return.  

This is called Jensen's alpha.
\end{frame}

%%%%%%%%%%%%%%%%%%%%%%%%%%%%%%%%%%%%%%%%%%%%%%%%%%%%%%%%%

\begin{frame}{$\alpha + \varepsilon$ }

Let $\tr_b$ denote the benchmark return.  Run the regression
$$\tr - R_f \ = \ \alpha + \beta (\tr_b-R_f) + \tve$$

Consider the excess return
$$\tr - R_f -\beta (\tr_b-R_f) \ = \ tr - \beta \tr_b - (1-\beta)R_f$$
This equals
$$\alpha + \tve$$

So, you can get $\alpha$ at zero cost by taking on the residual risk $\tve$.  Is it worthwhile to take on this risk?

\end{frame}

%%%%%%%%%%%%%%%%%%%%%%%%%%%%%%%%%%%%%%%%%%%%%%%%%%%%%%%%%

\begin{frame}{Utility Improvements}
Consider an investor who holds the benchmark and consider a return with a positive alpha.

Will he get a utility improvement by adding some of $\alpha + \tve$?  

Suppose he changes to $\tr = \tr_b + \lambda(\alpha+\tve)$ for some (possibly small) $\lambda$.

We can do a Taylor series expansion to get
$$\mye[u(\tr)] \ \approx \ \mye[u(\tr_b)] + \lambda \mye[u'(\tr_b)(\alpha+\tve)]$$
What is the sign of $\mye[u'(\tr_b)(\alpha+\tve)]$?
\bi
\im If returns are normally distributed?
\im If utility is quadratic?
\ei
\end{frame}

%%%%%%%%%%%%%%%%%%%%%%%%%%%%%%%%%%%%%%%%%%%%%%%%%%%%%%%%%

\begin{frame}{Mean-Variance Improvements}
Let's add some of $\alpha+\tve$ to $\tr_b$ without changing the mean.

To keep the mean constant, we'll subtract some of $\tr_b-\tr_f$.  The new return is
$$\tr \ = \ \tr_b + \lambda(\alpha+\tve) - \frac{\lambda\alpha}{\mye[\tr_b]-R_f}(\tr_b-\tr_f)$$

The new variance is
$$\left(1-\frac{\lambda\alpha}{\mye[\tr_b]-R_f}\right)^2\var(\tr_b) + \lambda^2\var(\tve)$$
Is this larger or smaller than $\var(\tr_b)$?
\end{frame}

%%%%%%%%%%%%%%%%%%%%%%%%%%%%%%%%%%%%%%%%%%%%%%%%%%%%%%%%%

\begin{frame}{Factor Models and Jensen's Alpha}

Suppose we conjecture a factor model with excess returns as factors
$$\mye[\tr] - R_f =  \Cov(\tX,\tr)'\Cov(\tX)^{-1}\mye[\tX]\,.$$

Suppose we run the regression
$$\tr - R_f = \alpha + \beta'\tX + \tve$$
and decide that a return has a positive alpha.

Then adding some of the return will generate utility or mean-variance improvements for an investor who holds 
$$R_b \ = \ R_f + \beta'\tX$$

\end{frame}

%%%%%%%%%%%%%%%%%%%%%%%%%%%%%%%%%%%%%%%%%%%%%%%%%%%%%%%%%

\section{Fama-French}\subsection{}

\bfr{Fama-French-Carhart Model}
 Fama-French (1993): SMB and HML in addition to the market return.
\bi
\im SMB = Small Minus Big
\im HML = High book-to-market (value) Minus Low book-to-market (growth)
\ei

Carhart (1997): UMD = Up Minus Down

Fama-French (2015): RMW and CMA
\bi
\im RMW = Robust (high profits) Minus Weak (low profits)
\im CMA = Conservative (low growth) Minus Aggressive (high growth)
\ei

See Hou-Xue-Zhang (2015) also.

\end{frame}

\begin{frame}

Basic idea of CAPM: only market risk should earn a risk premium.  We measure market risk by the beta in the projection
$$\tr - R_f \ = \ \alpha + \beta(\tr_m-R_f) + \tve$$
The CAPM says the risk premium is
$$R_f + \beta\big(\mye[\tr_m]-R_f\big)$$
The CAPM says the projection residual $\tve$ doesn't earn a risk premium.

Other models (for example, Fama-French): systematic components of the  residual may earn risk premia, meaning correlations with SMB, HML, RMW, and CMA (and/or UMD, percent change in oil price, \ldots).
\end{frame}

\begin{frame}
Project excess returns of a stock on $\tr_m-R_f$ and the other FF factors (multiple regression).  Obtain $\beta_{\text{mkt}}$, $\beta_{\text{SMB}}$, $\beta_{\text{HML}}$, $\beta_{\text{RMW}}$, and $\beta_{\text{CMA}}$.

Fama-French model asserts risk premium of the stock should be
\begin{multline}
\mye[\tr_m]-R_f \ = \ \beta_{\text{mkt}}\big(\mye[\tr_m]-R_f\big) + \beta_{\text{SMB}}\mye[\text{SMB}] \\+ \beta_{\text{HML}}\mye[\text{HML}] + \beta_{\text{RMW}}\mye[\text{RMW}] + \beta_{\text{CMA}}\mye[\text{CMA}]
\end{multline}

\end{frame}

\begin{frame}{Construction of Characteristics}

Portfolios formed at end of June each year

Size = market cap at end of June

Momentum = compound return over first 11 of prior 12 months

Book-to-market, operating profitability, and asset growth all from prior calendar-year annual report (so from end of prior December if fiscal year ends in December and earlier otherwise)
\bi
\im 

Book-to-market = book equity divided by market cap at end of December.  Book equity = shareholders' equity + deferred taxes - preferred stock.

\im Operating profitability = (sales - cogs - sg\&a - interest) / (book equity + minority interests) at end of December

\im Asset growth = percent change in assets
\ei

\end{frame}


\begin{frame}{Size and Book-to-Market Sorted Portfolios}

Average monthly returns of value-weighted portfolios 1970-2020.  25 groups formed each month by intersecting 5 size groups and 5 book-to-market groups (based on NYSE quintiles).

\begin{center}
\begin{tabular}{lrrrrr}
\toprule
{} \\
bm &   BM1 &   BM2 &   BM3 &   BM4 &   BM5 \\
size &       &       &       &       &       \\
\midrule
ME1  &  0.59 &  1.17 &  1.12 &  1.32 &  1.37 \\
ME2  &  0.90 &  1.19 &  1.20 &  1.24 &  1.31 \\
ME3  &  0.97 &  1.19 &  1.12 &  1.22 &  1.36 \\
ME4  &  1.07 &  1.06 &  1.09 &  1.17 &  1.22 \\
ME5  &  0.96 &  0.99 &  1.01 &  0.88 &  1.05 \\
\bottomrule
\end{tabular}

\end{center}
\end{frame}

\begin{frame}{Size and Momentum Sorted Portfolios}

Average monthly returns of value-weighted portfolios 1970-2020.  25 groups formed each month by intersecting 5 size groups and 5 momentum groups (based on NYSE quintiles).

\begin{center}
\begin{tabular}{lrrrrr}
\toprule
{} \\
mom & PRIOR1 & PRIOR2 & PRIOR3 & PRIOR4 & PRIOR5 \\
size &        &        &        &        &        \\
\midrule
ME1  &   0.46 &   1.00 &   1.23 &   1.36 &   1.66 \\
ME2  &   0.60 &   1.07 &   1.19 &   1.34 &   1.54 \\
ME3  &   0.71 &   1.00 &   1.10 &   1.15 &   1.49 \\
ME4  &   0.64 &   1.03 &   1.10 &   1.19 &   1.39 \\
ME5  &   0.59 &   0.96 &   0.89 &   0.99 &   1.18 \\
\bottomrule
\end{tabular}

\end{center}
\end{frame}

\begin{frame}{Size and Investment Sorted Portfolios}

Average monthly returns of value-weighted portfolios 1970-2020.  25 groups formed each month by intersecting 5 size groups and 5 investment groups (based on NYSE quintiles).

\begin{center}
\begin{tabular}{lrrrrr}
\toprule
{} \\
inv &  INV1 &  INV2 &  INV3 &  INV4 &  INV5 \\
size &       &       &       &       &       \\
\midrule
ME1  &  1.30 &  1.30 &  1.32 &  1.16 &  0.67 \\
ME2  &  1.23 &  1.25 &  1.28 &  1.24 &  0.87 \\
ME3  &  1.26 &  1.26 &  1.16 &  1.16 &  0.96 \\
ME4  &  1.15 &  1.17 &  1.14 &  1.16 &  1.00 \\
ME5  &  1.18 &  1.02 &  0.96 &  0.96 &  0.89 \\
\bottomrule
\end{tabular}

\end{center}
\end{frame}

\begin{frame}{Size and Profitability Sorted Portfolios}

Average monthly returns of value-weighted portfolios 1970-2020.  25 groups formed each month by intersecting 5 size groups and 5 profitability groups (based on NYSE quintiles).

\begin{center}
\begin{tabular}{lrrrrr}
\toprule
{} \\
prof &   OP1 &   OP2 &   OP3 &   OP4 &   OP5 \\
size &       &       &       &       &       \\
\midrule
ME1  &  0.97 &  1.30 &  1.23 &  1.29 &  1.20 \\
ME2  &  1.03 &  1.15 &  1.19 &  1.18 &  1.33 \\
ME3  &  1.01 &  1.11 &  1.14 &  1.15 &  1.31 \\
ME4  &  1.00 &  1.07 &  1.09 &  1.12 &  1.21 \\
ME5  &  0.74 &  0.83 &  0.90 &  0.93 &  1.01 \\
\bottomrule
\end{tabular}

\end{center}
\end{frame}

\bfr{Construction of FF Factors}
\bi
\im All of the factors are excess returns (long minus short).
\im Split stocks into small and big (NYSE 50th percentile).
\im Split by the characteristic (book-to-market, momentum, profitability, or investment) as 30\%,40\%,30\%.  \im Intersect the sorts to form 6 groups.  HML, UMD, RMW, and CMA are all defined as:
$$\frac{1}{2}\text{TopBig} + \frac{1}{2}\text{TopSmall} - \frac{1}{2}\text{BottomBig} - \frac{1}{2}\text{BottomSmall}$$
\im SMB $\,=\frac{1}{2}\text{TopSmall} + \frac{1}{2}\text{BottomSmall} - \left(\frac{1}{2}\text{TopBig} - \frac{1}{2}\text{BottomBig}\right)$
\im There is an SMB for each sort (value/growth, profitability, asset growth).  Average the three SMBs.
\ei
\end{frame}

\section{APT}\subsection{}

\bfr\frametitle{Statistical Factor Models}
A factor pricing model explains \alert{risk premia} in terms of covariances or betas with respect to a set of factors.

 A statistical factor model explains \alert{covariances} in terms of covariances with respect to a set of factors.

 The residual in the projection of a return $\tr$ onto the span of a constant and $\tX = (\tx_1, \ldots, \tx_k)'$ is
\bq\tag{$\star$}
\tve \eqdef \tr - \left\{\overline{R} 
+ \Cov(\widetilde{X},\tr)'\Cov(\widetilde{X})^{-1} [\tX - \overline{X}]\right\}
\eq

 We say that returns satisfy a \alert{statistical factor model} with factors $\tX$ if the residuals of the returns are pairwise \alert{uncorrelated}.  The residuals are called idiosyncratic risks.
\end{frame}

\bfr\frametitle{}
 For any return $\tr_i$, let $\beta_i$ denote the column vector of multiple regression betas:
$$\beta_i \ = \ \Cov(\widetilde{X})^{-1}\Cov(\widetilde{X},\tr_i)$$

 Equation ($\star$) is $\tr_i = \overline{R}_i + \beta_i'[\tX - \overline{X}] + \tve_i$.

 Because the $\tve$'s are mutually uncorrelated (and orthogonal to the factors),
$$\cov(\tr_i,\tr_j) \ = \ \cov(\beta_i'[\tX - \overline{X}] + \tve_i,\beta_j'[\tX - \overline{X}] + \tve_j] = \beta_i'\Cov(X)\beta_j$$

 Thus, the covariance between any two returns depends only on their exposures to the common factors (their betas).  Those exposures are called systematic risks.

 Simplest model: market model.  Only factor is the market return.  All stocks tend to go up and down with the market.  Other risks are uncorrelated.

\end{frame}

\bfr\frametitle{Arbitrage Pricing Theory}
 The APT asserts that statistical factors are also pricing factors (approximately).

 Intuition: In a statistical factor model, only systematic risks should earn risk premia, because idiosyncratic risks can be diversified away.  Thus, risk premia should depend on betas.

\end{frame}

\bfr\frametitle{APT with No Idiosyncratic Risks}
 To get intuition about the APT, assume there are no idiosyncratic risks: $\tr_i = \overline{R}_i + \beta_i'[\tX - \overline{X}]$.  

 Assume there is an SDF (this is the arbitrage part of the APT).  Then
\begin{align*}
 1 = \mye[\tm\tr] \ &= \ \mye[\tm]\overline{R} + \beta'\mye[\tm(\tX - \overline{X})]\\
 &= \ \mye[\tm]\overline{R} + \beta'\Cov(\tX,\tm)
 \end{align*}
 
  So
 $$\mye[\tr] \ = \ \frac{1}{\mye[\tm]} - \frac{1}{\mye[\tm]}\beta'\Cov(\tX,\tm)$$
 
  Thus, the vector of prices of risk is
 $$\lambda \eqdef - \frac{1}{\mye[\tm]}\Cov(\tX,\tm)$$

\end{frame}

\bfr\frametitle{APT with Idiosyncratic Risks}
 Following the same algebra with $\tve \neq 0$ yields
$$\mye[\tr] \ = \ \frac{1}{\mye[\tm]} +\beta'\lambda - \frac{\mye[\tm\tve]}{\mye[\tm]}$$

 Call
$$\delta \eqdef - \frac{\mye[\tm\tve]}{\mye[\tm]}$$
the pricing error.

 Assume there is an infinite number of assets $i = 1, 2, \ldots$ and an SDF.  We can show that
$\sum_{i=1}^\infty \delta_i^2 < \infty$.

 Thus, for any given number $\epsilon>0$, most (all but a finite number) of the pricing errors are small (satisfy
$|\delta_i|\ < \epsilon$).

\end{frame}


\end{document}

