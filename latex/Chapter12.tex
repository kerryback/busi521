\documentclass[10pt]{beamer}

\usetheme[progressbar=foot]{metropolis}
\usepackage{appendixnumberbeamer}

\usepackage{booktabs}
\usepackage[scale=2]{ccicons}

\usepackage{pgfplots}
\usepgfplotslibrary{dateplot}
\pgfplotsset{compat=1.18} 

\usepackage{xspace}
\usepackage{xcolor}

\DeclareMathOperator{\stdev}{stdev}
\DeclareMathOperator{\var}{var}
\DeclareMathOperator{\cov}{cov}
\DeclareMathOperator{\corr}{corr}
\DeclareMathOperator{\prob}{prob}
\DeclareMathOperator{\n}{n}
\DeclareMathOperator{\N}{N}
\DeclareMathOperator{\Cov}{Cov}

\newcommand{\hlf}{\frac{1}{2}}
\newcommand{\bi}{\begin{itemize}}
\newcommand{\ei}{\end{itemize}}
\newcommand{\im}{\item}
\newcommand{\D}{\mathrm{d}}
\newcommand{\E}{\mathrm{e}}
\newcommand{\mye}{\ensuremath{\mathsf{E}}}
\newcommand{\myreal}{\ensuremath{\mathbb{R}}}
\newcommand{\bq}{\begin{equation}}
\newcommand{\eq}{\end{equation}}
\newcommand{\eqdef}{\;\buildrel \text{d{}ef}\over = \;}
\newcommand{\xstar}{\buildrel *\over X}
\newcommand{\pmax}{p^{\text{max}}}
\newcommand{\qmax}{q^{\text{max}}}
\newcommand{\bfr}{\begin{frame}}
\newcommand{\bfrp}{\begin{frame}[plain]}
\newcommand{\efr}{\end{frame}}
\newcommand{\F}{\mathcal{F}}
\newcommand{\FF}{\mathbb{F}}
\newcommand{\ve}{\varepsilon}
\newcommand{\lh}{\hat{\lambda}}
\definecolor{mycolor}{gray}{0.8}
\definecolor{mymaincolor}{rgb}{0.6862745098039216,0.9333333333333333,0.9333333333333333}
\newcommand{\alr}[1]{\textcolor{blue}{#1}}
\definecolor{LightCyan}{rgb}{0.88,1,1}
\newcommand{\yel}{\cellcolor{yellow}}
\newcommand{\blue}{\cellcolor{SkyBlue}}
\newcommand{\gr}{\cellcolor{SpringGreen}}
\newcommand{\pink}{\cellcolor{pink}}
\newcommand{\apr}{\cellcolor{Apricot}}
\newcommand{\tve}{\tilde{\varepsilon}}
\newcommand{\tw}{\tilde{w}}
\newcommand{\ttth}{\tilde{\theta}}
\newcommand{\te}{\tilde{e}}
\newcommand{\ts}{\tilde{s}}
\newcommand{\tx}{\tilde{x}}
\newcommand{\ty}{\tilde{y}}
\newcommand{\tv}{\tilde{v}}
\newcommand{\tp}{\tilde{p}}
\newcommand{\tF}{\tilde{F}}
\newcommand{\tf}{\tilde{f}}
\newcommand{\tZ}{\tilde{Z}}
\newcommand{\ow}{\overline{w}}
\newcommand{\tm}{\tilde{m}}
\newcommand{\tc}{\tilde{c}}
\newcommand{\tz}{\tilde{z}}
\newcommand{\tr}{\widetilde{R}}
\newcommand{\tR}{\widetilde{\mathbf{R}}}
\newcommand{\bms}{\begin{multline*}}
\newcommand{\ems}{\end{multline*}}
\newcommand{\bas}{\begin{align*}}
\newcommand{\eas}{\end{align*}}
\newcommand{\qr}{\mathbb{Q}}
\newcommand{\tX}{\tilde{X}}
\newcommand{\tY}{\tilde{Y}}

\setbeamertemplate{frame footer}{BUSI 521/ECON 505, Spring 2024}

\title{Chapter 12: Brownian Motion and Stochastic Calculus}

\date{}
\author{Kerry Back\\ 
BUSI 521/ECON 505\\
Rice University}


\begin{document}

\maketitle

\section{Preliminaries}\subsection{}

\begin{frame}{Review: Discrete-Time Martingales}
          \bi 
        \im A martingale is a sequence of random variables $Y$ such that $Y_s = \mye_s[Y_t]$ for all $s<t$.
        \im Equivalently, $E_s[Y_t-Y_s] = 0$.
        \im Consider any payoff at date $u$ with value $W_t$ at date $t$.  Then
        \begin{enumerate}
            \im The sequence $M_tW_t$ is a martingale (up to $u$).
            \im The sequence
            $$\frac{W_t}{(1+r_{f1}) \cdots (1+r_{ft})}$$
            is a $Q$-martingale.
        \end{enumerate}
        \im This holds for any self-financing wealth process $W$, meaning that no money is taken out or in after date 0 -- e.g., a dividend-reinvested asset price.
        \ei
\end{frame}

\bfr\frametitle{Continuous-Time Model of a Stock Price}
\bi
\im Notation: 
$S=\,$ stock price, $B = \,$ Brownian motion, $\mu$ and $\sigma$ are constants or stochastic processes.
\im Stock price model:
$$\frac{\D S}{S}=\mu\,\D t+\sigma\,\D B$$
\im 
$\mu\,\D t =\,$ expected rate of return, $\sigma\,\D B=\,$ risk
\im Our goal is to understand what equations like this mean and to learn how to work with them.
\im The first task is to explain Brownian motion.
\ei
\end{frame}

\bfr\frametitle{Stochastic Process}
\bi
\im A stochastic process $X$ in continuous time is a collection of random variables  $X_t$ for $t\in [0,\infty)$ or for $t\in[0,T]$.  
\im The state of the world $\omega$ determines the value $X_t(\omega)$ at each time~$t$.  
\im A stochastic process can be viewed as a random function of time $t \mapsto X_t(\omega)$.  
\im For a given $\omega$, the function of time is called a path of the stochastic process.
\ei
\end{frame}

\section{Brownian Motion}
\subsection{}

\bfr\frametitle{Brownian Motion}
\bi
\im A Brownian motion \index{Brownian motion} is a continuous-time stochastic process~$B$ with the property that, for any dates $t<u$, and \alert{conditional on information at date~$t$, the change $B_u-B_t$ is normally distributed with mean zero and variance $u-t$}. 
\im Equivalently, \alert{$B_u$ is conditionally normally distributed with mean $B_t$ and variance $u-t$}.  In particular, the distribution of $B_u-B_t$ is the same for any conditioning information and hence is independent of conditioning information.  This is expressed by saying that the Brownian motion has \alert{independent increments}.  
\im We can regard \alert{$\Delta B = B_u-B_t$ as noise that is unpredictable by any date--$t$ information}.  The starting value of a Brownian motion is typically not important, because only the increments $\Delta B$ are usually used to define the randomness in a model, so we can and will take $B_0=0$.
\ei
\end{frame}

\bfr\frametitle{Brownian Motion and Information}
\bi
\im A Brownian motion with respect to some information might not be a Brownian motion with respect to other information.  
\im For example, a stochastic process could be a Brownian motion for some investors but not for better informed investors, who might be able to predict the increments to some degree.  
\im It is part of the definition of a Brownian motion that the past values $B_s$ for $s \leq t$ are part of the information  at each date~$t$.
\ei
\end{frame}

\bfr\frametitle{Continuous Nondifferentiable Paths}
\bi
\im The paths of a Brownian motion make many small up-and-down movements with extremely high frequency, so that the limits $\lim_{s\rightarrow t} (B_t-B_s)/(t-s)$ defining derivatives do not exist.  
\im With probability 1, a path of a Brownian motion is
\bi
\im continuous
\im almost everywhere \alert{nondifferentiable}
\ei
\im The name ``Brownian motion'' stems from the  observations by the botanist Robert Brown of the erratic behavior of particles suspended in a fluid.  
\ei
\end{frame}

\bfr\frametitle{Quadratic Variation of Brownian Paths}
\bi
\im Let~$B$ be a Brownian motion.  Consider a  discrete partition
$$s=t_0 < t_1 < t_2 < \cdots < t_N=u$$
of a time interval $[s,u]$.
\im Consider the sum of squared changes
$$\sum_{i=1}^N (B_{t_i}-B_{t_{i-1}})^2$$
in some state of the world.
\im If we consider finer partitions (i.e., increase~$N$) with the maximum length $t_i-t_{i-1}$ of the time intervals  going to zero as $N \rightarrow \infty$, the limit of the sum is called the quadratic variation of the path of~$B$.  
\im The quadratic variation of the path of a Brownian motion over any interval $[s,u]$ is equal to $u-s$ with probability~1.
\ei
\end{frame}


\bfr\frametitle{Quadratic Variation of Usual Functions of Time}
\bi
\im The quadratic variation of any continuously differentiable function is zero.  
\im Consider, for example, a
linear function of time: $f_t = at$ for some constant~$a$.  
\im Taking $t_i-t_{i-1} = \Delta t = (u-s)/N$ for each~$i$, the sum of 
squared changes over an interval $[s,u]$ is
$$\sum_{i=1}^N (f_{t_i}-f_{t_{i-1}})^2 = \sum_{i=1}^N  (a\,\Delta t)^2  = Na^2 \left(\frac{u-s}{N}\right)^2 = \frac{a^2(u-s)^2}{N} \rightarrow~0$$
as $N \rightarrow \infty$.  
\ei
\end{frame}


\bfr\frametitle{Total Variation of Brownian Paths}
\bi
\im Total variation is defined in the same way as quadratic variation but with the squared changes  replaced by the absolute values of the changes.   
\im Brownian paths have infinite total variation (with probability~1).
\bi
\im In general, for continuous functions, finite total variation $\,\Rightarrow\,$ zero quadratic variation.
\im So, nonzero quadratic variation $\,\Rightarrow\,$ infinite total variation.
\ei
\im Infinite total variation means that if we were to straighten out a path of a Brownian motion to measure it, its length would be infinite.  This is true no matter how small the time period over which we measure the path.
\ei
\end{frame}

\section{Martingales}
\subsection{}

\bfr\frametitle{Continuous Martingales}
\bi
\im A martingale \index{martingale!continuous martingale} is a stochastic process~$X$ with the property that $\mye_t[X_u]= X_t$ for each $t<u$ (equivalently, $\mye_t[X_u-X_t] = 0$). 
\bi
\im In discrete time, if $M$ is an SDF process and $W$ is a self-financing wealth process, then $MW$ is a martingale.
\ei
\im A continuous martingale is a martingale for which all of the paths are continuous (up to a null set).  
\im Every continuous martingale that is not constant has infinite total variation.
\ei
\end{frame}

\bfr\frametitle{Levy's Theorem}
\bi
\im Aa continuous martingale is a Brownian motion if and only if its quadratic variation over each interval $[s,u]$ equals $u-s$.
\im Thus, if a stochastic process has (i) continuous paths, (ii) conditionally mean-zero increments, and (iii) quadratic variation over each interval equal to the length of the interval, then its increments must also be \bi
\im (iv) independent of conditioning information and 
\im (v) normally distributed.
\ei
\im It is possible to deform the time scale (speeding up or slowing down the clock) to convert any continuous martingale into a Brownian motion.
\im Also, we can form continuous martingales from Brownian motions via stochastic integrals.
\ei
\end{frame}

\section{It\^o Integral}
\subsection{}

\bfr\frametitle{Stochastic Integrals}
If~$\theta$ is a stochastic process adapted to the information with respect to which~$B$ is a Brownian motion, is jointly measurable in $(t,\omega)$, and satisfies
$$
\int_0^T \theta_t^2\,\D t < \infty
$$
with probability~1, and if $M_0$ is a constant,
then we can define the stochastic process
$$
M_t = M_0+\int_0^t \theta_s\,\D B_s
$$
for $t \in [0,T]$.  This is called an It\^o integral \index{It\^o integral} or stochastic integral.

\end{frame}

\bfr\frametitle{Approximating Stochastic Integrals}
For each~$t$, the stochastic integral can be approximated as (is a limit in probability of)
$$
\sum_{i=1}^N \theta_{t_{i-1}}(B_{t_i}-B_{t_{i-1}})
$$
given  discrete partitions
$$0=t_0 < t_1 < t_2 < \cdots < t_N=t$$
of the time interval $[0,t]$ with the maximum length $t_i-t_{i-1}$ of the time intervals going to zero as $N \rightarrow \infty$.   Note that~$\theta$ is evaluated in this sum at the beginning of each interval $[t_{i-1},t_i]$ over which the change in~$B$ is computed.
\end{frame}

\bfr\frametitle{Differential Form}
Given 
$$
M_t = M_0+\int_0^t \theta_s\,\D B_s
$$
 we write
$$
\D M_t = \theta_t\,\D B_t
$$ or, more simply, 
$$\D M = \theta\,\D B$$ 
We can define $M$ from the formula $\D M = \theta\,\D B$ and the initial condition $M_0$  by ``summing'' the changes $\D M$ as
$$M_t = M_0 + \int_0^t \D M_s =  M_0+\int_0^t \theta_s\,\D B_s\,.$$
\end{frame}



\bfr\frametitle{It\^o Process}
The sum of an ordinary integral and a stochastic integral is called an It\^o process. \index{It\^o process} Such a process has the form
$$
Y_t = Y_0 + \int_0^t \alpha_s\,\D s + \int_0^t \theta_s\,\D B_s\,,
$$
which is also written as
$$
\D Y_t = \alpha_t\,\D t + \theta_t\,\D B_t
$$
or, more simply, as $$\D Y = \alpha\,\D t+ \theta\,\D B$$
We recover $Y$ from this differential form by ``summing'' the changes $\D Y$ over time.
The process~$\alpha$ is called the \alert{drift} of~$Y$.
\end{frame}

\section{Returns}
\subsection{}

\bfr\frametitle{Asset Return}
\bi
\im Suppose that between dividend payments the price~$S$ of an asset satisfies
$$
\frac{\D S}{S} = \mu\,\D t + \sigma\,\D B
$$
for a Brownian motion~$B$ and stochastic processes (or constants)~$\mu$ and~$\sigma$.  
\im We interpret $\D S/S$ as the instantaneous rate of return of the asset and $\mu\,\D t$ as the expected rate of return.  
\im The equation for $S$ can be written equivalently as $\D S = S\mu\,\D t + S\sigma\,\D B$.  
\im The real meaning is the ``summed'' version:
$$ S_u = S_0 + \int_0^u S_t\mu_t\,\D t + \int_0^u S_t\sigma_t\,\D B_t
$$
\ei
\end{frame}

\bfr\frametitle{Money Market Account}
\bi
\im Suppose there is an asset that is locally risk-free, \index{locally risk-free asset} meaning that its price~$R$ satisfies
$$
\frac{\D R}{R} = r\,\D t
$$
for some~$r$ (which can be a stochastic process).  
\im This equation for $R$ can be solved explicitly as
$$R_u = R_0\exp\left(\int_0^u r_t\,\D t\right)\,.$$
\im 
We interpret $r_t$ as the interest rate at date~$t$ for an investment during the infinitesimal period $(t,t+\D t)$.  
\im If the interest rate is constant, then
$R_u=R_0\E^{ru}$, 
meaning that interest is continuously compounded at the constant rate~$r$.  
\im We call~$r$ the instantaneous risk-free rate or the locally risk-free rate \index{locally risk-free rate}or the short rate.
\ei
\end{frame}



\bfr\frametitle{Portfolio Return}
\bi
\im A portfolio of the asset with price $S$ (the risky asset) and the money market account is defined by the fraction $\pi_t$ of wealth invested in the risky asset at each date~$t$.  
\im If no funds are invested or withdrawn from the portfolio during a time period $[0,T]$ and the asset does not pay dividends during the period, then  the wealth
process~$W$ satisfies
$$
\frac{\D W}{W} = (1-\pi)r\,\D t + \pi \frac{\D S}{S}
$$
\im This is called the intertemporal budget constraint.  It states that wealth grows only from interest earned and from the return on the risky asset.

\ei
\end{frame}

\bfr\frametitle{Intertemporal Budget Constraint}
The intertemporal budget constraint with no labor income and no consumption is
\begin{align*}
\frac{\D W}{W} &= (1-\pi)r\,\D t + \pi \frac{\D S}{S}\\
&= (1-\pi)r\,\D t + \pi \mu\,\D t + \pi\sigma\,\D B\\
&= r\,\D t + \pi(\mu-r)\,\D t + \pi\sigma\,\D B
\end{align*}
We can also write it as
$$
\D W = rW\,\D t + \pi(\mu-r)W\,\D t + \pi\sigma W\,\D B
$$
With labor income $Y$ and consumption $C$ (both as rate per unit time), it is
$$
\D W = rW\,\D t + \pi(\mu-r)W\,\D t + \pi\sigma W\,\D B + Y\,\D t - C\,\D t
$$
\end{frame}

\section{It\^o's Formula}
\subsection{}

\bfr\frametitle{Notation for Quadratic Variation}
\bi
\im Convenient notation:  $(\D B)^2 = \D t$.
\im The motivation comes from quadratic variation.  Consider discrete partitions
$$s=t_0 < t_1 < t_2 < \cdots < t_N=u$$
of a time interval $[s,u]$.  
\im With $N \rightarrow \infty$ and the maximum length $t_i-t_{i-1}$ of the time intervals  going to zero, 
\begin{align*}
\sum_{i=1}^N (B_{t_i}-B_{t_{i-1}})^2  & = \sum_{i=1}^N (\Delta B)^2\\
\rightarrow \int_s^u (\D B)^2
= \int_s^u \D t = u-s
\end{align*}
\ei
\end{frame}

\bfr\frametitle{Quadratic Variation of a Stochastic Integral}
The quadratic variation of a stochastic integral $\D M_t = \theta_t\,\D B_t$

over an interval $[s,u]$ is
$$\int_s^u (\D M_t)^2 = \int_s^u (\theta_t\,\D B_t)^2 = \int_s^u (\theta_t)^2 (\D B_t)^2 = \int_s^u \theta_t^2\,\D t$$

\end{frame}

\bfr\frametitle{Quadratic Variation of an It\^o Process}
\bi
\im More convenient notation: $(\D t)^2 = 0$, $(\D B)(\D t) = 0$.  
\im The motivation for $(\D t)^2=0$ is that the quadratic variation of a continuously differentiable function of time is zero.
\im The quadratic variation of an It\^o process $\D X_t = \alpha_t\,\D t + \theta_t\,\D B_t$ 
over an interval $[s,u]$ is
$$\int_s^u (\D X_t)^2 = \int_s^u (\alpha_t\,\D t + \theta_t\,\D B_t)^2 = \int_s^u (\theta_t)^2 (\D B_t)^2 = \int_s^u \theta_t^2\,\D t$$
\ei
\end{frame}

\bfr\frametitle{Variance and Quadratic Variation in Discrete Time}
\bi
\im Suppose $M$ is a martingale in discrete time.  Define $X$ to be the changes in $M$:
$$X_1 = M_1-M_0, \quad X_2 = M_2 - M_1, \quad X_3 = M_3-M_2, \quad \ldots$$
\im The process $X$ is called a martingale difference series.  It is serially uncorrelated.
\im Proof: for $t<u$,
$$\cov(X_t,X_u) = \mye[X_tX_u] = \mye\bigg[\mye_t[X_tX_u]\bigg] = \mye\bigg[X_t \mye_t[X_u]\bigg] = 0$$
\im The variance of $M_t$ is
$$\var(M_t) = \var(M_0 + X_1 + X_2 + \cdots + X_t) = \sum_{i=1}^t \var(X_i) = \mye\left[\sum_{i=1}^t X_i^2\right]$$
\ei
\end{frame}


\bfr\frametitle{Chain Rule of Ordinary Calculus}
\bi

\im Define $y=f(x)$ for some continuously differentiable function~$f$, so 
$$
\D y = f'(x)\,\D x$$
\im Now let $x$ be a nonrandom continuously differentiable function of time and define $y_t=f(x_t)$.  The chain rule gives us
$$\frac{\D y_t}{\D t} = f'(x_t)\frac{\D x_t}{\D t} \quad \Leftrightarrow \quad \D y_t = f'(x_t)\,\D x_t$$
\im 
The fundamental theorem of calculus states that we can ``sum'' the changes over an interval $[0,t]$ to obtain
$$
y_t = y_0 + \int_0^t f'(x_s)\,\D x_s\,.
$$
Of course, we can  substitute $\D x_s = x'_s\,\D s$ in this integral.

\ei
\end{frame}

\bfr\frametitle{Chain Rule from Multivariate Calculus}
\bi
\im Define $y=f(t,x)$, so
$$\D y = \frac{\partial f}{\partial t}\,\D t + \frac{\partial f}{\partial x}\,\D x$$
\im Now let $x$ be a nonrandom continuously differentiable function of time and define $y_t=f(t,x_t)$.  The chain rule gives us
$$\frac{\D y}{\D t} = \frac{\partial f}{\partial t} + \frac{\partial f}{\partial x}\,\frac{\D x}{\D t} \quad \Leftrightarrow \quad \D y_t = \frac{\partial f}{\partial t}\,\D t + \frac{\partial f}{\partial x}\,\D x_t$$
\im 
This implies
$$
y_t = y_0 + \int_0^t  \frac{\partial f(s,x_s)}{\partial s}\,\D s + \int_0^t  \frac{\partial f(s,x_s)}{\partial x}\,\D x_s
$$
Of course, we can  substitute $\D x_s = x'_s\,\D s$ in this integral.

\ei
\end{frame}

\bfr\frametitle{It\^o's Formula}
\bi
\im  Let $f(t,x)$ be continuously differentiable in~$t$ and twice continuously differentiable in~$x$.  
\im Define $Y_t = f(t,B_t)$ for a Brownian motion~$B$.
\im  
It\^o's formula states that
$$
\D Y = \frac{\partial f}{\partial t}\,\D t + \frac{1}{2}\frac{\partial^2 f}{\partial B^2}\,\D t +\frac{\partial f}{\partial B}\,\D B 
$$
\im Thus, $Y$ is an It\^o process with
$$\frac{\partial f}{\partial t} + \frac{1}{2}\frac{\partial^2 f}{\partial B^2}$$
as its drift and
$(\partial f/\partial B)\,\D B$ as its stochastic part.  
\im It\^o's formula means that, for each~$t$,
$$
Y_t = Y_0+\int_0^t \left(\frac{\partial f(s,B_s)}{\partial s}+ \frac{1}{2}\frac{\partial^2 f(s,B_s)}{\partial B^2}\right)\,\D s + \int_0^t \frac{\partial f(s,B_s)}{\partial B}\,\D B_s
$$
\ei
\end{frame}

\bfr\frametitle{It\^o's Formula cont.}
\bi
\im  Recall our notation $(\D B)^2 = \D t$.
\im In terms of this notation, It\^o's formula is
$$
\D Y = \frac{\partial f}{\partial t}\,\D t +\frac{\partial f}{\partial B}\,\D B + \frac{1}{2}\frac{\partial^2 f}{\partial B^2}\,(\D B)^2
$$
\ei
\end{frame}


\bfr\frametitle{Example of It\^o's Formula}
\bi
\im  Let $Y_t=B_t^2$, so $Y_t = f(B_t)$ where $f(x)=x^2$.  
\im Apply It\^o's formula.  Using the notation $(\D B)^2= \D t$, we have
\begin{align*}
\D Y &= f'(B_t)\,\D B + \frac{1}{2}f''(B_t)\,(\D B)^2 \notag\\
&= 2 B_t\,\D B_t + (\D B)^2
\end{align*}
\im Compare this to discrete changes. Consider the increment $\Delta Y = Y_u-Y_s$ over an interval $[s,u]$.  Set $\Delta B = B_u-B_s$.
\im We have
\begin{align*}
\Delta Y &= B_u^2 - B_s^2\notag\\
&= [B_s + \Delta B]^2 - B_s^2\notag\\
&= 2 B_s\Delta B + (\Delta B)^2
\end{align*}
\ei
\end{frame}


\bfr\frametitle{It\^o's Formula for Functions of It\^o Processes}
\bi
\im  Let $X$ be an It\^o process: $\D X = \alpha\,\D t + \theta\,\D B$.
\im Recall our notation: $(\D t)^2=0$, $(\D t)(\D B) = 0$, $(\D B)^2 = \D t$.  
\im Recall
$$(\D X)^2 = (\alpha\,\D t + \theta\,\D B)^2 = \theta^2\,\D t$$
\im Let $f(t,x)$ be continuously differentiable in~$t$ and twice continuously differentiable in~$x$.  
\im Define $Y_t = f(t,X_t)$.
\im It\^o's formula is:
\begin{align*}
\D Y &=  \frac{\partial f}{\partial t}\,\D t +\frac{\partial f}{\partial X}\,\D X + \frac{1}{2}\frac{\partial^2 f}{\partial X^2}\,(\D X)^2\\
&=\frac{\partial f}{\partial t}\,\D t +\frac{\partial f}{\partial X}(\alpha\,\D t + \theta\,\D B) + \frac{1}{2}\frac{\partial^2 f}{\partial X^2}\theta^2\,\D t
\end{align*}
\ei
\end{frame}

\section{GBM}\subsection{}

\bfr\frametitle{Geometric Brownian Motion}
\bi
\im  Suppose, for constants $\mu$ and $\sigma$, that
$$\frac{\D S}{S}=\mu\,\D t+\sigma\,\D B$$
\im We will solve this like we solved for the price of the money market account.  
\im Define $Y_t = \log S_t$.  The process $S$ is an It\^o process, so we can apply It\^o's formula to $Y$ to obtain
\begin{align*}
\D \log S &= \frac{1}{S}\,\D S + \frac{1}{2}\cdot \left(-\frac{1}{S^2}\right)(\D S)^2\\
&= \mu\,\D t + \sigma\,\D B - \frac{1}{2}\sigma^2\,\D t
\end{align*}
\ei
\end{frame}

\bfr\frametitle{Geometric Brownian Motion cont.}
\bi
\im Summing the changes gives
$$\log S_t = \log S_0 + \left(\mu - \frac{1}{2}\sigma^2\right)t + \sigma B_t$$
\im Exponentiating both sides gives
$$S_t = S_0 \E^{\mu t - \sigma^2t/2 + \sigma B_t}$$
\im This is the solution of the equation
$$\frac{\D S}{S}=\mu\,\D t+\sigma\,\D B$$
\ei
\end{frame}

\section{Multivariate}
\subsection{}

\bfr\frametitle{Covariation (Joint Variation)}
\bi
\im  Consider a  discrete partition
$s=t_0 < t_1 < t_2 < \cdots < t_N=u$
of a time interval $[s,u]$.
\im For any two functions of time~$x$ and~$y$, consider the sum of products of changes
$$\sum_{i=1}^N \Delta x_{t_i}\Delta y_{t_i}\,,$$
where $\Delta x_{t_i} = x_{t_i}-x_{t_{i-1}}$ and $\Delta y_{t_i} = y_{t_i}-y_{t_{i-1}}$.
\im The covariation (or joint variation)  of~$x$ and~$y$ on the interval $[s,u]$ is defined as the limit of this sum as $N \rightarrow \infty$ and the lengths $t_i-t_{i-1}$ of the intervals  go to zero.  
\im If $x=y$, then this is the same as the quadratic variation.  
\im If both functions are continuous and one is continuously differentiable, then the covariation is zero.
\ei
\end{frame}


\bfr\frametitle{Covariation of Brownian Motions}
\bi
\im  If $B_1$ and $B_2$ are Brownian motions, then there is a process $\rho$ with $|\rho_t|\leq 1$ for all $t$, such that, with probability~1, the covariation of the paths of $B_1$ and $B_2$ over any interval $[s,u]$ equals
$$\int_s^u \rho_t\,\D t$$
\im The Brownian motions are independent if and only if $\rho \equiv 0$. 
\im We write $(\D B_1)(\D B_2) = \rho\,\D t$.
\im Then we can ``calculate'' the covariation as the sum of products of changes:
$$\int_s^u (\D B_{1t})(\D B_{2t})$$

\ei
\end{frame}




\bfr\frametitle{Covariation of It\^o Processes}
\bi
\im  Consider two It\^o processes $\D X_i = \alpha_i\,\D t + \theta_i\,\D B_i$.  
\im The covariation of $X_1$ and $X_2$ over any interval $[s,u]$ is
$$\int_s^u (\D X_{1t})\,(\D X_{2t})$$
\im Here,
\begin{align*}
(\D X_{1t})\,(\D X_{2t}) &= (\alpha_{1t}\,\D t + \theta_{1t}\,\D B_{1t})(\alpha_{2t}\,\D t + \theta_{2t}\,\D B_{1t})\\
&= \theta_{1t}\theta_{2t}(\D B_{1t})(\D B_{2t})\\
& = \theta_{1t}\theta_{2t}\rho_t\,\D t
\end{align*}
where $\rho$ is the correlation process of the two Brownian motions.
\im We also call $\rho$ the correlation process of the two It\^o processes.
\ei
\end{frame}


\bfr\frametitle{General It\^o's Formula}
\bi
\im  Consider $n$ It\^o processes $\D X_i = \alpha_i\,\D t + \theta_i\,\D B_i$.
\im Suppose $(t,x) \mapsto f(t,x):[0,\infty) \times \myreal^n \rightarrow \myreal$ is continuously differentiable in $t$ and twice continuously differentiable in $x$.
\im Define $Y_t = f(t,X_{1t}, \ldots, X_{nt})$.
\im Then
$$\D Y = \frac{\partial f}{\partial t}\,\D t + \sum_{i=1}^n \frac{\partial f}{\partial X_i}\,\D X_i + \frac{1}{2}\sum_{i=1}^n \sum_{j=1}^n \frac{\partial^2 f}{\partial X_i \partial X_j}\,(\D X_i)\,(\D X_j)$$
\im For example, if $n=2$, then
\begin{multline*}
\D Y = \frac{\partial f}{\partial t}\,\D t + \frac{\partial f}{\partial X_1}\,\D X_1 + \frac{\partial f}{\partial X_2}\,\D X_2 \\+ \frac{1}{2}\frac{\partial^2 f}{\partial X_1^2 }\,(\D X_1)^2 + \frac{1}{2}\frac{\partial^2 f}{\partial X_2^2 }\,(\D X_2)^2 + \frac{\partial^2 f}{\partial X_1 \partial X_2}\,(\D X_1)\,(\D X_2)
\end{multline*}
\ei
\end{frame}


\bfr\frametitle{Product Rule (Integration by Parts)}
\bi
\im  Suppose $X_1$ and $X_2$ are It\^o processes and $Y_t = X_{1t}X_{2t}$.
\im To calculate $\D Y$, we apply It\^o's formula with $n=2$ and $f(t,x_1,x_2) = x_1x_2$.
\im We obtain
$$\D Y = X_1\,\D X_2 + X_2\,\D X_1 + (\D X_1)(\D X_2)$$
\ei
\end{frame}

\end{document}

\section{Example}
\subsection{}

\bfr\frametitle{Exercise 12.5}
Solve the following stochastic differential equation for $X$, where $\kappa$, $\theta$, and $\sigma$ are constants:
$$\D X  = \kappa(\theta-X)\,\D t + \sigma\,\D B$$

Solution: Define $Y=X-\theta$, so
$$\D Y  = -\kappa Y\,\D t + \sigma\,\D B$$
We will solve for $Y$ and then define $X$ as $X=Y+\theta$.

Intuition: the term $-\kappa Y\,\D t$ creates exponential decay towards~0.  The initial value $Y_0$ decays towards 0 and each of the shocks $\sigma\,\D B$ decays towards zero.  This suggests
\begin{align*}
 Y_t &= \E^{-\kappa t}Y_0 + \int_0^t \E^{-\kappa(t-s)}\sigma\,\D B_s\\
 \Rightarrow \quad X_t &= \theta + \E^{-\kappa t}(X_0-\theta) + \int_0^t \E^{-\kappa(t-s)}\sigma\,\D B_s
\end{align*}

\end{frame}

\bfr\frametitle{Verification of the Solution}
Define 
$$Z_t = \int_0^t \E^{\kappa s}\sigma\,\D B_s \quad \Rightarrow \quad \D Z_t = \E^{\kappa t}\sigma\,\D B_t$$

The proposed solution is
$$Y_t = \E^{-\kappa t}Y_0 + \E^{-\kappa t}Z_t$$

To calculate $\D Y$, use the integration by parts formula for the second term.  This produces
\begin{align*}
\D Y &= -\kappa \E^{-\kappa t}Y_0\,\D t - \kappa \E^{-\kappa t}Z_t\,\D t + \E^{-\kappa t}\,\D Z_t\\
&= - \kappa Y_t\,\D t + \sigma\,\D B_t
\end{align*}
\end{frame}


\section{Local Marts}
\subsection{}

\bfr\frametitle{Local Martingales}
\bi
\im A stochastic integral $\D M = \theta\,\D B$ is a local martingale.  Martingales are a subset of the class of local martingales.  A stochastic integral might or might not be a martingale.
\im ``Local'' means that it is like a martingale at each instant in time (its expected change is zero).  But it still might not be a martingale.
\ei
\end{frame}

\bfr\frametitle{Example: Doubling Strategy}
\bi
\im Suppose we flip a fair coin at an infinite number of dates $t=0$, $t=1/2$, $t=3/4$, $t=4/5$, $t=5/6$, \ldots.  
\im We stop when I win.  We bet \$1 on the first flip.  If I lose, we bet \$2 on the next flip.  If I lose, we bet \$4 on the next flip, etc.
\im Define my wealth as $W_0=0$ with $W$ constant between flips and $W$ constant after we stop.
\im The wealth process $W$ is a local martingale, because the coin is fair.  My expected change in wealth is 0 at each flip.
\im But I win \$1 with probability 1, due to the law of large numbers. Hence, $W_0=0$ and $W_1 = 1$ with probability 1, so $W_0 \neq \mye[W_1]$.
\ei
\end{frame}

\bfr\frametitle{When Is a Stochastic Integral a Martingale?}
\bi
\im A sufficient condition for a stochastic integral $\D M = \theta\,\D B$ to be a martingale (with finite variance) on an interval $[0,T]$ is that the expected quadratic variation be finite:
$$\mye \int_0^T \theta_t^2\,\D t < \infty$$
In this case, the variance of $M_T$ is the expected quadratic variation.
\im In general, there are three possibilities:
\bi
\im The expected quadratic variation is finite, and $M$ is a finite variance martingale.
\im The expected quadratic variation is infinite, and $M$ is a martingale with infinite variance.
\im The expected quadratic variation is infinite, and $M$ is not a martingale.
\ei 
\ei
\end{frame}

\bfr\frametitle{When Is an It\^o Process a Martingale?}
\bi
\im An It\^o process $\D X_t = \alpha_t\,\D t + \theta_t\,\D B_t$ is a local martingale if and only if $\alpha_t = 0$ almost everywhere with probability~1.
\im If $\alpha=0$, then the It\^o process is a stochastic integral, and the previous slide applies.
\im The condition 
$$X = \text{local martingale} \quad \Leftrightarrow \quad \alpha=0$$
is the source of partial differential equations that are important in finance.
\ei
\end{frame}

\section{Mart Repr}
\subsection{}

\bfr\frametitle{Martingale Representation}
\bi
\im Suppose $M$ is a local martingale adapted to a Brownian motion $B$ (adapted means that, for each $t$, $M_t$ depends only on the history of $B$ up to time $t$).
\im Then there exists a stochastic process $\theta$ such that $\D M = \theta\,\D B$.
\im This extends to multiple Brownian motions.  If $M$ is adapted to a vector of Brownian motions $(B_1,\ldots,B_k)$, then there exist $\theta_1,\ldots,\theta_k$ such that
$$\D M = \theta_1\,\D B_1 + \cdots + \theta_k\,\D B_k$$
\im This is a spanning property and underlies many important results in finance.
\ei
\end{frame}


\end{document}

\end{document}

\bfr\frametitle{Quadratic Variation of a Stochastic Integral}
The stochastic process~$M$  defined by 
$$
M_t = M_0+\int_0^t \theta_s\,\D B_s
$$
has continuous paths and has  quadratic variation equal to
$$
\int_t^u \theta_s^2\,\D s
$$
on each interval $[t,u]$.  Given a partition
$$t=t_0 < t_1 < t_2 < \cdots < t_N=u$$,
we have
$$\sum_{i=1}^N \bigg(M_{t_i}-M_{t_{i-1}}\bigg)^2 \approx \sum_{i=1}^N \bigg(\theta_{t_{i-1}}(B_{t_i}-B_{t_{i-1}})\bigg)^2$$
\end{frame}


\bfr\frametitle{Correlation Process}
\bi
\im $\rho$ is called the correlation process of the Brownian motions
\im The conditional covariance of increments $\Delta B_1 \eqdef B_{1u} - B_{1s}$ and $\Delta B_2 \eqdef B_{2u}-B_{2s}$ is
$$\mye_s \int_s^u \rho_t\,\D t$$
\im
 The conditional correlation of the increments is the average value of $\rho$ over the interval:
$$\frac{\cov(\Delta B_1,\Delta B_2)}{\sqrt{\var(\Delta B_1)\var(\Delta B_2)}} =  \mye_s \left[\frac{1}{u-s}\int_s^u \rho_t\,\D t\right]$$
\ei
\end{frame}

\bfr\frametitle{Solving the ODE for the Money Market Account}
\bi
\im We assume the price of the money market account satisfies
$$
\frac{\D R}{R} = r\,\D t
$$
for some~$r$ (which can be a stochastic process). 
We use ordinary calculus to solve this differential equation.  
\im First, note that
$$\frac{\D \log R}{\D R} = \frac{1}{R} \quad \Leftrightarrow \quad \D \log R = \frac{\D R}{R}$$
\im Thus, the price of the money market account satisfies
$$\D \log R = r\,\D t$$
\ei
\end{frame}

\bfr\frametitle{Solving the ODE for the Money Market Account cont.}
\bi
\im We compute $\log R$ by summing the changes:
$$\log R_t = \log R_0 + \int_0^t \D \log R_s = \log R_0 + \int_0^t r_s\,\D s$$
\im Then we exponentiate both sides:
$$ R_t = R_0\exp\left(\int_0^t r_s\,\D s\right)$$

\ei
\end{frame}

\end{document}